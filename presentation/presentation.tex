\documentclass{beamer}
%\usepackage[osf,biolinum]{libertine}
\usepackage{tgheros}
\usepackage[varqu]{inconsolata}
\usepackage{../packages/mylistings}
\usepackage{csquotes}

\usepackage{tikz}
\usepackage{tikz-uml}

%\usecolortheme{beaver}

\lstset{
  language=Scala,
  %style=blackwhite
  style=colored,
  %belowskip=0pt
  basicstyle=\ttfamily\small,
  columns=[c]fixed
}

\usetheme[compress]{Singapore}
\setbeamercolor{structure}{fg=textblue}


\author{Philipp Gabler}
\title{Designing Embedded\\ Domain-Specific Languages\\ in Scala\\
  {\small\itshape A Case Study with Action Systems}}
\date{2015-06-17}

%%%%%%%%%%%%%%%%%%%%%%%%%%%%%%%%%%%%%%%%%%%%%%%%%%%%%%%%%%%%%%%%%%%%%%%%%%%%%%%%%%%%%%%%%%%%%%%%%%%%
\begin{document}

\begin{frame}
  \maketitle
\end{frame}

\section{Introduction}
\subsection{Intro}

\begin{frame}
  \frametitle{Overview of the work}
  \begin{block}{Exploring Scala's capabilities for embedded DSLs}
    \begin{itemize}
    \item What helpful features and techniques are there?
    \item How (far) can they be used, and what are their limits?
    \item Some examples of their usage: explanatory and \enquote{in the wild}
    \end{itemize}
  \end{block}
\end{frame}

\begin{frame}
  \frametitle{Overview of the work}
  \begin{block}{Construct a DSL for the Action Systems used in model-based mutation testing}
    \begin{itemize}
    \item How does the process of development with this goal look like?
    \item Where does the language help with the embedding, and where does is stand in our way?
    \item What features are missing, and what could have been done better (in other languages, or
      with different means)?
    \end{itemize}
  \end{block}
\end{frame}

\section{Scala}

\subsection{Intro}

\begin{frame}
  \frametitle{\enquote{Regular} language features}
  \begin{itemize}
  \item Completely object oriented (no strange \enquote{primitive types})
  \item Functional: immutability preferred, proper closures and lambdas
  \item Powerful type system: bounded polymorphism done right, with higher kinds; structural types
  \item Traits: better interfaces, mixin inheritance
  \item Sophisticated modules, ADTs + pattern matching, monad comprehensions,\ldots
  \end{itemize}
\end{frame}

\begin{frame}
  \frametitle{Not so regular features (a selection)}
  \begin{itemize}
  \item Operators: Inline calls and proper expressions
  \item Empowered function calling: blocks and by-name args
  \item Implicits: Extension wrappers and less boilerplate
  \item Extractors (pattern synonyms/active patterns), interpolators, macros\ldots
  \end{itemize}
\end{frame}


\subsection{Operators}

\begin{frame}[containsverbatim]
  \lstset{gobble=6}
  \frametitle{Expressions that look like expressions}
  \begin{block}{Java}
    \begin{lstlisting}[language=Java]
      // not so logical...
      Expr e1 = new Or(new Var("a"), 
                       new And(new Var("b"), 
                               new Not(new Var("c"))));
    \end{lstlisting}
  \end{block}
  \begin{block}{Scala}
    \begin{lstlisting}
      // as simple as that!
      val e1: Expr = "a" || "b" && !"c"
    \end{lstlisting}
  \end{block}
\end{frame}

\begin{frame}[containsverbatim]
  \lstset{gobble=4}
  \frametitle{Expressions that look like expressions}
  
  The magic behind it:
  \begin{lstlisting}
    def ||(other: Expr) = Or(this, other)
    def &&(other: Expr) = And(this, other)
    def unary_! = Not(this)
    implicit def stringToExpr(s: String): Var = Var(s)
  \end{lstlisting}
  Also not much more code than Java variant.
\end{frame}

\begin{frame}[containsverbatim]
  \lstset{gobble=4}
  \frametitle{Combinators!}
  \begin{lstlisting}
    def identifier: Parser[String] = "[^()' ]+".r
    def readMacroIdentifier: Parser[String] = "'"

    def atom = (identifier ^^ Atom) <~ whiteSpace.?

    def cons =  
      (parenthesized(sexpr.*) ^^ ConsList) <~ whiteSpace.?

    def readMacro = (readMacroIdentifier ~ sexpr) <~ 
      whiteSpace.? ^^ { case s~e => ReadMacro(s, e) }

    def sexpr = whiteSpace.? ~> (readMacro | cons | atom)
  \end{lstlisting}
\end{frame}

\begin{frame}[containsverbatim]
  \lstset{gobble=4} \frametitle{\enquote{Natural language
      interface}\footnote{\scriptsize\protect\url{http://www.scalatest.org/}}}
  \begin{lstlisting}
    class ExampleSpec extends FlatSpec with Matchers {
      "A Stack" should "behave right" in {
        val stack = new Stack[Int]
        stack.push(1)
        stack.push(2)
        stack.pop() should be (2)
        stack.pop() should be (1)
      }

      it should "throw NoSuchElementException" in {
        val emptyStack = new Stack[Int]
        a [NoSuchElementException] should be thrownBy {
          emptyStack.pop()
        } 
      }
    }
  \end{lstlisting}
\end{frame}


\subsection{Blocks}

\begin{frame}[containsverbatim]
  \lstset{gobble=6}
  \frametitle{Blocks \& by-name args I}
  \begin{block}{Defining this\ldots}
    \begin{lstlisting}[language=Java]
      def _while(condition: => Boolean)(body: => Unit): Unit = {
        if (condition) { body; _while(condition)(body) }
      }
    \end{lstlisting}
  \end{block}
  \begin{block}{\ldots we get this!}
    \begin{lstlisting}
      var x = 10
      _while(x > 0) {
        print(x)
        x -= 1
      }
      // 10987654321
    \end{lstlisting}
  \end{block}
\end{frame}

\begin{frame}[containsverbatim]
  \lstset{gobble=6}
  \frametitle{Blocks \& by-name args II}
  This has also very practical semantics:
  \begin{block}{Java}
    \begin{lstlisting}[language=Java]
      Socket socket = new Socket("example.com", 80);
      try { 
        socket.getOutputStream().write("GET".getBytes());
      } catch (IOException e) { ... }
      // what should we return? null?
    \end{lstlisting}
  \end{block}
  \begin{block}{Scala}
    \begin{lstlisting}
      val socket = Try(new Socket("example.com", 80))
      socket map { s => 
        s.getOutputStream write "GET".getBytes
      } recover { 
        case e: IOException => ??? 
      }
      // can simply return socket, or better: socket.toOption
    \end{lstlisting}
  \end{block}
\end{frame}

\begin{frame}[containsverbatim]
  \lstset{gobble=4}
  \frametitle{Blocks \& by-name args III}
  We even could implement this:
  \begin{lstlisting}
    on error in {
      socket.getOutputStream write "GET".getBytes
    } resume next
    // error-free code!
  \end{lstlisting}
\end{frame}


\subsection{Implicits}

\begin{frame}
  \frametitle{Implicits}
  Have you noticed them?
  
  No, because you shouldn't!
\end{frame}

\begin{frame}[containsverbatim]
  \lstset{gobble=4}
  \frametitle{Implicits}
  Patching strings:
  \begin{lstlisting}
    trait Read[T] { def read(s: String): T }

    implicit object boolIsRead extends Read[Boolean] {
      def read(s: String) = s match {
        case "true" => true
        case "false" => false
      }
    }

    implicit class StringReadOps(val self: String) {
      def readAs[T](implicit readT: Read[T]): T = 
        readT.read(self)
    }
  \end{lstlisting}

  No more helpers:
  \begin{lstlisting}
    "true".readAs[Boolean] 
    // becomes: StringReadOps("true").readAs[Int](boolIsRead)
    // which actually is: boolIsRead.read("true")
  \end{lstlisting}
\end{frame}

\begin{frame}[containsverbatim]
  \lstset{gobble=4}
  \frametitle{Implicits}
  We can even nest this:
  \begin{lstlisting}
    implicit def numericIsRead[N](implicit numN: Numeric[N])
      : Read[N] = new Read[N] {
      def read(s: String) = numN.fromInt(s.toInt)
    }
  \end{lstlisting}
  \begin{lstlisting}
    "42".readAs[Int]  
    // is actually: numericIsRead(intIsNumeric).read("42")
  \end{lstlisting}
\end{frame}

\begin{frame}[containsverbatim]
  \lstset{gobble=4}
  \frametitle{Implicits are present everywhere}
  \begin{lstlisting}[lineskip=1ex]
    val duration = 1.second + 42.millis
    Seq((1,3), (2,1), (1,2)).sorted
    42 + " is the answer!"
    sender ! Received(answer)
  \end{lstlisting}
  \begin{lstlisting}
    val f: Future[String] = Future {
      s + " future!"
    }
  \end{lstlisting}
\end{frame}




%%%%%%%%%%%%%%%%%%%%%%%%%%%%%%%%%%%%%%%%%%%%%%%%%%%%%%%%%%%%%%%%%%%%%%%%%%%%%%%%
\section{Actium}

\subsection{Introduction}

\begin{frame}
  \frametitle{Context: Action Systems \& testing}
  \begin{itemize}
    \item Back, 1983: description of  distributed sytems, alternative to CSP formalism
    \item \emph{Processes} can participate in one \emph{action} at a time, if it is enabled
      (its \emph{guard} is true)
    \item Now: Action Systems are useful \emph{test models} for \emph{model based testing}
    \item Specifically: Model based mutation-testing of nondeterministic systems
  \end{itemize}
\end{frame}

\begin{frame}
  \frametitle{Example system}
  \centering
  \scriptsize
  \hspace{1.8cm}
  \begin{tikzpicture}
    \usetikzlibrary{positioning}
    \tikzumlset{font=\scriptsize\sffamily}
      \umlbasicstate[x=0,y=0,name=Destroyed]{Destroyed}
      \umlbasicstate[x=4,y=0,name=AirAndOn]{AirAndOn}
      \umlbasicstate[x=4,y=-2.5,name=GroundAndOn]{GroundAndOn}
      \umlbasicstate[x=0,y=-2.5,name=GroundAndOff]{GroundAndOff}
      \umlstateinitial[x=0,y=-4,name=Start]
      \draw [->] (Start.north) -- (GroundAndOff.south);
      \draw [->, transform canvas={yshift=+2mm}] (GroundAndOff.east) -- (GroundAndOn.west)
          node[above, midway]{PowerOn};
      \draw [->, transform canvas={yshift=-2mm}] (GroundAndOn.west) -- (GroundAndOff.east)
          node[below, midway]{PowerOff};
      \draw [->, transform canvas={xshift=+2mm}] (GroundAndOn.north) -- (AirAndOn.south)
          node[right, midway]{Start};
      \draw [->, transform canvas={xshift=-2mm}] (AirAndOn.south) -- (GroundAndOn.north)
          node[left, midway]{Land};
      \draw [->] (AirAndOn.west) -- ([yshift=-1pt] Destroyed.east) node[above, midway]{Destroy};
      \draw [rounded corners] ([yshift=-2mm] AirAndOn.east) 
          -| +(0.6, 0.4) -- ([yshift=2mm] AirAndOn.east) [->];
      \node [right=0.6cm of AirAndOn.east] {Move(x, y)};
  \end{tikzpicture}
\end{frame}

\begin{frame}[containsverbatim]
  \lstset{gobble=4}
  \frametitle{Reference (existing) syntax}
  \begin{lstlisting}[language={}]
    destroy() if mode == Air then
    {
      mode := Destroyed;
      engine := 0;
    };

    move(x:MyNat, y:MyNat) if mode == Air && engine == 1 then
    {
      pos_x := pos_x + x;
      pos_y := pos_y + y;
    };
  \end{lstlisting}
\end{frame}

\subsection{Implementation}

\begin{frame}[containsverbatim]
  \lstset{gobble=4,deletestring=[b]'}
  \frametitle{Implemented syntax}
  \begin{lstlisting}
    when('Destroy) given mode === Air then_do (
      mode := Destroyed,
      engine := F
    )

    when('Move('dx, 'dy)) given (
      mode === Air && engine === T) then_do (
      pos_x := pos_x + 'dx,
      pos_y := pos_y + 'dy
    )
  \end{lstlisting}
\end{frame}

\begin{frame}[containsverbatim]
  \lstset{gobble=4} 
  \frametitle{Gherkin
    example\footnote{\scriptsize\protect\url{https://github.com/cucumber/cucumber/wiki/Gherkin}}}
  \begin{lstlisting}[language={}]
    7:   Given some precondition
    8:     And some other precondition
    9:    When some action by the actor
    10:    And some other action
    11:    And yet another action
    12:   Then some testable outcome is achieved
    13:    And something else we can check happens too
  \end{lstlisting}
\end{frame}

\begin{frame}[containsverbatim]
  \lstset{gobble=4}
  \frametitle{Overview of the ActionSystem trait}
  \begin{lstlisting}
    trait ActionSystem {
      // concrete:
      def run: Stream[Choice]
      def initialize(assignments: Assignment[State]*): Unit
      def addAction(action: Action): Unit

      // abstract:
      type State
      def chooseAction(actions: Seq[Action]): Option[Action]
      def chooseParameters(label: Label, 
                           params: Seq[Variable[State]]) 
        : Map[Variable[State], State]
    }
  \end{lstlisting}
\end{frame}

\begin{frame}
  \frametitle{Look of the plain implementation}
  \centering\Huge
  \textcolor{textgreen}{Showing code in IDE}
\end{frame}



\begin{frame}
  \frametitle{Properties of the implementation}
  \begin{itemize}
  \item The system contains a mutable environment to represent its state
  \item Actions are represented symbolically by expression ADTs, which are executed at evaluation
    (deep embedding)
  \item Running happens by lazily evaluating a \lstinline|Stream[Choice]|, which is defined
    recursively
  \item As much semantics as possible is left abstract and can be mixed in; actions only define
    structure and behaviour, not way of execution
  \end{itemize}
\end{frame}

\begin{frame}[containsverbatim]
  \lstset{gobble=6,deletestring=[b]'}
  \frametitle{Extra features}
  \begin{itemize}
  \item Multiple actions with same name automatically supported (internally kept separate)
  \item External statements (additionally to assignments):
    \begin{lstlisting}
      externally {
        println(s"E: x = ${'x.value}, y = ${'y.value}")
        if ('x.value < 0 || 'y.value < 0) {
          println("Aborting: x < 0 || y < 0.")
          abort
        }
      }
    \end{lstlisting}
  \end{itemize}
\end{frame}

\begin{frame}
  \frametitle{Possible improvements}
  \begin{itemize}
  \item Better Parametrization of choice methods
  \item Using macros to allow using blocks instead of parameter lists
  \item Better support for state types (currently mainly ints), and actually use Scala's type
    system (or Shapeless generics)
  \item Wildcard parameters \& pattern matching, nested actions
  \item Make evaluation immutable (currently: stateful updates)
  \end{itemize}
\end{frame}

\begin{frame}
  \centering\Huge
  \textcolor{textgreen}{Thank you!\\[1em] Questions?}
\end{frame}

\end{document}
