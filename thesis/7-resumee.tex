\chapter{Résumé: What Can Be Learned From This}
\label{sec:resumee}

This section shall reflect on the practical part, with respect to the questions stated in the
introduction: how do the described language features help with real programming problems, which of
them are most useful, and what is missing? What are advantages and disadvantages of the taken
approach? Where could it be improved, and what are possible future directions? How did the
development process took course, and what can be learned from the implementation and its design?

\newthought{In general, it can be said} that Scala fulfilled most of the expectations about its
syntax and semantics' abilities. Comparing the original syntax:
\begin{lstlisting}[language={}]
  powerOn() if engine == 0 && !(mode == Destroyed) then
  {
    engine := 1;
  };
\end{lstlisting}
with the finally achieved one:
\begin{lstlisting}
  when('PowerOn) given engine === 0 && mode =!= Destroyed then_do (
    engine := 1
  )
\end{lstlisting}
we see that a well matching analogy could be created, which would probably be understood right away
by someone used to the external syntax. That this matching is quite accurate can also be observed by
comparing the full original example in~\aref{sec:simplerocket_original} with its translation
in~\aref{sec:simplerocket}.

Furthermore, a few useful extra features have been implemented, such as a bit more flexibility in
the keyword usage (|given| and |when| can be swapped, |and|, |but| and or used for building the
guards, and multiple actions given the same label), or the addition of external statements to allow
useful mixing between the system behaviour and Scala code. The feasibility of integrating the
library with an existing toolchain has not been practically tested, but a translation of the
symbolic representation of \actium{} Action Sytstems to the format used by \texttt{as2bmc} has been
provided with little effort. Since a fully symbolic approch is used anyway, other translations
should also not really be a problem. 

\newthought{The process of developing the library} went through multiple steps. At first, a kind of
top-down approach was tried: this consisted of only defining an interface, which should resemble the
\enquote{look and feel} of the existing syntax, and contained just dummy implementations. Using this
syntax, a few difficulties and dead ends could already be ruled out, and most of the concrete
keywords and the \enquote{style of writing} were fixed (like the definition of all actions within
the primary constructor of a subclass of |ActionSystem|, or the idea of using Gherkin's keywords
|given| and |when|). Still, some syntactic subtleties of the \dsl{} and some practical needs could
not be identified at this stage.

Therefore, when it came to actually implementing the functionality of running Action Systems, the
previous code needed to be rewritten completely, leaving behind only its main ideas. The reason for
this was that it turned out to be much easier to try out different variants of expressing domain
specific constructs when a working system was already given~-- this is because then the constructs
can immediately be tested for fitting the underlying system, and for not interfering with each other
later. With that insight in mind, at first, an almost completely working non-\dsl{} implementation
of |ActionSystem| was written, which essentially stayed the same until the final version. On top of
this underlying implementation, the additional mixin |GherkinSyntax| was provided, building on the
few underlying constructs provided by |ActionSystem| in the way described
in~\autosubref{sec:improving_syntax}.

This turned-around approach proved much more practical. While for a smaller, more strictly defined
target \dsl{} syntax, it might actually be easier to start at the \enquote{top} of the library and
then implement out dummy interfaces, until an underlying implementation is completed, the more
bottom-up way of layering the convenience syntax above a predefined simple, conventional, and
working functionality actually turned out to be more fruitful. In retrospection, having such a
separation is also more desirable from architectural and maintainance points of view; for it allows
to separate concerns much better, leads to less coupling, and probably facilitates testing
(although, honestly speaking, since the whole development was of a rather experimental nature, these
concerns were not much considered and often neglected in praxis~-- but this does not relativize named
points). The a-posteriori introduction of other, not primarily considered \dsls{} (for example, an
external variant), or stacking multiple ones, is also enabled and simplified in this way.

\newthought{Concerning the advantages of Scala's syntax}, a certain dichotomy can be observed. On
the one hand, it almost everything that was wanted to be expressed could be implemented somehow. In
this respect, the ease of defining \abbrev{ADT}s with custom operators was of much help; but
especially implicits turned out to be the rescue to many of the more intricate problems (so, for
example, the solution to \enquote{lazy capturing} of state variables described
in~\autoref{lst:external_example}). Their universal applicability to interfere almost anywhere in an
expression, and in a way transform the types of values there, is a unique and highly beneficial
concept.

On the other hand, there were some problems that could not be solved (or rather, some desired syntax
that could not be expressed) by standard means. The primary example of this is the need to put the
statements of an action in round parentheses, separated by commata, instead of a block and thus
looking like a language statement. This is due to the fact that \actium{} needs to use a deep
embedding (to be able to operate on its underlying representation), which requires passing every
\enquote{syntactic form} of the \dsl{} as a Scala expression. Now, if a block were used to pass
statements into |then_do|, that would always evaluate to only the last expression in its body. By
defining |then_do| to take a variadic parameter list, this can be resolved, at the cost of a less
intuitive syntax. This problem could in principle be solved by using a macro for |then_do|, which
takes the block argument as an expression, splits it into the individual statment expressions, and
puts them in a list; however, as mentioned, practical macro implementations are outside the scope of
this work.

Another feature of Scala's syntax has a sometimes unintuitive behaviour as well: the dotless calls
used for |given| and |when|. Concretely, due to the way the parser is working and how the resulution
of dottless calls is defined, they cannot be interrupted by newlines; that is, writing
\begin{lstlisting}
  when('PowerOn) 
  given engine === 0 && mode =!= Destroyed then_do (
    engine := 1
  )
\end{lstlisting}
would not work, since the |given| in the second line is not recognized as a continued invocation of
the result of |when|, but as another statement. This is a basic limitation of Scala, and common
knowledge, but can be unexpected from the thinking perspective of a \dsl{} user. It is not a new
problem when using a deep embedding; for example, the internal \dsl{} variant of\texttt{sbt}'s
configuration syntax (used for automatizing Scala builts, somewhat like \texttt{make}) suffers from
the same limitation~\cite{sbt2015:reference}.

In a similar fashion, though not as much striking, is the in some ways complicated definition of
operator precedence used by Scala. In many cases, defined operators tend to \enquote{just work},
because the most common symbols are assigned their \enquote{usual} precedences, and sensible
exceptions are made for some special cases (such as recognizing |:=| as an assignment
operator). However, sometimes, a certain operator name is desired to be used in a certain position,
but will not do the right thing there without parentheses, because it associates in wrong ways. This
was the case when, at first, the inequality operator for values was defined as |!==|; the redefinion
as |=!=| does the right thing due to another exception rule, but noting this, and having to think
about it for every operator, can be frustrating for both developer and \dsl{} user. An explicit
possibility to manually define precedences, as in some other languages with operators, would be
desireable here (like Haskell's |infixl|/|infixr| declarations).

\newthought{When it comes to the expressiveness} of the Action Systems definable by \actium{}, there
are some limitations compared to the original implementation. This is due to the fact that this
work's main concern was to explore the limits of Scala for \dsl{} implementation; some additional
features present in other languages based on Action Systems were therefore left out.

For one, while the |ActionSystem| trait is parametrized by the type |State|, this is currently not
of too much use, and |State| is in all examples and tests defaulted or constrained to |Int|. While,
with some further work, the \dsl{} could be adapted to some custom state type (this requires syntax
and semantics for values, expressions, and predicates on it), this quite complicated (if not
impossible with standard features) to be done in general, in a satisfying way. The reason for this
is, again, that we are working with a deep embedding~-- which means that type information needs to
be present symbolically in the representation at runtime. On the other hand, the implementation is
currently layed out to support static type checking in Scala, which is in principle desirable (since
it allows to prevent certain errors at compile time).

The difficulty now lies in the fact that to get all desired features, one would need to represent
types both symbolically in the \abbrev{AST} and statically in Scala. This could theoretically be
done by using a more complicated encoding (using type tags, singletons, or some other technique
relying on more advanced typing techniques), but it is unclear whether such an encoding could be
done without breaking the current syntax, and without having to require the definer of a type to
write too much unnecessary boilerplate again. It would maybe be possible to achieve a satisfying
result using macros, again. (The automatic derivation of generic type representations in
Shapeless\footnote{\protect\url{https://github.com/milessabin/shapeless/wiki/Feature-overview:-shapeless-2.0.0\#generic-representation-of-sealed-families-of-case-classes}
  (visited on 2015-07-03)} looks primising for this purpose.)

Furthermore, there are some syntactic constructs which \actium{} currently lacks. For example, many
concrete implementations using Action Systems allow to define actions in a nested way (that is, to
define additional guards with different statements inside of one guard); adding support for these
would mainly consist of adding a syntactic transformation, and adapting the \dsl{} to behave
properly when the keywords are used not at top level.

Moreover, there has also been come up with the idea of adding a kind of pattern matching support for
action parameters. This would allow to make the writing of actions with similar guards easier and
more readable, especially combined with some sort of wildcard pattern. Implementing this would
require extending the parameter-passing mechanism a bit, and, most importantly, an algorithm for
unification and ordering patterns by specificity. Whether implementing a reasonable variant of the
latter can be done easily, without having to resort to, \eg, an external solver, is unknown to the
author.

Besides these enhancements of functionality, it would also be worthwhile to examine and enhance the
coherence of the implementation with some formal execution semantics. For instance, currently, the
order of execution of statements in an action is somewhat undefined (in practice, they are always
executed sequentially). But since Action Systems are a means of formalizing concurrent systems,
being able to exactly specify or know execution order would be a desirable property. It might also,
even in the current variant, sometimes be unclear how, for example, the side effects of two external
statements will be sequenced and influence each other.

There is also an issue arising from the frequent use of traits and mixins, which in principle turned
out to be a very practical pattern to separate concerns. While traits mostly allow to specify
several degrees of coupling very exactly, they are sometimes still too weak to describe what is
wanted to be achieved. This limitation was most obvious for the choice mixins, which define how
parameters and actions are chosen during execution of an Action System. There, it soon became
visible that while mixins are very practical for adding orthogonal static implementations, they are
not so easy to use for configuring behaviour. Concretely, at the moment, one can only define an
|ActionSystem| with certain choice behaviour by using the |with| syntax; but it often would be nice
to allow finer configuration, such as setting specific weights for actions in the |RandomChoice|
trait. Here, one could either investigate ways of elegantly allowing parametrization of mixin
behaviour, or think about another alternative not using mixins in this way at all (such as a
conventional design pattern). Reflecting this, we can conclude that traits are extremely useful for
orthogonal static, but less so for runtime parametrization.

Finally, the current implementation is a bit unsatisfactory from a \enquote{purist} functional
programming point of view. This is because all settings, definitions, but also the execution of
systems themselves are based on mutable updates. While this is not a major flaw, it would be
desirable to rewrite as much as possible of the execution to using immutable updates~-- this would
enhance some minor points, such as allowing easier pesistance of immediate states, or removing the
necessity of resetting systems between test runs. More immutability would also perceived as more
elegant by the author, and could lead to more correct code, as errors in mutable updates tend to be
harder to find. The best way to introduce immutability would probably be to run an |ActionSystem|
like a Mealy machine, or similar to a state monad (copying new state, instead of updating). On the
other hand, there is little that can be done to remove mutable updates from the action definition
syntax (the |given|\slash |when| statements in the primary constructor); but this seems more
acceptable, as it is done only once at object initalization (at least one could revisit the visibity
of |addAction|, and make it as private as possible).

%%% Local Variables: 
%%% TeX-master: "document"
%%% End: