\documentclass[11pt,a4paper,twoside]{memoir}
\usepackage[final,protrusion,babel]{microtype}
\usepackage[utf8]{inputenc}
\usepackage[LGR,T1]{fontenc}
\usepackage[polutonikogreek,english]{babel}
\usepackage[style=english]{csquotes}
\usepackage{xspace}
\usepackage[final]{graphicx}
\usepackage[textsize=tiny]{todonotes}
\usepackage[final,hyperfootnotes=false]{hyperref}

\usepackage{placeins}           % \FloatBarrier
%\usepackage{float}             % H float placement
\usepackage[originalcommands]{ragged2e} % improved ragged paragraphs in abstract

\usepackage{amsmath}
\usepackage{amssymb}
%\usepackage{pifont}
\usepackage{fourier-orns}

\usepackage{../packages/typographic_setup}
\usepackage[final]{../packages/mylistings}

%-------------------------------------------------------------------------------
% TITLING
%
\newcommand{\upinitial}[2]{{\textup{#1}\kern#2}}

\title{Designing Embedded \\ Domain-Specific Languages \\ in Scala} 
\subtitle{\upinitial{A}{0pt} \upinitial{C}{-0.8pt}ase %
  \upinitial{S}{-0.8pt}tudy with \upinitial{A}{-0.7pt}ction \upinitial{S}{-1.4pt}ystems}

\author{Philipp Gabler}

\date{}

\hypersetup{
  pdfinfo={
    Author={Philipp Gabler},
    Title={Designing Embedded Domain-Specific Languages in Scala}
  },
  colorlinks=false
}

%-------------------------------------------------------------------------------
% BIBLATEX
%

\usepackage[%
  backend=biber,
  citestyle=alphabetic,
  style=alphabetic,
  sortcites=true,
  sorting=nyt,
  %natbib=true,
  firstinits=true,
  %url=false,
  isbn=false,
  date=iso8601,
  urldate=iso8601
]{biblatex}
\DeclareNameAlias{default}{last-first}

% only capitalize real titles
% http://tex.stackexchange.com/a/22981/46356
\DeclareFieldFormat{sentencecase}{\MakeSentenceCase{#1}}
\renewbibmacro*{title}{%
  \ifthenelse{\iffieldundef{title}\AND\iffieldundef{subtitle}}
    {}
    {\printtext[title]{%
        \printfield[sentencecase]{title}%
        \setunit{\subtitlepunct}%
        \printfield[sentencecase]{subtitle}}
      \newunit}%
  \printfield{titleaddon}}

\defbibheading{memoir}[\bibname]{\chapter*{#1}}
\setcounter{biburllcpenalty}{7000}
\setcounter{biburlucpenalty}{8000}

\AtEveryBibitem{\clearlist{language}} % clears language
\renewcommand*\bibnamedash{\rule[0.48ex]{3em}{0.14ex}\space}
\renewcommand*{\multinamedelim}{;\space}
\renewcommand*{\finalnamedelim}{;\space}

\addbibresource{document.bib}

\usepackage{breakurl}
%\hypersetup{breaklinks=false}

%-------------------------------------------------------------------------------
% OTHER SETTINGS
%

% listings
\lstset{
  gobble=1,
  language=Scala,
  style=blackwhite
}

\AtBeginDocument{%
  \lstMakeShortInline[style={inline}]|
}

% toc
\setlength{\cftbeforechapterskip}{1ex} % decrease space between sections

% floats and epigraphs
\newlength{\capwidth}
\addtolength{\capwidth}{\textwidth}
\addtolength{\capwidth}{-4ex}
\captionwidth{\capwidth}
\captionstyle[\raggedright]{}

\setlength{\epigraphwidth}{\textwidth}
\setlength{\epigraphrule}{0pt}

\renewcommand{\textfloatsep}{\baselineskip}

\setcounter{topnumber}{1}       % allow only one float per page

% no need for colors here...
\colorlet{textred}{darkgray!80}
\colorlet{textblue}{darkgray!80}

% change default autoref names
\addto\extrasenglish{%
  \renewcommand{\chapterautorefname}{Section}
  \renewcommand{\sectionautorefname}{Subsection}
  \renewcommand{\pageautorefname}{Page}
  %\renewcommand{\pageautorefname}{Page}
  %\renewcommand{\subsectionautorefname}{Subsection}
}
\newcommand{\lstlistingautorefname}{Listing}

% csquotes
% redefine spacing above/below; hacking original latex definition from 
% http://mirrors.ctan.org/macros/latex/base/classes.dtx
\makeatletter
\newenvironment{myblockquote}
               {\vspace{-1em}\list{}{\listparindent 1.5em%
                        \itemindent    \listparindent
                        \rightmargin   \leftmargin
                        \parsep        \z@ \@plus\p@}%
                \item\relax}
               {\endlist\vspace{-0.5\baselineskip}}
\makeatother
\SetBlockEnvironment{myblockquote}

% continuous footnotes
% http://compgroups.net/comp.text.tex/preventing-footnote-counter-to-reset-with-every/1925434
\makeatletter
\@removefromreset{footnote}{chapter}
\makeatother

% GREEK FONT
% \usepackage{gfscomplutum}
\usepackage{../packages/neohellenic_fixed} % otherwise, this overwrites numbers in math mode
                                           % and some symbols (like \textbullet)
\providecommand{\greek}[1]{%
  \textneohellenic{\fontsize{0.7em}{\baselineskip}\selectfont\selectlanguage{polutonikogreek}#1}}
% \providecommand{\greek}[1]{%
%   \selectlanguage{polutonikogreek}#1}

% TIKZ SETUP
\usepackage{tikz}
\usepackage{tikz-uml}
\pgfkeys{%
  /tikzuml/fill class=darkgray!20, 
  /tikzuml/fill state=darkgray!20, 
  /tikzuml/fill package=white
}

\usepackage{rotating}

%-------------------------------------------------------------------------------
% DOCUMENT MACROS
%
\newcommand{\ie}{i.e.\xspace}
\newcommand{\eg}{e.g.\xspace}
\newcommand{\cf}{cf.\xspace}
\newcommand{\margintodo}[1]{\todo[noline, size=\tiny]{#1}}
\newcommand{\dsl}{\abbrev{DSL}}
\newcommand{\dsls}{\abbrev{DSL}s}
\newcommand{\actium}{\texttt{actium}}
\newcommand{\actiumtitle}{{\small\texttt{ACTIUM}}}
\newcommand{\csharp}{{C\nolinebreak[4]\raisebox{0.03em}{\#}}}
\newcommand{\fsharp}{{F\nolinebreak[4]\raisebox{0.03em}{\#}}}
\newcommand{\CC}{{C\nolinebreak[4]\hspace{-.05em}\raisebox{.22ex}{\small\textbf{++}}}}
%\newcommand{\autosubref}[1]{{\hyperref[#1]{Subsection~\ref*{#1}~--~\nameref*{#1}}}}
\newcommand{\autosubref}[1]{\autoref{#1}}
\newcommand{\aref}[1]{\hyperref[#1]{Appendix~\ref{#1}}}
\newcommand{\githubsymbol}{%
  \includegraphics[height=0.75\baselineskip]{fig/github-logo-large}}
\newcommand{\github}[2][https://github.com/phipsgabler]{\href{#1/#2}{\protect\githubsymbol}}
\newcommand{\githubcommit}[2]{\href{#1#2}{\nolinkurl{#1}}}

\newcommand{\mathlst}[1]{\text{\lstinline|1|}}


%-------------------------------------------------------------------------------
% DOCUMENT
%-------------------------------------------------------------------------------

\begin{document}
\pagestyle{simple}
\chapterstyle{hangnum}
\frontmatter

\begingroup
  \thispagestyle{empty}
  \centering
  \vspace{3cm}
  {\LARGE Philipp Gabler \par}
  \vspace{2cm}
  {\Huge\bfseries \thetitle \par}
  \vspace{0.5cm}
  {\Large\itshape\bfseries \upinitial{A}{0pt} \upinitial{C}{-0.8pt}ase %
    \upinitial{S}{-0.7pt}tudy with \upinitial{A}{-0.6pt}ction \upinitial{S}{-1.4pt}ystems \par}
  \vspace{2.2cm}
  {\Large\scshape Bachelor's Thesis}
  \vfill
  {\large Graz University of Technology \par}
  \vspace{1ex}
  {\large Institute for Software Technology \par}
  \vspace{1cm}
  {Supervisor: Ao.\thinspace{}Univ.-Prof. Dipl.-Ing. Dr.\thinspace{}techn. Bernhard Aichernig \par}
  \vspace{1.1cm}
  {\small\slshape Graz, June 2015 \par}
\endgroup


\movetoevenpage
\begin{adjustwidth}{\absleftindent}{\absrightindent}
  \phantomsection
  % \addcontentsline{toc}{chapter}{Licenses}
  \label{license}
  \currentpdfbookmark{License}{license}

  \abstracttextfont
  \vspace*{\stretch{1}}
  \begin{center}
    This work is licensed under a \\ \href{http://creativecommons.org/licenses/by-sa/4.0/}{Creative
      Commons Attribution-ShareAlike 4.0 International License}.
  \end{center}
  \begin{center}
    \includegraphics[scale=1]{fig/by-sa}
  \end{center}
  \vspace{\stretch{2}}
  \begin{center}
    All code samples, unless otherwise noted or cited from other sources, \\ are also available under an
    \href{http://opensource.org/licenses/MIT}{\abbrev{MIT} license}:
  \end{center}
  \vspace*{-1ex}
  \begin{ttfamily}
    \setlength{\parskip}{12pt}
    \setlength{\parindent}{0pt}
    The MIT License (MIT)

    Copyright (c) 2015 Philipp Gabler

    Permission is hereby granted, free of charge, to any person obtaining a copy of this software
    and associated documentation files (the "Software"), to deal in the Software without
    restriction, including without limitation the rights to use, copy, modify, merge, publish,
    distribute, sublicense, and/or sell copies of the Software, and to permit persons to whom the
    Software is furnished to do so, subject to the following conditions:

    The above copyright notice and this permission notice shall be included in all copies or
    substantial portions of the Software.

    THE SOFTWARE IS PROVIDED "AS IS", WITHOUT WARRANTY OF ANY KIND, EXPRESS OR IMPLIED, INCLUDING
    BUT NOT LIMITED TO THE WARRANTIES OF MERCHANTABILITY, FITNESS FOR A PARTICULAR PURPOSE AND
    NON\-IN\-FRINGE\-MENT. IN NO EVENT SHALL THE AUTHORS OR COPYRIGHT HOLDERS BE LIABLE FOR ANY
    CLAIM, DAMAGES OR OTHER LIABILITY, WHETHER IN AN ACTION OF CONTRACT, TORT OR OTHERWISE, ARISING
    FROM, OUT OF OR IN CONNECTION WITH THE SOFTWARE OR THE USE OR OTHER DEALINGS IN THE SOFTWARE.
  \end{ttfamily}
  
  \vspace{2ex}
  
  \begin{adjustwidth}{0.5\absleftindent}{0.5\absrightindent}
    \begin{Center}
      A full |sbt| project containing many of the code samples can be found at
      \url{https://github.com/phipsgabler/dsl-examples}. 

      The \LaTeX{} source and a screen-optimized version of this document \\
      are available at request.\footnote{\texttt{pgabler@student.tugraz.at}}
    \end{Center}
  \end{adjustwidth}
  
  \vspace{\stretch{1}}

\end{adjustwidth}


\movetooddpage
\phantomsection
% \addcontentsline{toc}{chapter}{Abstract}
\label{abstract}
\currentpdfbookmark{Abstract}{abstract}
\begin{abstract}

\noindent
Programming languages have been and still are becoming more and more abstract, and increasingly
specialized complicated applications and libraries are implemented. With that tendency, a great
amount of \enquote{linguistic} flexibility is both available and needed. However, the combination of
the power at hand of the programmer and the need to describe the increasingly complex systems has
yet to be fully explored; this is the traditional habitat of Domain Specific Languages (\dsls).

\textsc{Dsl}s have been in long use in the programming community. The philosopy behind them is the
following: do not try to come up with a syntax so general that it can concisely express every
problem (that is probably impossible); rather, define a smaller language to describe the specific
problem, and embed that into a system which can combine the specific languages. In the case of
\emph{embedded} \dsls, that outer system is itself a general-purpose programming language, though
preferably one which properly supports embedding small sublanguages in a convenient way.

Often, \abbrev{LISP} has been termed such a \enquote{programmable programming language}. As of
today, Scala probably comes closest to this high claim, at least from the perspective of a
programmer used to conventional languages. This bachelor thesis tries to explore how Scala makes it
easy to write \dsls{}, which allow to express specific problems such that there is no additional
\enquote{code noise} around them, and that the description really behaves in a transparent way, like
a reader would expect it from the code.

For that purpose, in the first part an overview is given of \dsls{} in general and Scala's special
features that help with implementing them, complemented with a description of some useful patterns
for that purpose. Then, in the second part, the implementation of a specific \dsl{} for Action
Systems is described. Action Systems are a formalism originally used to formalize behaviour of
distributed, concurrent systems; in the context of this work, however, they are treated more as in
their interpretation as non-deterministic transition systems, which is a useful point of view for
performing model-based testing.
\end{abstract}

\begingroup
\cleartorecto
\hypersetup{linkcolor=black, hyperindex=true}
\label{toc}
\currentpdfbookmark{Contents}{toc}
\tableofcontents*
\endgroup

\cleartorecto
\mainmatter

\phantomsection
\addcontentsline{toc}{chapter}{Introduction}
\chapter*{Introduction}

Exploring Domain Specific Languages (\dsls{}) is a very broad topic. They have been around for a
long time, and inspected in many ways, for many different purposes, in many implementations. This
work therefore tries to focus on some core aspects in a limited setting. On the one hand, Scala's
capabilities for writing embedded \dsls{} shall be investigated and summarized. This includes the
inquiry of the relevant syntactic and semantic features of the language, of how and to what extent
they help in the implementation of \dsls{}, and the examination of existing patterns for that
purpose (of theoretical nature as well as occuring in practical software). This part of the
exploration is mostly theoretical or based on small examples~-- it contains explanations about the
language, about its background, and about how it can be used.

On the other hand, a somewhat larger, concrete example of an embedded \dsl{} shall be constructed in
practice. This example is a library for the definition of Action Systems, as they are used in
model-based mutation testing~\cite{aichernig2014:killing,aichernig2009:mutation}. With this
practical project, the actual development of a \dsl{} library in Scala will be analyzed, with the
aim of gaining insight into what a development process with such goals can look like (and what
difficulties it might involve), into where the language helps with the embedding of a \dsl{} and
where it hampers it, and finally into how useful the previously examined language featureas are in
practice~-- expecially, if some desireable properties are missing, or if some parts could in fact be
done better with different means.

\newthought{The work is structured as follows}: There are two logical parts. In the first three
sections, concepts and techniques for \dsls{} in Scala are explained.  At the beginning,
\autoref{sec:concepts} gives an overall view of domain-specific languages and Scala as a programming
language. It shortly introduces Scala's background and reviews \dsls{} in general, explaining their
foundation and position in the history of programming languages, discusses advantages and
disadvantages, and defines some commonly used terms. Then, \autoref{sec:syntax} gives an intoduction
into parts of Scala's syntax which are the most relevant and useful to writing embedded \dsls{},
includung examples of their usage. Thirdly, \autoref{sec:patterns} shows how the previously
introduced syntactic constructs can be combined into some useful patterns, which help program
organization and development when working with \dsls{}.

Building on that, the remaining three sections consist of the description of the practical part of
the thesis: an implementation of an embedded \dsl{} for describing Action Systems, with the working
title \actium. Initially, \autoref{sec:action_systems} gives the theoretical introduction into
Action Systems, and explains their usage and special variant in model-based (mutation) testing. It
follows \autoref{sec:implementation}, in which the implementation of \actium{} is
summarized. Special focus is layed on the design parts where the described above language features
were used; notable decisions or interesting constructs used are pointed out. This section also gives
an overview of the functionality and usage of the implementation. Finally, \autoref{sec:resumee}
reflects on the practical part, mentions possible improvements or further directions, and summarizes
the advantages and disadvantages for this type of implementation, as opposed to other possibilities.


%%% Local Variables: 
%%% TeX-master: "document"
%%% End:

\chapter{Concepts: Scala and DSLs}
\label{sec:concepts}

This part will first give a short overview of different features of Scala, followed by a basic
introduction to \dsls, tracing them into the history of programming languages and concluding with
the paradigm of language-oriented programming.


%%%%%%%%%%%%%%%%%%%%%%%%%%%%%%%%%%%%%%%%%%%%%%%%%%%%%%%%%%%%%%%%%%%%%%%%%%%%%%%%%%%%%%%%%%%%%%%%%%%%%
\section{Scala: Background and Paradigms}
\label{sec:scala}

Scala is a very flexible, multi-paradigm language, including a host of ideas by its
inventor Martin Odersky. For one, it provides advanced object-oriented constructs: almost everything
is in a class; there is a sophisticated type hierarchy; traits serve as \enquote{better interfaces},
and provide very flexible ways of inheritance, like mixins; there are a lot of constructs for
different aspects of class modelling, and they all can be nested. There is also a really smoothly
working and consistent module concept (which is probably influenced by Odersky's experience with
Modula-2). As a whole, Scala has been designed to serve as a multi-paradigm language, allowing
writing programs efficiently and expressively in the small as well as in the large scale~-- hence
also its name, stemming from \emph{scalable language}.

Furthermore, Scala is also quite a powerful statically typed functional language, not only providing
functions as first-order constructs with convenient syntax (as is minimally expected from a
functional language), but also being equipped with an elaborate type system, allowing to express
more things than in most other languages (like higher kinds, and certain types of
polymorphism). Also, the border to object-orientation is layed out in a very uncomplicated and
well-functioning way: case classes serve as class hierarchies as well as providing algebraic data
types (\abbrev{ADT}s), which can be pattern matched on, and the subtyping and parametric
polymorphism systems can be fused nicely by aid of type bounds and variance annotations.

Still, it is not only this wide semantic span of paradigms that makes Scala suited for a wide range
of use cases and architectorial considerations. There has been spent considerable amount of thought
on a lot of larger and smaller syntactic enhancements and shortcuts, most of which allow programmers
write code in a much more readable and natural way than usually. These features range from
\emph{string interpolation} and \emph{\abbrev{XML} literals} over \emph{blocks} and \emph{infix
  method calls} to quite unique inventions like \emph{implicit definitions}. The subsequent sections
will introduce the most useful of them for the development of \dsls, accompanied with examples of
their usage and patterns of combining them well.

\newthought{A few words on the level of details}: surely, one cannot explain a whole language
here. This text is also not an introduction to Scala. In general, it will always be tried to
explain as much as possible, using plenty of examples, so that a programmer having experience in a
few languages will be able to figure out what code means. However, some basic knowledge about Scala
syntax is assumed, as well as background knowledge about Java, functional programming, and some
aspects of semantics and type theory that are commonly discussed along with pure functional
programming. In particular, this concerns concepts related to types and functions, since functional
programming is considered a preliminary to all described aspects.

A more detailed foundation of Scala's syntactic features and standard libraries can be found in
Odersky's introductory book \citetitle{odersky2008:programming}~\cite{odersky2008:programming}, or
the Scala language specification~\cite{odersky2014:scala_spec}. Java, its \abbrev{API}, and the
\abbrev{JVM} are as well described in detail by their standardization
documents~\cite{gosling2013:java_spec, lindholm2013:jvm_spec, oracle:java_api_spec} (for
version~7). General introductions into common functional programming techniques are the
classic~\cite{abelson1996:structure} on the practical side (including untyped lambda calculus, pure
and impure functional programming and a general introduction into \dsl{} principles),
and~\cite{thompson1991:type}, going into theoretical foundations and type systems (especially higher
typed lambda calculi). A more thorough account of the practical type systematic approaches used in
Scala can be found in~\cite{pierce2002:types}, by which the language has noticably been influenced.

\newthought{The larger examples in this part} (listings on the top, marked with rules) are, in most
cases, published in a Github
repository\footnote{\protect\url{https://github.com/phipsgabler/dsl-examples}}, for the sake of
better reproducibility and further study. This repository containins a compileable \abbrev{SBT}
project with their implementations, and sometimes improvements, usage examples, and other
showcases. Such examples are marked and hyperlinked with an Octocat
icon~(\raisebox{-0.16em}{\github{dsl-examples}}).


%%%%%%%%%%%%%%%%%%%%%%%%%%%%%%%%%%%%%%%%%%%%%%%%%%%%%%%%%%%%%%%%%%%%%%%%%%%%%%%%%%%%%%%%%%%%%%%%%%%%%
\section{Overview: Why DSLs?}
\label{sec:overview}

Reviewing the history of computer science, as far as it is concerned with programming languages, one
can distinguish two great paradigms of how computation can be expressed (these are
\enquote{paradigms} more in an epistemiological than in a technical sense): \emph{imperative} and
\emph{declarative}. This distinction can be rooted in the earliest abstractions of computation that
were formulated, namely, Turing machines and the lambda calculus~\cite{valverde2015:punctuated}. As
has been early proved, both formulations are in fact equivalent; there is however a split, when
practically applying these concepts, which are purely mathematical at their core.

This split has its origins in the engineering perspective of programming, which strongly influenced
the evolution of programming languages. At the beginning of the usage of physical computers, all
there was was machine language, which was hand-written by specialized personnel (maybe using the
first assemblers, but not much more). Higher-level languages in these circles were believed to be a
purely academic exercise and not efficient enough in practice (not an unreasonable claim at the
time); it was not until the mid-fifties that they began being used. Then, between 1954 and 1957, the
ancestors of all modern imperative languages, Fortran, \abbrev{ALGOL}, and \abbrev{FLOW-MATIC} (soon
superseded by its more popular offspring \abbrev{COBOL}) were invented and became
popular~\cite{sammet1972:programming}.

What was convincing about these three languages was on the one hand the Fortran compiler's ability
to produce executables equivalently efficient to average hand-written machine language, but with
less effort of writing~-- this led to the general acceptance of compilation being feasible. On the
other hand, the idea behind \abbrev{FLOW-MATIC/COBOL} (and, to a lesser extend, \abbrev{ALGOL}) was
to allow programs to be written in English language, in a descriptive fashion completely unlike the
previously prevalent style of native instructions. Soon after this short phase, the number of
programming languages invented started to increase almost exponentially. Languages were designed for
different purposes and with different backgrounds: \abbrev{SQL} for relational algebra queries,
\abbrev{APL} for vectorized numerical computations, C for systems programming, and so on. What is
common to all languages in this tradition is their intertwined struggle between generality and
expressivity~-- some paths evolved to what is nowadays called \enquote{general-purpose} languages
(such as \CC), while others became more and more specialized and even lost their abilities for
programming in the whole (such as \abbrev{SQL}).

This language family has machine language as its proto-language, and its members are all based on
some form of compilation of a syntax to an underlying machine model, or, in the case of interpreted
languages, executing them step-by-step on such a model. They are thus, in a sense, Turing-oriented:
syntax is considered an extra layer above the actual execution on an abstract machine; and the
engineering perspective mentioned above came up because concrete computers are just instances of
such abstract machines.

There evolved however another language family, fundamentally different from this concept:
\abbrev{LISP} and its successors. In contrast to the languages mentioned above, \abbrev{LISP} had
never been designed as a successor to machine language. Instead, the initial idea was just a formal
lambda calculus with primitive types, macros, and better syntax (\enquote{recursive functions of
  symbolic expressions}); the actual implementation on a computer happened only \enquote{by
  accident}~\cite{mccarthy1960:recursive}. To this comes the fact that initially, the main user
group of the language was the \abbrev{AI} community, finding in it a way to concisely describe their
abstract concepts, regardless of underlying implementation.

While \abbrev{LISP} over the course of its growth incorporated also imperative parts, it is
fundamentally geared to the lambda calculus, with its term-rewriting style of thinking, reinforced
by its own homoiconicity. In this tradition, the language lead to the invention of techniques that
probably would not have been devised from the Turing perspective: these are, for example, the
first-class membership of functions and continuations, dynamic typing at runtime, and garbage
collection. Such concepts came into being because the language was used by people who were not using
it to write machine instructions on a higher level, but who approached programming from an almost
mathematical perspecitive, wanting to specify \emph{what} something shall be, not \emph{how} it
should be executed. In that way, \abbrev{LISP} pioneered the idea of considering programming
languages as an end in themselves.

\newthought{These are the historical foundations} of programming languages, summarized in
short. From them, we can learn how the means of expressing intent to a machine evolved, and from
what perspectives they started.

Nowadays, (most) people need not anymore be convinced that high-level languages are something
positive and worth while. The challenge is to overcome the view of code as something to be solely
compiled down, an intermediate, beneath form of thought, between the brain and the machine. If we
instead accept code as a form of language, and take its means of expression not as syntactic
niceties, but as fundamental properties of it as a system, new perspectives will come up. In that
spirit, Abelson and Sussman say in \citetitle{abelson1996:structure} that \blockcquote[][Preface to
the First Edition]{abelson1996:structure}[.]{(\ldots\kern-1pt) we want to establish the idea that a
  computer language is not just a way of getting a computer to perform operations but rather that it
  is a novel formal medium for expressing ideas about methodology. Thus, programs must be written for
  people to read, and only incidentally for machines to execute}
\setlength{\parskip}{0cm} % a hell of a hack because I don't know why \list{} is resetting
                          % \parskip in twopage output???

As we have seen above, programming languages tend to oscillate between extreme poles: imperative
versus declarative, and general versus specific. And as the complexity of systems and architectures
grows, and their construction becomes a more professional business, all these poles have proven
their value in some or the other way; but most importantly, it can be said that none of them is the
key to salvation on its own. In fact, successful complex systems most often rely on a patchwork of
multiple, specialized components, glued together by general frameworks of architecture. And if we
bend our concept names a bit, general-purpose languages could actually be considered special-purpose
for general \enquote{glue} code, and imperative code considered declarative descriptions of
imperative processes. 

If we so refrain from looking for \enquote{one language to rule them all}, and accept the above view
about the importance of languages in themselves, we arrive at a new point of view~-- in some form, a
new paradigm: \emph{language-oriented programming}~\cite{dmitriev2004:language}. Its philosopy is
exactly the just described manner of organizing systems: splitting them in smaller, coherent parts,
and program them out using the respective \emph{domain specific language}, which is a language as
close as possible to the terms and concepts of the specific part.

\newthought{This language-oriented paradigm} and its main constituents, \dsls{}, have actually been
around for some, often under different names, or in the context of other methodologies. For example,
the programming approach suggested in \cite{abelson1996:structure} is there called \emph{metalinguistic
  abstraction} and proposed as a rather general approach to problem solving; but then, there have
always been used small configuration or mini-languages for systems such \abbrev{UNIX} (like the
\texttt{sed} program), which are not backed by such a general idea, but just evolved as
practical. Moreover, there is range of highly specific languages to help with specialized tasks,
including the already mentioned \abbrev{SQL}, the grammar specification language of \abbrev{ANTLR}
or other parser generators, and \LaTeX. We can also regard special syntaxes for specific purposes as
a kind of \dsl{}, like Matlab's matrix literals and special operators, or the simulation statements
of hardware description languages such as Verilog.

A summary of the advantages of using \dsls{} is given in \cite{spinellis2001:notable}, which also
lists notable patterns found in the development process of them. \cite{mernik2005:dsls} offers an
overview of decision, analysis, design, and implementation phases of \dsl{} development, from a more
methodological view point, altough the work focuses more on external languages. Case studies for
\dsl{} development are~\cite{wile2004:lessons} with a lot of more abstract \enquote{lessons
  learned}, and~\cite{havelund2010:case}, which compares two implementations of the same \dsl{} in
Python and Scala with respect to their \enquote{linguistic elegance} (and even contrasts multiple
variations in Scala).

\label{dsl-definitions}
\newthought{With all this variety} of declarative and in some sense domain-specific languages, we
could try to classify them (see also \autoref{fig:dslterms}). The most prominent and used
distinction for \dsls{} is that between \emph{external} and \emph{internal} \dsls{}. Thereby, a
\dsl{} is called internal if it is used as part of another, more general programming language
(either through special syntax, or provided by a library), and called external, if it is executed by
a dedicated \enquote{evaluator} different from the main implementation language (which can be
anything from a compiler over an interpreter to a \abbrev{DBMS}).\footnote{Again, \abbrev{SICP}
  tries to be smarter than and blur this binary distinction:
  \textcquote[][Chapter~4]{abelson1996:structure}{The evaluator, which determines the meaning of
    expressions in a programming language, is just another program.}}

\begin{figure}
  \centering\sffamily
  \begin{tikzpicture}[%
    ->, thick, grow=right,
    level 1/.style={level distance=8.5em},
    level 2/.style={level distance=11em, sibling distance=3.5em},
    child anchor=west]
    \node {DSLs}
    child {node [] {Internal/embedded DSLs}
      child {node {Shallow Embeddings}}
      child {node {Deep Embeddings}}}
    child {node [yshift=1em] {External DSLs}
      child {node [text width=8em] {Parsed, compiled, interpreted, etc.}}};
  \end{tikzpicture}
  \caption{Taxonomy of terms specifying the different variants of Domain Specific
    Languages.\label{fig:dslterms}}
\end{figure}

While external \dsls{} could arguably be differentiated in arbitrarily complex ways, some basic
further discrimination is practical for internal ones. For one, there are the two said possibilities
of having an internal \dsl{} provided by the language or via a library. The latter variant in the
following will be referred to as \emph{embedded}\footnote{As far as is known to the author, there is no
generally accepted convention for a clear distinction between the terms \enquote{embedded} and
\enquote{internal}, but this usage seems useful.}. It is embedded \dsls {} that this work focuses on,
since they are in a sense the most \enquote{language oriented}~-- using the features of a language
to extend itself.

Within the range of embedded \dsls{}, there is a further differentiation to be made: that of
\emph{deep} and \emph{shallow} embeddings. This distinction concerns the evaluation of the embedded
syntax: in a shallow embedding, the provided language constructs are merely combinators for some
already-existing objects in language, for example, they may simply be stacking functions in an
intricated way~-- but evaluation is not different from the normal language behaviour. In contrast, a
deep embedding uses an internal \abbrev{AST}, or at least some symbolic representation, together
with a custom evaluation function. The latter has the advantage that there is often more semantic
flexibility involved, and that there might be multiple evaluation functions provided
(\enquote{backends}); however, this usually comes with greater complexity and runtime or space
costs. A survey and comparison of both approaches can be found in~\cite{gibbons2014:folding}, which
uses a Haskell example.


%%% Local Variables: 
%%% TeX-master: "document"
%%% End:

\chapter{What Syntax Can Do for Us}
\label{sec:syntax}

This section should show what makes Scala a particularly good choice to write \dsls\ in. It focuses
on syntactic features, using examples to explain their usage and motivation, and also introduces
some additional small tricks related to them, as well as useful idioms that fit nowhere else. The
features described here are selected because of their suitablity for building \emph{control} or
\emph{data abstractions}: they allow to build interfaces with structures to compose data and
behaviour in an expressive way, without burdening with too many boilerplate declarations.

%--------------------------------------------------------------------------------
\section{Variants of Method Calling}
\label{sec:method_calling}

Scala, being based on a platform initially constructed for object-oriented programming, and being
designed with full utilization of this paradigm in mind, naturally has a notion of
\emph{methods}. It also makes plenty of use of them, and in fact, all behavioural functionality of
the language can be reduced to method calls. However, unlike other object-oriented languages of
similar kind, notably Java, Scala allows for a much higher syntactic flexibility of these, while
under the hood still working with the same mechanisms. These syntactic flexibilities mainly appear
in the form of \emph{operators} and \emph{infix notation}.

In the this work, the term \enquote{operator} shall always mean a function whose name consists of
non-alphabetic symbols and that is written infix (if binary) or pre- or postfix (if unary)~-- the
same terminology as used in the language specification. With \enquote{infix notation}, the style of
calling a \enquote{normal} method without dots and parentheses, as in |v foo x| instead of
|v.foo(x)|, will be meant. To distinguish \enquote{regular} method calls from them, often the term
\enquote{dotted call} will be used.  It is worth noting that most things being said here for methods
hold as well for type constructors (\ie, class names like |::|), but these are not further
investigated.

\newthought{Overwriteable operators} are an idea that has been around for some time in other
languages, although mostly in quite restricted form. Often, there are ways to redefine arithmetic
and boolean operators, and comparisons. The former is what one usually needs when implementing
numeric data types, such as matrices, that are not part of the standard libraries; the latter serves
mainly for providing custom equality and orderings with syntactic integration, which is quite often
desirable. There is of course also some \enquote{bending of the rules}, like the often-appearing
string concatenation with |+|, or \CC{}'s stream piping with |<<| and |>>|. From all these use cases
we see that programmers have a tendency to use operators where they seem the most natural (like in
mathematical contexts), or where they seem to be able to express themselves better than with dotted
method calling style, because of shortness or since their infix style may resemble the logical or
data flow better.

Some languages with a different background, mostly Haskell and Agda, take an opposite and quite
radical approach: operators there are just names for functions, which by default can be written in
infix notation (similarly \abbrev{LISP}, though it exclusively uses Polish Notation). This
functionality has lead to a style of combinator libraries in them (like Haskell's |parsec| parser
combinators\footnote{\protect\url{http://hackage.haskell.org/package/parsec} (visited on
  2015-06-04).}), which can be quite expressive and concise at the same time (using these in Scala is
also discussed in {\autosubref{sec:combinators}}). Scala, being partly influenced by these languages,
provides an intermediate way, but almost as expressive: namely, method names are allowed to contain
almost all unicode characters; and methods containing only non-alphabetical characters can be
written inline. These infix calls are recognized and converted into standardized textual and dotted
forms:
\begin{lstlisting}[escapechar=!]
  xs ++ ys !$\;\equiv\;$! xs.++(ys) !$\;\equiv\;$! xs.$plus$plus(ys)
\end{lstlisting}
As can be seen, the intermediate form of using the symbolic name of an operator in a dotted call is
also allowed.

\begin{lstlisting}[style=floating,
  caption={A small \dsl{} to represent symbolic propositional formulae. The implicit conversion in
    Line~12 allows to leave out the \lstinline|Var| constructor; such conversions are explainened in
    \autosubref{sec:implicits}.
    \hfill\github{dsl-examples/blob/master/src/main/scala/dsl_examples/Logic.scala}},
  label=lst:logic]
  sealed trait Expr {
    def ||(other: Expr) = Or(this, other)
    def &&(other: Expr) = And(this, other)
    def unary_! = Not(this)
  }

  case class Var(s: String) extends Expr
  case class Or(a: Expr, b: Expr) extends Expr
  case class And(a: Expr, b: Expr) extends Expr
  case class Not(a: Expr) extends Expr
  
  implicit def stringToExpr(s: String): Var = Var(s)
\end{lstlisting}

This translation is quite mechanical, and it does really not add any new functionality. It can,
however, massively improve the readability of certain types of programs. For example, the logic
\dsl{} from \autoref{lst:logic} allows to write a logic formula like this:
\begin{lstlisting}
  "a" || "b" && !"c"
\end{lstlisting}
instead of the following, which would be necessary in Java:
\begin{lstlisting}[language=Java]
  new Or(new Var("a"), new And(new Var("b"), new Not(new Var("c"))))
\end{lstlisting}

This example raises the question of how operator precedence behave. Can we be sure that when writing
\lstinline[style=inline]{!"a" && "b" || "c"}, we do not, against our expectations, end up with %
|Not(And("a", Or("b", "c")))|? It turns out that we can, at least in this case. But determining
precedence of operators (or conversely, assigning the right precedence to an operator when defining
it) can be a bit awkward in Scala. That is because, where other languages allow one to explicitly
specify precedence and associativity of operators, in Scala these are determined solely by the name
(\ie, the symbols) of the operator in question. This hard-wired convention is defined in the
language specification \cite[][Chapter~6.12]{odersky2014:scala_spec}. The logic behind is basically
the following:
\begin{enumerate}\label{operators}
\item Precedence is determined by the first symbol
\item There are fixed values for the operator symbols occuring in practically any language
  (\lstinline[style=inline]{|^&=!<>:+-*/%}, in increasing order); all other non-alphabetic symbols
  equally bind stronger
\item There are exceptions for assignment operators, ending in |=|
\item There are exceptions of these exceptions for comparison operators
\item The default associativity is left, unless the operator ends with |:|
\end{enumerate}

It is also possible to define unary prefix and postfix operators. Postfix operators are not really a
special case, as they fit into the dotless method calling style explained below; prefix operators,
however, as the negation operator |unary_!| in Line~4 of \autoref{lst:logic}, deserve to be
mentioned separately. These are declared by defining a method with name
\lstinline[mathescape]|unary_$\circ$|, where \(\circ\) is the name of the actual operator~--
currently, however, the allowed symbols are constrained to |+-!~|.

There is another question concerning operators, which is that of integrating them into existing
types when allowing mixed-type applications. For example, given a matrix class |Matrix|, it is
possible to implement |*| as a method of that class, such that for |m: Matrix|, the expression %
|m * 2| typechecks; however, the expression |2 * m|, looking equally well-typed, cannot be provided
simply, since the |*| operator of |Int| does not have any overload taking |Matrix| arguments, and
|Int| cannot be just be \enquote{patched}. The usual workaround for this problem is to provide an
implicit conversion from |Int| to a wrapper class having the desired method;
\autosubref{sec:implicits} shows how that can be achieved in an \enquote{invisible} way, essentially
reintroducing the expression |2 * m| into the language without having to actually change any class.

\newthought{As soon as we realize} that operators can by reduced to simple method calls, we could
also go the other way round: using method names, like they were operators. This sounds like a
promising idea: expressions like |xs contains "bar"| become possible in this way. And in fact, such
\enquote{dotless} calls are a very commonly used style in Scala. They can help readablility in
different ways: for one, less-speaking method names like |map| or |withFilter| can be applied like
this in a pipeline-like fashion. The following is a quite typical example, taken from the practical
part of this work:
\begin{lstlisting}
  a.statements map (_.pretty) mkString ","
\end{lstlisting}
Or, even more readable (at least for the initiated):
\begin{lstlisting}
  require(nats forall (_ >= 0), "List contains negative numbers")
\end{lstlisting}
which defines a precondition on a value |nats| (a list of |int|s), by declaring that all its members
must be nonnegative (otherwise exiting with an error message).\footnote{Example from
  \protect\url{http://www.scala-lang.org/api/current/\#scala.Predef$} (visited on 2015-06-02).}%$

\begin{lstlisting}[style=floating, caption={Example of a ScalaTest \dsl{} for writing unit tests in
    a natural-language-like specification. This example is taken from \url{http://www.scalatest.org}
    (visited on 2015-05-21) and slightly modified. Also note the anaphoric use of \lstinline|it| in
    Line~10.},
  label=lst:scalatest]
  class ExampleSpec extends FlatSpec with Matchers {
    "A Stack" should "pop values in last-in-first-out order" in {
      val stack = new Stack[Int]
      stack.push(1)
      stack.push(2)
      stack.pop() should be (2)
      stack.pop() should be (1)
    }

    it should "throw if an empty stack is popped" in {
      val emptyStack = new Stack[Int]
      a [NoSuchElementException] should be thrownBy {
        emptyStack.pop()
      } 
    }
  }
\end{lstlisting}

This style of calling is often exploited in \dsls{} resembling phrases of natural language; such
usage can be pushed quite far, as \autoref{lst:scalatest} shows. When defining such an interface, a
standard trick is to construct wrapping objects closing over the relevant parameters. These objects
can then provide methods in an order and wording close to a natural utterance; even \enquote{no-op}
methods, like the |be| in the example, can then easily be included and interpreted. As an example,
we could define a kind of \enquote{filter pipeline} for lists in the following way:
\begin{lstlisting}
  def map[A, B](f: A => B) = new {
    def over(xs: List[A]) = new {
      def skipping(p: A => Boolean) = xs filter (p andThen (!_)) map f
    }
  }
\end{lstlisting}
% alternative:
% def map[A,B](f: A => B) = new {
%   def over(xs: List[A]) = new {
%     def skipping(p: A => Boolean) = xs filterNot p map f
%   }
% }
In this case, |map| closes over a function and creates an anonymous object with a method |over|,
which closes over a list and returns another object with a |skipping| method, in which the
closed-over parameters are finally used to perform a filter-and-transform operation over the
list. These methods can then be used in such a style:
\begin{lstlisting}
  scala> map ((_: Int) + 2) over List(1, 2, 3, 4, 5) skipping (_ < 3)
  res0: List[Int] = List(5, 6, 7)
\end{lstlisting}
Unfortunately, in this case, an explicit type annotation is necessary in the first lambda
expression, because of Scala's local type inference; however, in \dsls{}, parameters will often be
of fixed custom types, relieving this effect. The example also uses |reflectiveCalls| to allow using
anonymous types for shortness; in real code, every returned interface should be explicitly given by
a trait. Using traits for this style also has the advantage that different \enquote{paths} of
calling can be restricted by defining methods only as extensions on mixin combinations, and not
directly on the traits themselves.

\newthought{There are some additional} syntactic features to be mentioned in this subsection. These
concern the ability to define getters and setters. It is a very common style in object-oriented
languages not to expose fields of a class directly, but to provide instead methods like %
|void setFoo(T value)| and |T getFoo()| to encapsulate the underlying value. This also has the
advantage that invariants of the internal state can be ensured using pre- and postconditions, and
more fine-grained access control is possible (like disallowing setting). On the other hand, such
methods are an overhead: they are not standardized, and introduce the look of a method call where
logically |foo = t| should be enough.

Other languages, such as Python and \csharp{}, do have the ability to implement overloaded getting
and setting of field assignment syntax, (with the |get|/|set| statements in \csharp{}, and the
|__getattr__|/|__setattr__| magic methods in Python). This is very commonly used to ensure
invariants on the values, but also to provide \enquote{virtual} fields, which instead of wrapping a
variable are computed on-demand from other information. Now, when we already have overloaded member
assignment, there naturally comes the desire to overload indexed (array member) assignment,
too. While in \csharp{} and Python, it is possible to overload |()| and |[]| separately, in Scala,
these are both unified under |()|, since square brackets are already in use for type arguments.

\begin{lstlisting}[style=floating, 
  caption={A mutable wrapper around an immutable dictionary type \lstinline|DictImpl|,
    ensuring the size to stay below a fixed capacity. The capacity can be reset, but only
    increasingly~-- it always has to be bigger than the current actual size, which is ensured in the
    setter as a precondition.
    \hfill\github{dsl-examples/blob/master/src/main/scala/dsl_examples/MutableDict.scala}},
  label=lst:mutable_dict]
  class MutableDict[T](private[this] var c : Int) {
    private[this] var dict: DictImpl[T] = DictImpl.empty
    private[this] var size: Int = 0

    def update(key: String, value: T): Unit = {
      if (dict.get(key).isEmpty) {
        require(size < capacity, "Capacity exceeded!")
        size += 1
      }
      dict = dict.updated(key, value)
    }

    def apply(key: String): Option[T] = dict.get(key)

    def capacity: Int = c
    def capacity_=(newCapacity: Int): Unit = {
      require(newCapacity >= size, "Capacity too small!")
      c = newCapacity
    }
  }
\end{lstlisting}

To examplify Scala's syntax for all this, consider \autoref{lst:mutable_dict}. It contains a small
mutable wrapper around a (not further specified) immutable dictionary implementation, and has four
methods: for updating and retrieving dictionary entries, and for getting and setting the maximum
capacity of the dictionary. Retrieving an entry is implemented using the method |apply|
(Line~13). This method is always used when an object is called like a function: in this case, given
|d: MutableDict[Int]|, we could say |d("foo")| to get out the value at |"foo"|. The application gets
translated to |d.apply("foo")|, which resembles the fact that indexing is not logically different
from calculating a function value. Because of this, |apply| is also the whole magic that lies behind
the |Function| interfaces of the standard library.

Consistent with this, we can also write |d("foo") = 42|, to update that entry. This is achieved by
providing |update| (Line~5), which is similar to apply, but takes as an additional parameter the new
value, and returns |Unit| (since updating is only a side effect; it is also common to return the
updated dict here, to allow method chaining). In this case, the updating operation also keeps track
of the size, and ensures that the dictionary does not exceed its maximal capacity. Both |update| and
|apply| can in principle take multiple parameters of any types, which allows to implement complex,
multi-dimensional access, if necessary.

The examples for field assignment are Lines 15 and 16. The getter of |capacity| is just a
parameterless method with the same name, which is not a special syntactic feature. But it is also
possible to accompany any such a method, having name \(x\), with a setter function, called
\lstinline[style=inline,mathescape]|$x$_=|, which is used for assignment syntax; given such a
method, \lstinline[style=inline,mathescape]|obj.$x$ = y| will be translated to
\lstinline[style=inline,mathescape]|obj.$x$_=(y)|. In the above case, |capacity_=| has the
additional effect of ensuring that the newly set capacity cannot be less than the current size.

%--------------------------------------------------------------------------------
\section{Advanced Parameter Passing}
\label{sec:parameter_passing}

Scala has a lot of syntactic features that make it easier to write \dsls{} in a very flexible, but
at the same time intuitive way. Since we now have seen how conventional calling chains can be broken
up using infix notation, we can complement this advantage with various ways of \enquote{advanced
  parameter passing}. Combining the techniques of both kinds leads to a style that looks a lot like
a were native syntax, but is actually only syntactic sugar for various forms of method calls.

\newthought{One important concept} of advanced parameter passing is that of a \emph{block}. With
blocks, Scala generalizes such notions as \enquote{bindings}, \enquote{local closures}, or similar
constructs. The concept is quite simple: an expression of the form
\begin{lstlisting}[mathescape]
  {
    stmt$_1$
    $\vdots$
    stmt$_n$
    expr
  }
\end{lstlisting}
creates a local scope (like in other curly-brace languages), which, when evaluated, results in the
sequenced side effects of the |stmt|$_i$\,; but at the same time, the whole block represents an
expression soleley consisting the value of the last expression |expr| (modulo bindings
introduced). This is, first of all, useful for defining local variables and occasionally
interleaving side effects (\eg, for debugging):
\begin{lstlisting}
  val x = foo {
    val y = some + complicated(nested, expression)
    println("Debug: " + y)
    y
  }
\end{lstlisting}
which is denotationally equal to
\begin{lstlisting}
  foo(some + complicated(nested, expression))
\end{lstlisting}
but also will print out the intermediate value of |y|.

Above, we can also observe a second important syntactic feature: unary function calls, which usually
require the argument to be put in parentheses, can be called directly on a block (|foo| here would
just have a type like |T1 => T2|). This style allows the user to program against a library like it
consisted of language statements~-- given that the library was intended to be used as such. Still,
often this possibility is not enough to provide a natural interface, because the default evaluation
order of application does not correspond to what is expected by the reader of the code. How this can
be resolved is examplified in the next subsection.

Blocks can also be used as result of lambda expressions, relieving the language from providing
binding constructs such as |let| forms and generally unifying the style of programs. In the
following, we define a function |f|, which calculates |x| as above, but with |nested| provided
through an argument:
\begin{lstlisting}
  val f: (A => B) = nested => {
    val y = some + complicated(nested, expression)
    println("Debug: " + y)
    foo(y)
  }
\end{lstlisting}

\begin{lstlisting}[style=floating,floatplacement=t,
  caption={Scheme-inspired \lstinline|Delayed| implementation, using a function from
    \lstinline|Unit| and and a mutable variable for
    caching.\hfill\github{dsl-examples/blob/master/src/main/scala/dsl_examples/delay/Delay1.scala}},
  label=lst:delay]
  class Delayed[+T](thunk: Unit => T) { 
    private[this] var cached: Option[T] = None

    def force: T = cached match {
      case Some(value) => value
      case None => {
        val value = thunk(())  // evaluation happens here
        cached = Some(value)
        value
      }
    }
  }
\end{lstlisting}

\newthought{In most programming languages}, parameters passed to a method are evaluated when the
method is called. That is, in a statement like
\begin{lstlisting}
  foo(bar(), y + z, { log("hi!"); foo })
\end{lstlisting}
every one of the arguments will show its side effects at the time |foo| gets called. As mentioned,
however, this will sometimes be a behaviour that we want to avoid. To illustrate the concepts to
resolve this, in the following an implementation of \enquote{delays} or \enquote{thunks} will be
used~-- a means to encode lazy evaluation into a strict functional language, having long been used
in Scheme~\cite[][Chapter~4.2]{abelson1996:structure} and going back to the examination of lazy
evaluation orders of lambda calculus.

The original idea is to simply represent a thunk of type |T|, that is, a value to be caluculated by
need, as a function of type |Unit => T|, where |Unit| is the trivial type containing only one value,
|()| (also pronounced \enquote{unit}). This way, instead of passing around some |T|, which of course
would be evaluated immediately, we can hold a function which can compute the desired |T| later, if
we actually need its value (this is usually done using a function called \enquote{force}). Once
evaluated, the value is also cached, so that expensive evaluations happen at most once. One way of
translating this into Scala would be the one shown in \autoref{lst:delay}.

Here, |private[this]| is a Scala specific scope modifier marking |cached| as \emph{object private}:
that way, not even objects of the same class can access the field. This restriction is even stronger
than what is possible in most other object-oriented languages, but serves an excellent purpose of
hiding mutable state, so that from outside, we can still speak of a purely functional (thus
immutable) object. To compare with \CC{}, this has a similar spirit to marking fields used for
caching |mutable|, but leaving the accessors |const|~-- holding the promise of referential
transparency lies in the programmers responsibility, but can improve performance if used wisely.

Still, the above implementation leaves room for improvement. To construct a |Delayed| value, one
needs to manually call the constructor on a an anonymous function:
\begin{lstlisting}
  val d1 = new Delayed((_unit) => {
    println("heavy computation")
    2
  })
\end{lstlisting}
We even need to name the superfluous |Unit| argument. Additionally, while the construction as shown
is practically immutable from outside, it would be desireable to even get rid of such small traces
of impurity. 

Fortunately, both aspects can be improved by the help of the language. For this, we can use two new
features: \emph{call-by-name} functiona arguments (indicated by the |=>| before the type) and
\emph{call-by-need} values (|lazy val|). Using them, we end up with the code from
\autoref{lst:delay_improved}, which is shorter, and at the same time purer and more easily usable.
\begin{lstlisting}[style=floating,
  caption={Improved \lstinline|Delayed| implementation, using Scala's laziness and block
    constructs. The \lstinline|delay| \enquote{factory} is declared inside the companion object,
    while the default constructor of \lstinline|Delayed| is set private.
    \hfill\github{dsl-examples/blob/master/src/main/scala/dsl_examples/delay/Delay2.scala}},
  label=lst:delay_improved]
  class Delayed[+T] private (thunk: () => T) {
    private[this] lazy val cached: T = thunk()
    def force: T = cached  // evaluation happens here implicitly
  }

  object Delayed {
    def delay[T](delayed: => T) = new Delayed(() => delayed)
  }
\end{lstlisting}
Using that, we can write the following:
\begin{lstlisting}
  val d2 = delay {
    println("heavy computation")
    2
  }
\end{lstlisting}

The changes introduced in this improved version are to use Scala's laziness constructs: the |cached|
value is now a |lazy| reference to the evaluation of the thunk, and |force| simply returns the value
directly~-- which in turn evaluates the thunk. The subtle difference between call-by-name arguments
and |lazy| is the caching behaviour: while both are evaluated non-strictly, a |lazy val| is stored,
as soon as it has been evaluated, while by-name parameters are reevaluated every time (internally,
they are converted to |def|s like the |thunk| representation). It is the interplay of these two
strategies that makes this way of argument passing so useful.

Furthermore, a factory method |delay| is provided, which takes as an argument an unevaluated |T| and
wraps it into the old constructor taking a function, totally hiding the underlying implementation~--
consequently, the actual constructor is now marked |private|. This unevaluated |T| is intended to be
provided as a block, a form in which form the code now barely looks different from a native language
construct.\footnote{Thinking this example further, we could just get rid of \lstinline|Delayed| at
  all and only use call-by-name and lazy values, since they in fact use the same techniques under
  the hood. The purpose of this was, however, to show the possibilities of defining new syntax-like
  methods, not of seriously implementing behavioural data structures.}

\label{sec:currying}
\newthought{As can be seen} in the last example, it is possible for functions to be called on blocks
(with braces), which makes them look like they were language statements. In fact, every application
to a single argument can be replaced by a block. When designing \abbrev{API}s in Scala, this
possibility is regularly exploited, and also occurs frequently in the standard library. But often,
we also want to simulate a parametrized statement, like in in this example:
\begin{lstlisting}
  repeat(5) {
    println("hello!")
  }
\end{lstlisting}
which would just print out |"hello!"| five times.

\begin{lstlisting}[style=floating,
  caption={Simple combinator for repeating an action \lstinline|n| times, using a curried parameter
    list. The \lstinline|block| needs to be passed by name, as it is evaluated in each iteration for
    its side effects. 
    \hfill\github{dsl-examples/blob/master/src/main/scala/dsl_examples/Imperative.scala}},
  label=lst:repeat]
  def repeat(n: Int)(block: => Any): Unit = 
    if (n > 0) {
      block
      repeat(n - 1)(block)
    }
\end{lstlisting}

Defining methods with this interface is possible as well, if we take into account the more accurate
description of the \enquote{block application} syntax: namely, that the \emph{last single
  application} of the parameter lists of a method can be replaced by a block. From this it follows
that we can use \emph{curried methods} to write \enquote{pseudo-statements} with parameters. An
implementation of the example |repeat| function is shown in \autoref{lst:repeat} (which also is
another example of call-by-name).

Curried methods are methods with multiple parameter lists, written like this:
\begin{lstlisting}[mathescape]
  def fn(x$_{11}$: X$_{11}$, ..., x$_{1k_1}$: X$_{1k_1}$)...(x$_{n1}$: X$_{n1}$, ..., x$_{nk_n}$: X$_{nk_n}$): Y
\end{lstlisting}
They are included into the language to simplify partial application, of which the above use case is
an instance. Partial application is the mechanism allowing to \enquote{fix a parameter} of a
function; that means, if we have |f: (A, B) => C| and some given |a: A|, then we could write %
|g: B => C = (b => f(a, b))|. But this can be expressed simpler, if |f| is curried: from %
|f2: A => B => C|, we can make |g2: B => C = f(a) _|. (That these are always interchangeable is a
consequence of the fact that there is an isomorphism (to be exact, an adjunction) between all types
|(A, B) => C| and |A => (B => C)|.)

This does not by itself reveal immediate usefulness (especially since there has to be included a
\enquote{dummy} |_|), but is a practical feature when dealing with higher-order
functions. Additionally, besides the described syntactic trick, there are other cases in which
curried functions are helpful: \eg, they can help with type inference, since (speaking simplified)
for each parameter list, there is a separate type unification step, after which some type parameters
can be fixed (this is often exploited in multi-parameter polymorphic higher-order functions such as
|foldRight|/|foldLeft|). Similarly, implicit method parameters (explained later in
\autosubref{sec:implicits}) are a form of curried parameters, and the implicit parameter list is
resolved separately from the other parameters.


%--------------------------------------------------------------------------------
\section{Monads and For-Comprehensions}
\label{sec:monads}

In the previous subsection, a (rather primitive) implementation of delayed values was shown. This
implementation was already able to do what is expected, and provides an intuitive construction
interface; however, in its minimal state, composing different delayed actions quickly becomes
tedious. Imagine a scenario in which we have ways to query a remote database and a web \abbrev{API},
both in a delayed way (using something like the above implementation):
\begin{lstlisting}
  def getUrl(id: String): Delay[String] = delay {
    dbConnection.getId(id).address
  }
  def getFromApi(url: String, path: String): Delay[String] = delay {
    webRequest(url).get(path)
  }
\end{lstlisting}
Now, if we want to combine these requests, we would have to do the following:
\begin{lstlisting}
  val getInformation(id: String): Delay[String] = delay {
    val address = getUrl(id).force
    val info = getFromApi(address, "/info").force
    "Address " + address + " has info " + info
  }
\end{lstlisting}
Here, |force| needs to be called twice inside the |delay| block to unwrap the results of the nested
delayed calls, which are then combined and wrapped again.

This was just an easy case. If there happen to be more layers, or we have to interleave other
operations such as validations of data or error handling, the repeated nesting of |delay| and
|force| does not anymore keep its conciseness, but starts to look awkward and even becomes a source
of error and confusion (a result which is known as \enquote{callback hell} in the JavaScript
community).

The complication introduced by combining this style of operations, involving types of the form %
|T => M[T]| where |M[_]| is a kind of \enquote{context} (in this case, the delay of values), has
been known for long; and there is a pattern to resolve it, based on the observation that most of the
time, the |M[_]| used will be a \emph{monad}.

A monad is a category-theoretic construct on functors, requiring them to be able to be transformed
in certain ways according to some laws. A theoretically founded treatment, which still refers to the
programmer's point of view of \enquote{context containers}, can be found in \cite{moggi1991:monads};
but in essence, a monad can be described as a parametrized type |M[_]|, with a \enquote{constructor
  function} of type |A => M[A]| for all A (in the following called |pure|), such that on every
|M[A]| there are methods
\begin{lstlisting}
  def map[B](f: A => B): M[B]
  def flatMap[B](f: A => M[B]): M[B]
\end{lstlisting}
subject to the following conditions:
\begin{enumerate}
\item |m map (x => g(f(x)))| \(\,\equiv\,\) |m map f map g|
  % \\ \(\forall (\text{\lstinline[style=inline]|m: M[A]|}), (\text{\lstinline[style=inline]|f: A =>
  %   B|}), (\text{\lstinline[style=inline]|g: B => C|})\)
\item |pure(x) flatMap f| \(\,\equiv\,\) |f(x)| 
  % \quad \(\forall (\text{\lstinline[style=inline]|x: A|}), 
  % (\text{\lstinline[style=inline]|f: A => M[B]|})\)
\item |m flatMap (pure _)| \(\,\equiv\,\) |m|
  % \quad \(\forall (\text{\lstinline[style=inline]|m: M[A]|})\)
\item |(m flatMap f) flatMap g| \(\,\equiv\,\) |m flatMap { x => f(x) flatMap g }|
 % \\ \(\forall (\text{\lstinline[style=inline]|m: M[A]|}), (\text{\lstinline[style=inline]|f: A =>
 %    M[B]|}), (\text{\lstinline[style=inline]|g: B => M[C]|})\)
\end{enumerate}
where |m| is a monadic value of type |M[A]|, and |f|, |g| are \enquote{actions} of types |A => M[B]|
and |B => M[C]|.

It is the type of |flatMap| that allows to sequence actions of types like |A => M[B]| in a concise
form. The \enquote{purpose} of the monad laws thereby is to ensure that using |flatMap| behaves
sensibly with respect to the structure of the underlying type; they are also the responsible for the
ability to rearrange the below-mentioned comprehension syntax in a meaningful way.

\begin{lstlisting}[style=floating,
  caption={Full example of the \lstinline|Delayed| implementation, with monadic
    functions. Additionally, the \lstinline|Delayed| class was replaced by a trait, of which
    anonymous instances are created by the factory \lstinline|delay|.
    \hfill\github{dsl-examples/blob/master/src/main/scala/dsl_examples/delay/Delay3.scala}},
  label=lst:delay_monadic]
  sealed trait Delayed[+T] {
    import Delayed._

    def force: T

    def map[U](f: T => U): Delayed[U] = delay {
      f(this.force)
    }

    def flatMap[U](f: T => Delayed[U]): Delayed[U] = delay {
      f(this.force).force
    }
    
    def foreach(f: T => Unit): Unit = f(this.force)
  }

  object Delayed {
    def delay[T](delayed: => T) = new Delayed[T] {
      private[this] lazy val cached: T = delayed
      def force: T = cached
    }
  }
\end{lstlisting}

\newthought{The point of using monads} is that Scala provides so-called \emph{for-comprehensions}
for them for free, for any type implementing the necessary interface (usually, |map| and |flatMap|;
possibly also |foreach| and |withFilter|). These comprehensions generalize the |for|-syntax of the
language to arbitrary monadic types, allowing one to write nested (flat)mapping of functions in a
linear, somewhat imperative-looking way. As an example, in \autoref{lst:delay_monadic}, the
|Delayed| class of the last subsection has been rewritten to support a monadic interface. Using the
comprehension syntax with this implementation, the previously mentioned example can be rewritten
like this:
\begin{lstlisting}
  val getInformation(id: String) = for {
    address <- getUrl(id)
    info <- getFromApi(address, "/info")
  } yield ("Address " + address + " has info " + info)
\end{lstlisting}
This comprehension gets translated as follows:
\begin{lstlisting}
  getUrl(id)
  .flatMap(((address) => getFromApi(address, "/info"))
  .map((info) => "Address " + address + " has info " + info)
\end{lstlisting}
Desugaring for-comprehensions happens before type checking and does not ensure any behavioural
guarantees, so it practically amounts to duck-typing of whatever values are used. The exact way of
translating is documented in the language specification
\cite[][Chapter~6.19]{odersky2014:scala_spec}.

In fact, the |Future| type of the standard library (|scala.concurrent.Future|) is not so much
different from this version of |Delayed|, except that it runs thunks not later on demand, but
asynchronously to the current context (an implicit |ExecutionContext| is passed with most methods on
futures). Futures, too, have a factory called |Future| and implement the monadic interface, and it
is considered good style to use the available combinators implemented on it, instead of nesting
|Future| calls and blockings.

\newthought{The type of behaviour} that is provided by monads can be compared to \abbrev{SQL}
statements, which is also the reason that for-comprehension are used mostly for collections (for
comparison, the \csharp{} equivalent of for-comprehensions, called \abbrev{LINQ} (Language
Integrated Query)\footnote{Monad comprehensions as \enquote{generalized for} have found their way
  not only into \csharp; the idea, initially implemented in Haskell's \emph{do-notation}, is also
  present in \fsharp's \emph{computation expressions}, which generalize the concept even more by the
use of continuation-passing techniques \cite[][p.~62]{syme2012:fsharp}.},
even uses \abbrev{SQL} keywords like |from| and |select|). The type of |flatMap| allows to express
\enquote{nested selection} and \enquote{joins} on arbitrary \enquote{containers}, with whatever
semantics make sense for them: sequencing, nondeterminism, delayed continuation, and so on. For
example, we can provide the following combinator:
\begin{lstlisting}[label=lst:join]
  def join[A, B](a: Iterable[A], b: Iterable[B]) = new {
    def by[C](f: (A, B) => C): Iterable[C] = for {
      x <- a
      y <- b
    } yield f(x, y)
  }
\end{lstlisting}
which can be used like this\footnote{Calling this function will lead to a warning signalling that
  this only works using the compiler flag\kern-1ex \lstinline|reflectiveCalls|\kern-0.9em. Using a
  trait instead of an anoymous structural type would be considered better style, and is probably
  safer, but requires more setup than this example. The function also does not work really well with
  the collections hierarchy, as it does not properly preserve container types.}:
\begin{lstlisting}[style=break-lines]
  scala> join(List(1, 2, 3), List('a', 'b', 'c')) by ((_, _))
  res0: Iterable[(Int, Char)] = List((1,a), (1,b), (1,c), (2,a), (2,b), (2,c), (3,a), (3,b), (3,c))
\end{lstlisting}
(Note also the trick of introducing the intermediate object with a |by| method, which is called in
infix notation!)

The same code, replacing |Iterable| with |Future|, can be used to get a combinator with
\enquote{await both} semantics for Futures; in fact, it would make some sense for any monad. For
this reason, monads are often subsumed under a type class (\autosubref{sec:type_classes}), and there
are libraries like
Scalaz,\footnote{\protect\githubcommit{https://github.com/scalaz/scalaz}{/tree/24cf7f3dcf9baa9ea438269eea6b24fd9476a544}
  (visited on 2015-05-16).} providing a range of general methods and combinators for all monads (and
many other general classes). Another notable monad, which is not immediately appearent as such, is
the |Parser| type of parser combinators (\cf~\autosubref{sec:combinators}), for which the monad
provides conditional and error-preserving sequencing of parsing functions.

There is one additional thing to say about the use of the term \enquote{monad} or \enquote{monadic}
in Scala or other languages supporting syntactic sugar for them: sometimes, for a type, the monad
syntax (\ie, the comprehensions) still makes sense, even if the monad laws are not strictly
fulfilled. For example, |scala.util.Try| from the standard library does not behave the same on both
sides of the second above-mentioned law, because of its special exception-capturing behaviour. Or,
types for random variables, which are monads too, do not fulfill the laws by conventional equality;
instead, they only guarantee the left and right hand sides of the equations to have identical
probability distributions. But in all these cases, using comprehension syntax is nontheless useful
and adequate enough to not prohibit it. For that reason, also such types are still commonly called
\enquote{monadic}.

% \begin{lstlisting}
%   for {
%     a <- delay { println("a"); 1 }
%     b <- delay { println("B"); 2 }
%   } yield (a + b)
% \end{lstlisting}
% %
% is translated to:
% %
% \begin{lstlisting}
%   delay({
%     println("a");
%     1
%   }).flatMap(((a) => delay({
%     println("B");
%     2
%   }).map(((b) => a.$plus(b)))))
% \end{lstlisting}


%--------------------------------------------------------------------------------
\section{Working with Literals}
\label{sec:literals}

When developing custom data types, we want them to be usable as naturally as possible, especially in
the context of \dsls. This includes working with values of them in an intuitive and integrated
way~-- namely, handling custom literals should not be different from handling \enquote{primitive}
values, and ideally, custom types should be able to be written exactly as natural in their
domain. Scala also provides some means to achive this: first, it has the possibity to provide custom
\emph{string interpolators}; and furthermore, new \emph{extractors} can be defined for arbitrary
types, allowing to create new pattern matching ability without having to define case classes.

\newthought{String interpolators} are, for one, a way of providing readable constructors of data
types through strings, essentially by decomposing normal string literals interleaved with arbitrary
expressions into a function application. There are three interpolators provided by the Scala
standard library: |s|, |f|, and |raw|~\cite{suereth:string_interpolation}. The first one does not do
much more than providing syntactic sugar for string building (using |toString|):
\begin{lstlisting}
  val x = 123
  val y = (true, List(1, 2, 3))
  val str = s"Number: $x, boolean: ${y._1}"
\end{lstlisting}
Here, |str1| will result in the string |"Number: 123, boolean: true"|. As can be seen, the
interpolator escapes identifiers starting with |$|; and in general, arbitrary block expressions can
be written inside curly braces following a |$|. So, continuing the above example, we can write an
expression like
\begin{lstlisting}
  s"""The sum is ${if (y._2.sum % 2 == 0) "even" else "odd"}"""
\end{lstlisting} %$
which evaluates in this case to |"The sum is even"| (the quotes inside braces are automatically
\enquote{escaped}, since desugaring happens independently before literal parsing). The other two
standard interpolators work in a similar way, but additionally allow |printf|-style format
specifications or ignore escape sequences (like |\n|).

% The other interpolators have different semantics: |f| works similar to |s|, but additionally allows
% to specify |printf|-style format specifications after the expressions. For example, |f"$x%.02f"|
% will result in |"123.00"|. Similarly, |raw| ignores escape sequences: |raw"$x \n $y"| will result in
% |"123 \n (true,List(1, 2, 3))"|.

More interesting, however, is the fact that we can write our own interpolators. Every call to an
interpolator of the form |intp"some string"| gets rewritten by the compiler to a method call on
|StringContext|, which is essentially only a container for an array of strings: %
|StringContext("some string").intp|. If there are parameters in the string, these get passed to the
method: |intp"using $x's value"| becomes |StringContext("using ", "'s value").intp(x)|. %$

This rewrite rule is extensible insofar as we can add methods to |StringContext| through implicit
conversions (see \autosubref{sec:implicits}). These allow to (syntactically) extend classes without
inheriting from them. Using this technique, one can write new interpolators using a wrapper like the
following:
\begin{lstlisting}
  implicit class Bla(sc: StringContext): AnyVal {
    def foo(params: Any*) = fooImpl(params)
  }
\end{lstlisting}
This can then be applied to any string literal with an arbitrary number of arguments, like %
|foo"x $a y $b z"|, reducing to
\begin{lstlisting}
  Bla(StringContext("x ", " y ", " z")).foo(a, b)
\end{lstlisting}
While it is common and often useful to allow arbitrary parameters (|Any*| is internally represented
as a |WrappedArray|), an interpolator can in principle take any number of concretely typed
parameters. Since the interpolator call is translated to an equivalent method call anyway,
incompatible or insufficient calls are automatically refused by the type checker. And while the |s|
interpolator is relatively trivial (it just adds the parameters and the string chunks), a more
complex interpolator will often involve parsing the string which is interpolated, and even introduce
some extra internal syntax to modify the behaviour of the passed interpolants.

String interpolators are widely used for easier construction and deconstruction of text-based
data. Prominent examples are constructors for \abbrev{JSON}~\cite{voitot2013:json}, and recently
quasiquotes, which are used in macro programming and allow the construction of Scala \abbrev{AST}s
without having to construct them \enquote{by hand}~\cite{shabalin2013:quasiquotes,
  shabalin:quasiquotes}. With the help of macros, quasiquotes can also be pattern matched on. Even
the \abbrev{XML} literal syntax of Scala is planned to be replaced by interpolators in future
versions~\cite{odersky2015:scala_state}.

\newthought{Another important use} of literals, apart from constructing them, is deconstructing
them~-- that is, accessing their values. For that purpose, there is in principle already a solution:
just using members returning whatever is logically contained in a class. Now, since Scala has the
feature of case classes, which allow pattern matching, these will often be preferable; however, not
every datatype will be written as a case class. The most obvious example for that would be a class
implemented in Java. In other cases, the underlying implementation might be much more complicated
than the logical form of the concept, and is not wanted to be exposed; this would e.g. happen in a
set implementation. So, there arises the need for being able to define custom patterns.

Additionally, as soon as we have a means of introducing new patterns, we can do even more things
than were initially looked for: with newly definable patterns, we are also able to introduce
synonyms to other patterns (or means of deconstruction), or we can provide more content specific
matching of data, deconstructing objects in a semantically richer way.

\begin{lstlisting}[style=floating, label=lst:logic_extractors,
  caption={Extractor objects for disjunction and negation, as defined in
    \autoref{lst:logic}. Conjunction is analogous to disjunction. Note that, since there are no
    unary tuples in Scala, \lstinline|unapply| for \lstinline|!| only needs to return
    \lstinline|Option\[Expr\]|.
    \hfill\github{dsl-examples/blob/master/src/main/scala/dsl_examples/Logic.scala}}]
  object || {
    def unapply(e: Expr): Option[(Expr, Expr)] = e match {
      case Or(a, b) => Option(a, b)
      case _ => None
    }
  }

  object ! {
    def unapply(e: Expr): Option[Expr] = e match {
      case Not(a) => Option(a)
      case _ => None
    }
  }
\end{lstlisting}

The way Scala allows defining custom patterns is through \emph{extractors}. An extractor is an
object with an |unapply| method, taking as parameters the value that should be matched, and
returning an |Option| of a tuple of the possibly extracted values. The |None|-ness of the |Option|
signifies whether the match succeeded. As an example, look at the extractors in
\autoref{lst:logic_extractors}, which accompany the \dsl{} for logic expressions from
\autoref{lst:logic}. These definitions (plus the left-out one for |And|) allow to write an
evaluation method for |Expr| as follows:
\begin{lstlisting}
  def eval(env: Map[String, Boolean]): Option[Boolean] = this match {
    case Var(s) => env.get(s)
    case a || b => join(a.eval(env), b.eval(env)) by (_ || _)
    case a && b => join(a.eval(env), b.eval(env)) by (_ && _)
    case !(a) => a.eval(env).map(!_)
  }
\end{lstlisting}
(This uses the monadic |join| operation as described on \autopageref{lst:join}, adapted for
|Option|.)

Such extractors are also automatically generated for case classes, which is the reason they are
mainly used for data types that can be pattern matched on. On the other hand, there are types which
hide their internal implementation and expose only |apply| and |unapply| methods, for example
|Set|s. This allows to introduce an additional layer separating interface from representation.  A
different usage of them is to provide \enquote{additional views} on data. A nice example for this is
Scala's regular expression functionality, which uses extractors for retrieving the matched groups of
a regex. Consider the following definition of a |Regex| object~\cite[][p.~611]{odersky2008:programming}:
\begin{lstlisting}
  scala> val decimal = """(-)?(\d+)(\.\d*)?""".r
  decimal: scala.util.matching.Regex = (-)?(\d+)(\.\d*)?
\end{lstlisting}
The |decimal| object now has a variadic |unapply| method, which allows to extract all groups (inside
parentheses) to be assigned to valus in a pattern matching context:
\begin{lstlisting}
  scala> val decimal(sign, integerpart, decimalpart) = "-1.23"
  sign: String = -
  integerpart: String = 1
  decimalpart: String = .23
\end{lstlisting}
This can be considered much more readable than using \enquote{manual} methods of finding and
extracting groups from a match. It also shows a style of using extractors depending not so much on
the data matched on, but parametrized by external information.

%--------------------------------------------------------------------------------
\section{Implicit values}
\label{sec:implicits}

% |"bar".foldLeft(0)((a,_) => a+1)|
Scala has the curious property that it seemingly allows methods to be called on objects on which
they are not defined, or parameters to be passed in an automatic, scope dependent way without
mentioning them~-- but all that in a strongly-typed way. This works through the system of
\enquote{implicits}. They are what allows to write |"""(-)?(\d+)(\.\d*)?""".r| without |r| being
defined in |java.lang.String|, or to call |List(("b", 1), ("a", 2)).sorted| without having to pass
an anonymous |Comparator| for |(String, Int)|.

The unified, \enquote{explicit} treatment of such implicit values, allowing \emph{type-directed
  implicit parameter passing}~\cite{oliveira2010:type-classes}, is a language feature quite unique
to Scala. It subsumes in a very powerful way notions existent in other languages, such as type
classes, extension methods, monkey-patching, and dynamic scope. Implicits in principle do not really
add anything to the core language semantically~-- they can always be reduced to explicit method
calls. However, they allow for some huge reductions of boilerplate code at the call site, for
example when converting between representations, extending funtionality, or passing around
configuration infomation or context. This feature is ubiquitously used in the standard and other
libraries, and several patterns of using implicits have been established.

Implicits come in two flavors: \emph{implicit conversions} and \emph{implicit parameters}. Both in
practice often interact with each other, but have different properties when it comes to
\enquote{discovery} and typing. The foundation of them and their usage are described below in this
subsection, in a more theoretic overview. Since some major applications, such as
\hyperref[sec:extensions]{extension methods} and \hyperref[sec:type_classes]{type classes}, are
later in this work explained individually, and other usages are spread throughout many examples,
there will be no larger pieces of example code here.

The basic building block of using implicits, both conversions and parameters, are
\emph{implicit declarations}. All |val|s, |var|s, |def|s, |object|s and |class|es can be marked as
|implicit| (as long as they are not at top-level), and only such marked declarations are subject to
implicit resolution. Underlying idea of both is that functions and parameters can be applied without
being explicitly named, if they can be uniquely determined from the surrounding \emph{implicit
  scope}. The exact rules of how this scope is determined are defined in the specification
\cite{odersky2014:scala_spec}; Section~6.26 of it describes implicit conversion, and Chapter~7 is
dedicated to implicit parameters (plus some features related to them). In the rest of this
subsection, these peculiarities are not discussed; they basically amount to searching surrounding
imports, companion objects and supertypes, and preferring strictly \enquote{closer} and
\enquote{more specific} declarations to more \enquote{distant} ones.

\pagebreak[4]
\newthought{Implicit conversions} act on the level of typing context and method names. They are
declared by providing an implicit value of a function type, that is, an implicit |def| or |val| with
an |apply| method of suitable type. If then an object in a context does not have a compatible type,
an implicit conversion is sought and, if available, automatically applied. For example, when we call
|qux(baz)|, where |baz: A|, but |qux| does only take parameters of type |B| (and it is not the case
that |A <: B|), an implicit conversion function |a2b: A => B| being in scope would be automatically
inserted, resulting in an effective call of |qux(a2b(baz))|.

Furthermore, if some method is called on an object, say, |foo.bar(1)|, and the type of |foo| does
not contain a method |bar| with that name and signature, then instead of immediately rejecting this
as an error, the compiler searches for suitable implicit conversions to types \emph{with that
  method}. If |foo| has type |A| and the result of |bar| in this context is inferred to be |B|, then
a suitable conversion is an implicit value of type |A => { def bar(i: Int): B }| (modulo subtyping
relations). When such a conversion, call it |fooWrapper|, can uniquely be found, it is automatically
applied; the resulting term behaves the same as |fooWrapper(foo).bar(1)|.

Conversions are subject to an important restriction: they can not be \enquote{stacked}; that is,
they are not, in a sense, transitive. If |A| is implicitly convertible to |B| via |a2b| in implicit
scope, and similar |B| to |C| by |b2c|, there does \emph{not} automatically exist the implicit
conversion |b2c(a2b(_))|. This is to simplify the rules and reasoning, but also to prevent
exponential search complixity at compilation.

\newthought{The use of such automatic conversions} is manifold: for one, they can simplify code in a
setting where datatypes are wrapped in layers, but different layerings essentially mean the
same. For example, in the propositional logic \dsl{} from \autoref{lst:logic}, |Var("x")| does not
contain any more information than just |"x"|, although it is still desirable to represent variables
in a dedicated type wrapper. By providing
\begin{lstlisting}
  implicit def stringToExpr(s: String ): Var = Var(s)
\end{lstlisting}
both advantages can be kept: when writing an expression, it is possible to use string literals alone
(like |"x" && ~"y"|), but internally, always the case class is used.

A similar use case is the conversion between essentially equal data types not between layers, but
between foreign libraries or equally prominent representation. For example, such conversions exist
in the Scala collections library under |scala.collection.JavaConversions| for the Java
\abbrev{API}'s equivalents of |Iterable|, |Set|, |Map|, and so on. They are also a very common means
to provide Scala-native interfaces to Java libraries using anonymous classes as callbacks; in that
way, given
\begin{lstlisting}
  implicit def function2ActionListener(f: ActionEvent => Unit) = 
    new ActionListener {
      def actionPerformed(event: ActionEvent) = f(event)
    }
\end{lstlisting}
one does not need to pass a literal |ActionListener| to a Swing event listener, but can directly use
an anonymous function, which is much more natural (example taken
from~\cite[][p.~444]{odersky2008:programming}). Such conversions are also defined in the |Prelude|
module for primitive types and their boxed equivalents (such as |char2Character|).

In a more complex form, implicit conversions to custom types can be used to virtually extend other
types, without affecting their actual implementation. This pattern is known under \emph{implicit
  wrappers} and described in more detail in \autosubref{sec:extensions}. It is quite regularly made
use of in the standard library to extend the (sometimes rather poor, or at least unconvenient for
Scala) interface of Java standard classes for all array types, primitive types, and strings, which
are, for example, augmented with higher-order collection methods.

\newthought{Implicit parameters}, in contrast to conversion, act by the automatic insertion of
method parameters. Parameters subject to implicit resolution must be declared in their own parameter
group (see \autopageref{sec:currying} on currying). There can be only one such group, which must
come last in the series of parameter lists. Take the following method, assumed to be defined in
class |X|:
\begin{lstlisting}
  def fn(a: A)(implicit b: B): X
\end{lstlisting}
Given values |x|, |a|, and |b|, |fn| can always be called in a regular fashion:
|x.fn(a)(b)|. However, when a value of type |B| is in implicit scope, say |implB| (declared, \eg, as
an |implicit val|), we are allowed to leave out the second parameter group, and just write
|x.fn(a)|~-- which automatically will again resolve in |x.fn(a)(implB)|. (It should be noted that it
is always possible to explicitly pass values into implicit argument positions, as they are just
syntactic sugar~-- so no generality is lost, if \enquote{overrides} are desired.)

This kind of parameter passing is mostly used when an interface requires to pass a lot of callbacks
or context objects. By making these explicit in the type, but implicit for the call site, the
boilerplate effort for the user of the interface can be reduced, as well as the readablility
improved, while no \enquote{interface information} gets lost. The classic example of this is the
above-mentioned need to pass a |Comparer| to Java's |Collections.sort|, when trying to sort
something which is not a base type. When the sorting method would take the |Comparer| as an implicit
argument, we could provide it implicitly once and for all for every new type, and never see it being
passed explicitly (except cases when we want to change sort order in special ways, which are however
better fitted for |sortBy|). In fact, Scala's standard library's |sorted| works exactly that way,
and takes an implicit |Ordering[T]| for the type |T| to be sorted. In \autosubref{sec:type_classes},
the generalization of this powerful pattern is explained in more detail.

A non-type-class example of implicit parameters is |scala.concurrent.Future|, most of which's
combinators take an implicit |ExecutionContext|, like the method
\begin{lstlisting}
  foreach[U](f: T => U)(implicit executor: ExecutionContext): Unit  
\end{lstlisting}
That way, the kind of asyncronous execution needed for executing concrete |Future| callbacks is
always available internally, but never has to explicitly passed around. It can be changed globally
by importing or defining a new implicit |ExecutionContext| in close enough scope.
\enlargethispage{1em}

\newthought{Finally, a word of caution}: using implicits in a careless way \emph{can} lead to very
hard to find sources of error. This happens especially if an implicit conversion unrealizedly is in
scope, applies to some value, and then fails. The author learned this the \enquote{hard way} when an
expression of the form |foo + 12|, which was mysteriously typed as |Int| in the \abbrev{IDE}, failed
at runtime with a |NoSuchElementException|~-- only after inspecting the bytecode, it could be found
out that there was an implicit |Map| from |foo|'s type to |Int| lying around, whose |apply| method
was treated as a conversion and came into effect on |foo| (which the |Map| did not contain); even
worse, the implicit value was even \enquote{leaked} through the external interface. The problem was
then resolved by wrapping the |Map| in a purely internal case class, and changing the type of the
implicit parameters for which it was originally thought accordingly.

To avoid such situations, two general advices concerning implicits should in almost any situation be
followed:
\begin{itemize}
\item Implicit conversion should never fail at runtime with exceptions, or otherwise.
\item Implicit parameters, result types of implicit conversions, and every implicit declaration,
  should have as specific and customized types as possible~-- if in doubt, they should be wrapped in
  a class just for that purpose.
\end{itemize}
In summary: let the compiler do the work for you~-- but always be in control of it\ldots

% Implicit conversions and implicit parameters, incl. default values. \enquote{Unit} pattern for
% numeric literals; dynamic implicit parametriziation (aka \enquote{dynamic scope})

%--------------------------------------------------------------------------------
\section{Macros}
\label{sec:macros}

In contrast to the other subsections, this one tries to give more of an overview and background of
its topics than examples and concrete use cases. This is because macros are relatively new, and will
probably change soon. Nevertheless, they are an important (and practically used!) \enquote{last
  resort} for certain kinds of problems in the scope of this work, and certainly deserve to be
mentioned in an overview of \dsl{} techniques.

\newthought{Macros are a traditional topic} when it comes to \dsls{}, though their prominence in
software development has declined over time. They are in their many forms a much more general way to
extend the expressiveness of a language, not only by adding seemingly impossible syntax, but
especially by providing semantic extensions via changing or introducing non-standard evaluation
behaviour, which they can do because they operate \emph{before} function evalulation. Macros in the
sense used here (as proper syntactic transformations or \emph{metaprograms}, not only text
replacement) were originally introduced in \abbrev{LISP}, where they come out in a natural way from
the fact that S-expressions (defining functions) are able to be applied to other S-expressions
(representing functions)~-- so they are, in a way, a lucky side effekt of the syntactic simplicity
of \abbrev{LISP}~\cite{mccarthy1960:recursive}.

As indicated, \enquote{proper} macros are rare in today's programming languages. Frequently,
especially in object-oriented languages, we find so called \emph{reflection} facilities in the
standard libraries, which allow to inspect and sometimes manipulate the representation of programs
at runtime. Or, there are \abbrev{API}s for the compiler or interpreter allowing to extend
them. Both of this has been existing in Scala practically for free since a long time, because of the
already available Java reflection
\abbrev{API}~\cite[][\protect\lstinline|java.lang.reflect|]{oracle:java_api_spec}. The reasons to
avoid macros are, for one, that in most languages different from \abbrev{LISP} complex reification
and representation support is needed, which is more complicated to handle than to just write code
differently; additionally, macros can be considered problematic from the point of view of certain
software development philosophies or styles, since they are hard to test and somewhat opaque. As an
interesting example of a language which does have a very structured approach to metaprogramming in
the sense used here, though, we could mention the statistics language
R~\cite{r2015}.\footnote{Specifically concerning R's metaprogramming capabilities, see
  \protect\url{https://cran.r-project.org/doc/manuals/r-release/R-lang.html\#Computing-on-the-language}
  (visited on 2015-11-14).}

Still, there recently has been put effort into implementing a proper macro system for
Scala~\cite{burmako2013:macros}, along with a quoting syntax~\cite{shabalin2013:quasiquotes}, which
significantly reduces difficulty and improves readability of macro implementations. These
implementations are available in newer Scala versions (since 2.10), and, although considered
experimental, are already widely used in libraries. There is currently a project called
|scala.meta|\footnote{\protect\url{http://scalameta.org/} (visited on 2015-04-26).} which aims at
improving and unifying all metaprogramming facilities available in Scala (including reflection) and
will provide a completely new \abbrev{API} in the future.

\newthought{The desire to write functions} that transform not their input values, but the whole
expression trees they are called on, is not as exotic as it may seem at first. In fact, the
well-known C preprocessor does exactly that, though in a very limited form~-- which is the reason
the \enquote{functions} that can be |#define|d are called macros, too.

People write macros at this level for two reasons: inlining, and non-evaluation. The first one is
quite discussable~-- nowadays, the overuse of preprocessor macros for \enquote{optimization} is
often frowned upon, as they can lead to strange errors when not thoughtfully crafted, and explicit
inlining is mostly considered useless with modern compilers, since their optimizations turn out to
be superiour to anything that can be influenced manually. In Scala, for this purpose, there is an
annotation |@inline| (and a corresponding |@noinline|) in the standard library, serving as hints for
compilers. Additionally, there have been put efforts into the improvement of the performance of
implicit wrappers, which have lead to \emph{value classes}~\cite{odersky2012:value_classes} (they
allow to discard a wrapper class at runtime, similar to Haskell's |newtype|s).
% There are often only hints in the form of annotations, like the C/\CC{} |inline| modifier
% and \abbrev{GCC}'s |__attribute__((always_inline))|, or Scala's |@inline| static annotation. These
% are, however, mostly hints to the compiler, which it often can safely choose to ignore, and shall
% not concern us here.

The second traditional reason to use macros, non-evaluation, has its classic use case in the writing
of debug functions and other condition-like code simulating statements, which cannot easily be
written in the form of functions, since parameter passing in conventional imperative semantics is
strict and always evaluates the arguments. However, as we have seen, this is easily possible in
Scala, by passing blocks, evaluated on demand, as call-by-name parameters, which is a very common
technique present everywhere in the libraries.

But going a bit further with the debug example, we soon arrive at the limits of regular syntax and
semantics. For example, it is impossible with non-macro functions to implement an assertion function
whose error message automatically contains the actual condition it was given; that is the reason
|assert| from the standard library takes a string as its second parameter. Obviously, it would be
desirable if the expression |assert(foo > 0)| would, on error, automatically print a message like %
|"Assertion 'foo > 0' was violated"|, extracting the text |foo > 0| from its argument. (This is
certainly possible with preprocessor macros, but unsafe, since their expansion happens before
compilation on the raw program text.)

With macros, we have the ability not to deal with values, but with expressions, and this problem can
be solved: we could take the given expression |foo > 0|, convert it to a string representation, and
return an expression containing a non-macro assertion with the |foo > 0| as condition and a message
built from the string representation:
\begin{lstlisting}[mathescape]
  macroAssert(foo > 0) $\rightsquigarrow$ assert(foo > 0, "Expected foo > 0!")
\end{lstlisting}

\newthought{As soon as we have macros} at our disposal, a lot more power than in this simple example
becomes available. In current Scala libraries using macros, they are typically used for two
purposes: to rearrange or check closures/blocks, and to automatically generate boilerplate
implementations. Both cases are normally used in \dsls{} (or even \enquote{meta-\dsls{}}, simplifying
the implementation of libraries).

The first of these use cases mostly involves transforming a closure into something more complicated,
which nevertheless can be recovered from a simple block of code. One typical example of this would
be the |async| syntax~\cite{haller2013:async}, which provides a \dsl{} for computations with
futures.\footnote{Code available at
  \protect\githubcommit{https://github.com/scala/async}{/tree/1568a28842e2c538ca735a34274ae5e4ee5eca22}
  (visited on 2015-05-17). The subsequent examples are also taken from the documentation provided
  there.} Remember that we can combine future values in various ways by using their monadic
interface, which gives the ability to use for comprehensions:
\begin{lstlisting}
  def slowCalcFuture: Future[Int] = ...
  val future1 = slowCalcFuture
  val future2 = slowCalcFuture

  def combined: Future[Int] = for {
    r1 <- future1
    r2 <- future2
  } yield r1 + r2
\end{lstlisting}
The library simply provides a macro |async|, and a marker method |await|, which allow the above
comprehension to be written like this:
\begin{lstlisting}
  def combined: Future[Int] = async {
    await(slowCalcFuture) + await(slowCalcFuture)
  }
\end{lstlisting}
This macro, instead of transforming the code to monad function calls, like it would be done with a
comprehension, builds an object containing a state machine which is registered at the futures'
|onComplete| handlers. This is also more efficient, since it does not involve multiple anonymous
classes for the intermediate closures.

The async/await pattern is also implemented in \csharp{} (since version 5.0), and is becoming
popular in other languages, too (\eg, Python~\cite{selivanov2015:coroutines}). However,
using macros, it can be implemented in Scala purely as a library~-- the underlying transformations,
as applied by the \csharp{} compiler and the |async| macro, are quite the
same~\cite{torgerson2010:asynchronous}.

In a similar direction goes the |spores| proposal~\cite{miller2013:spores}. Its purpose is to allow
an interface taking a closure to be guaranteed that no local references (like |this|) are captured
in a dangerous way. Such a guarantee is necessary for tasks like serialization of functions, or
passing around closures in distributed systems. This is done by wrapping a function literal in a new
type |Spore[-T, +R]|, but allowing them only to be constructed through a macro which enforces that
all captured variables are explicitly copied before usage (example taken from the cited
\abbrev{SIP}):
\begin{lstlisting}
  val s = spore {
    val h = helper  // explicitly copy this reference
    (x: Int) => {
      val result = x + " " + h.toString
      println("The result is: " + result)
    }
  }
\end{lstlisting}
Using any variables not defined in the header in the lambda expression will result in a compilation
error.

Many examples for the second popular macro usage, automatic implementation of \enquote{redundant}
code, can be found in the |shapeless|
library.\footnote{\protect\githubcommit{https://github.com/milessabin/shapeless}{/tree/b613c33a0cb5901e25e2f0e87926f7ce41e85c3d}
  (visited on 2015-05-17).} This library provides, among other things, a wide range of higher-rank
polymorphic combinators; that is, typesafe functions working on general products and coproducts,
like folds over arbitrary tuples. These things are mostly achieved by using powerful type classes
(\autosubref{sec:type_classes}), which do not, however, have to be provided by the user. Instead,
|shapeless| relies on \emph{automatic type class derivation}. This method is based on \emph{implicit
  macros}~-- which are just implicit instances, for which the code is automatically generated by
macros at compile time. These derivations usually just involve inspecting the code of case classes
and producing type class witness |object|s according to their shape and type.


%%% Local Variables: 
%%% TeX-master: "document"
%%% End:

\chapter{Patterns for DSL Implementation}
\label{sec:patterns}

This section is based on the previous one, and shows examples of how the Scala features introduced
there (and some more) can be combined into patterns typically used in \dsls. By \enquote{patterns},
not the strict architectural layouts, as used in object oriented design, are meant. Rather, this is
a collection of ways commonly used to achieve a certain goal: implementing \dsls{} in a
syntactically and functionally satisfying way. It is noteworthy that almost all of the patterns are
virtually native to Scala, in the sense that the features used were introduced for the very purpose
of simplifying their application: traits for mixins, implicits for type classes and extension
wrappers, the syntactic flexibility of method calls for combinators and other kinds of syntactic
\dsl{} helpers.

Additionally, as showcases of these patterns and the interplay of various Scala techniques, a number
of existing \dsl{} libraries will be mentioned or used as examples.

%--------------------------------------------------------------------------------
\section{Extensions and Rich Wrappers}
\label{sec:extensions}

One of the simplest forms of \dsls{} is to extend existing data types with new operations that are
frequently needed for the domain under consideration, to provide shorter functions to interoperate
with new types, or to combine values into domain constructs using more concise syntax.

One common example of this would be |Range|, as it exists in the Scala prelude. A |Range| object
represents a contiguous sequence of integers, which can be stored efficiently only by its first and
last value (and optionally, step size). The conventional way to construct such a value would
probably be through a factory object like |Range(1, 10)|, or, if we want to avoid ambiguities, %
|Range.inclusive(1, 10)|. This is, however, not as easy as it could be~-- there are ways of
providing special syntax for ranges, which can be more readable. For example, in Haskell, one could
write |[1..10]| for the same thing.

\begin{lstlisting}[style=floating, label=lst:range, mathescape, floatplacement=t, belowskip=-1em,
  caption={Implicit wrapper class to build ranges from integers, for a given type \lstinline|Range|
    with a construction function.
    \hfill\github{dsl-examples/blob/master/src/main/scala/dsl_examples/Range.scala}}]
  // suppose {in|ex}clusiveRange: (Int, Int) $\text{\ttfamily\color{textgray}=}$> Range
  implicit class IntRangeOps(self: Int) {
    def to(bound: Int): Range = inclusiveRange(self, bound)
    def until(bound: Int): Range = exclusiveRange(self, bound)
  }
\end{lstlisting}

Fortunately, Scala has implicits and a very practical method calling syntax. Using both, we can
write code like in \autoref{lst:range}. There, we simply create an implicit wrapper class, closing
over the first |Int| and giving it methods |to| and |until|, which then construct the respective
ranges with a second |Int|. Using such a class, we now
% we're faking a float here... what a dirty hack.
\par\newpage
\begin{lstlisting}[mathescape, frame=lines, basicstyle=\ttfamily\small, captionpos=b,
  style=numbered, xleftmargin=0.5ex, framexleftmargin=0.5ex, numbersep=0pt, belowskip=\baselineskip,
  label=lst:units, floatplacement=H, caption={Implicit wrapper to construct \lstinline|Value|s,
    which are type-safely dimensioned doubles. The primitive units are definded literally and can be
    applied in postfix notation; \lstinline|of| allows to select an arbitrary dimension as its type
    argument (such as \lstinline|9.8.of\[Meter / Second / Second\]|). Some units have been left out 
    here.}
  \hfill\github{dsl-examples/blob/master/src/main/scala/dsl_examples/Units.scala}]
  // suppose definition: class Value[D $\text{\ttfamily\color{textgray}<:}$ Dimension]
  implicit class dimensionSymbols(d: Double) {
    def meters = new Value[Meter](d)
    def seconds = new Value[Second](d)
    def amperes = new Value[Ampere](d)
    def coloumbs = new Value[Ampere * Second](d)
  }
\end{lstlisting}
have new syntax like |1 to 10|.\footnote{Note
  that, when trying this out in an interpreter in the shown form, there will be a conflict with the
  implicit conversions of the same names defined in the prelude. A solution is to simply rename the
  methods.} Implicit classes are just syntactic sugar of a class declaration with an additional
implicit |def| for the respective conversion; they were introduced specifially for this use case. A
lot of such functionality-patching extensions can be found in the Scala standard library. Besides
|to| and |until|, there are also |StringOps| and |WrappedString|, and similarly |ArrayOps| and
|WrappedArray|.

The code in \autoref{lst:units}\footnote{In this case, the accompanying example on Github is far
  more experimental than the others, as it contains an \emph{unfinished} system trying to proove
  equality of dimensions at type level.} contains a further example of how implicit wrappers are
often utilized: to provide a nicer construction experience than regular constructors. The implicits
listed there allow to write numbers in \abbrev{SI} units in a more natural way, \eg, |42.0 meters|,
instead of |new Value[Meter](42.0)|. Since there are more possibilities of units than just the
bases, two means of including them are shown: one could, for convenience, provide similar
constructors for all well-known combinations, like it is done in |coloumbs|; or, a general interface
like |of| could be provided, allowing to construct arbitrarily dimensioned values by a type
argument. A similar system (although not using type parameters) can also be found in the standard
library at |scala.concurrent.Duration| for time durations, which get mostly used for specifying
timeouts in concurrent environments.

There is one caveat when using many implicit wrapper classes: although all calls look like they
would work directly on the objects, they are often in fact passed through multiple layers of nested
calls, which can lead to performance loss. Such cases should of course be inspected with a profiler
before applying premature optimization; still, if necessarry, there are some tricks available, which
were already brought up in \autosubref{sec:macros}. For one, there is the |@inline| annotation,
which can be used to hint the compiler to optimize implicit conversion calls. Then, in newer Scala
versions, there are \emph{value classes} available~\cite{odersky2012:value_classes}. These allow, if
certain conditions are fulfilled, to erase wrapper types at runtime, and to replace implicit calls
by static applications. Furthermore, if this is not enough, macros can be used to specialize
implicit wrappers again, and get rid of some layers.



% \begin{lstlisting}[style=floating, mathescape, label=lst:units, floatplacement=H,
%   caption={Implicit wrapper to construct \lstinline|Value|s, which are type-safely dimensioned
%     doubles. The primitive units are definded literally and can be applied in postfix notation;
%     \lstinline|of| allows to select an arbitrary dimension as its type argument (such as
%     \lstinline|9.8.of\[Meter / Second / Second\]|).}
%     \hfill\github{dsl-examples/blob/master/src/main/scala/dsl_examples/Units.scala}]
%   // suppose definition: class Value[D $\text{\ttfamily\color{textgray}<:}$ Dimension]
%   implicit class dimensionSymbols(d: Double) {
%     def meters = new Value[Meter](d)
%     def seconds = new Value[Second](d)
%     def kilograms = new Value[Kilogram](d)
%     def mols = new Value[Mol](d)
%     def candelas = new Value[Candela](d)
%     def kelvins = new Value[Kelvin](d)
%     def amperes = new Value[Ampere](d)

%     def coloumbs = new Value[Ampere * Second](d)
%     def of[D <: Dimension] = new Value[D](d)
%   }
% \end{lstlisting}

%--------------------------------------------------------------------------------
\section{Type Classes}
\label{sec:type_classes}

Type classes are another pattern built on implicits. They are a language construct originally
introduced by Haskell, and comparable to interfaces in object orientation, in that they allow to
specify operations that a type must support. The main purpose of them is to constrain generic
functions and to write implementations against an interface; however, they are in a few ways more
general than conventional interfaces, and solve some peculiarities of them. Among type theorists,
type classes are said to unify advantages of both \emph{ad-hoc polymorphism} (overloading) and
\emph{parametric polymorphism} (type parametric functions); this is described in
\cite{wadler1989:ad-hoc}, while the terminology goes back to~\cite{strachey2000:fundamental}. A
detailed description of how Haskell's type class concept has been translated into Scala, including
an overview of their applications and generalizations using implicits, can be found
in~\cite{oliveira2010:type-classes}.

\begin{lstlisting}[style=floating, 
  caption={\lstinline|Read| type class, together with implementations for strings and booleans.~%
  \github{dsl-examples/blob/master/src/main/scala/dsl_examples/Read.scala}},
  label={lst:read}]
  trait Read[T] { 
    def read(s: String): T 
  }

  object Read {
    implicit object stringIsRead extends Read[String] {
      def read(s: String) = s
    }

    implicit object boolIsRead extends Read[Boolean] {
      def read(s: String) = s match {
        case "true" => true
        case "false" => false
      }
    }
  }
\end{lstlisting}

To illustrate how type classes work, and how they are different from interface inheritance (embodied
in Scala by \enquote{normal} trait inheritance), consider \autoref{lst:read}. This example shows a
|Read| type class~-- a concept from Haskell, which says \enquote{we can parse a value of a type from
  a string}. This possibility is represented by the parametrized trait |Read|. The crucial
difference from conventional interfaces is that some type |T|, implementing the type class, is not
supposed to inherit from |Read|~-- instead, a \emph{witness object} of type |Read[T]| should be
provided. The witness object, in this case encapsulating only one function, can then be used by
third parties to perform the desired operation. In the example, such objects are shown for the
trivial |String| instance, and for |Boolean|. When another method uses the type class to constrain a
parameter, the witness needs to be passed in as an additional parameter, which can then be used
inside the method to operate on the constrained type.

This sounds like a syntactic hassle at the call site, similar to Java's handling of, for example,
custom equality. And in fact, passing an anonymous |Comparer<T>| to a sorting method is just an
explicit variant of the concept of a type class. However, in both Haskell and Scala, type classes
are empowered to prevent such reduncancy: in both languages, the witness dictionary is passed
implicitly to functions requiring a parameter to be instance of a type class. In Haskell, this
passing is automatically resolved and inserted by the compiler~\cite{hammond1990:type_classes},
while in Scala, it is made explicit at the definition side by using |implicit| parameters. Consider
the following typical method definition:
\begin{lstlisting}
  def readAll[T](l: List[String])(implicit readT: Read[T]): List[T] =
    l map readT.read
\end{lstlisting}
This reads every element of a list into a specified type, if the elements implement |Read|. Calling
this function is simple, provided that the necessary implicit value is in scope:
\begin{lstlisting}
  scala> readAll[Boolean](List("true", "false", "false"))
  res0: List[Boolean] = List(true, false, false)
\end{lstlisting}
Of course, the result type always has to be provided explicitly, since it cannot be inferred from a
string. Similarly, one type class implementation can also depend on another one; for instance, all
|Numeric| instances can be |read|, at least from integers:
\begin{lstlisting}
  implicit def numericIsRead[N](implicit numN: Numeric[N]): Read[N] = 
    new Read[N] {
      def read(s: String) = numN.fromInt(s.toInt)
    }
\end{lstlisting}

\begin{lstlisting}[style=floating, label={lst:read_wrapper},
  caption={Wrapper object for the \lstinline|Read| type class.
    \hfill\github{dsl-examples/blob/master/src/main/scala/dsl_examples/Read.scala}}]
  implicit class StringReadOps(val self: String)
  {
    def readAs[T](implicit readT: Read[T]): T = readT.read(self)
  }
\end{lstlisting}

In this form, method calling looks seamless and unsuspicious, but there is still an overhead
involved every time the operations are actually used on values. For this purpose, type classes often
provide wrapper objects (see \autosubref{sec:extensions}) with operations for objects that can be
used as first argument of a method of a type class. The names of these often end with |*Ops|. For
the |Read| example, such a wrapper object is shown in \autoref{lst:read_wrapper}. In this form, the
explicit mentioning of |implicit| parameters is constrained to a minimum, and type classes can be
used as if they just provided the right methods on the right types~-- all without inheritance:
\begin{lstlisting}
  def readAll[T: Read](l: List[String]): List[T] = l map _.readAs[T]
\end{lstlisting}
This example uses one more feature to fully eliminate mentioning the implicit passing: |T| is
constrained by a \emph{context bound}, since we do not need the witness object itself
anymore. Context bounds are automatically converted by the compiler to implicit parameters, and come
very close to the type class syntax of Haskell. Still, they require more effort at the side of the
library, in that for seamless usage, an |Ops|-wrapper has to be written (otherwise, the user of the
type class can always obtain the witness explicitly by the prelude function
|implicitly[T]|~\cite[][Chapter~12.5]{odersky2014:scala_spec}).\label{ops}

Furthermore, there is one more thing type classes allow: They can be \enquote{stacked}, which means
using a type class constraint inside the definition of another instance. For example, when we want
provide a |Read| instance for maps, a suitable definition would have the following signature:
\begin{lstlisting}
  implicit def mapIsRead[K, V](implicit readK: Read[K],
                               readV: Read[V]): Read[Map[K, V]]
\end{lstlisting}
The implicit witness for the wrapping type requires itself a witness for the contained key and value
types. Note that this \enquote{stacking} of implicits does not fall under the rules of implicit
scope resolution for values, which allow only one level of implicit wrapping~-- chains of implicit
parameter requirements are followed arbitrarily deep.

\newthought{The type class pattern}, as over-complicated as it may sound on first sight, is not only
a syntactic trick. It has the following actual semantic advantages over trait inheritance when it
comes to code architecture:
\begin{itemize}
\item It does not only allow separating interface from implementation, but also separation of
  \emph{data implementation} (of the class that implements the interface) and \emph{functionality
    implementation}~-- most additional functions and combinators provided by the type class can have
  their own, third place. This leads to a second property:
\item Type classes allow to extend types independently of their declaration. That means, a type
  class implementation can always be added at a later time, and somewhere different from the
  original class. In that way, the provide a kind of \enquote{interface for extension methods}.
\end{itemize}

But type classes are also strictly more expressive than conventional subtyping
polymorphism~\cite{wadler1989:ad-hoc}. Consider a hypothetical trait for a semigroup:
\begin{lstlisting}
  trait Semigroup {
    def plus(other: Semigroup): Semigroup
  }
\end{lstlisting}
First, with this idea, every type being a semigroup would have to be declared as such at its
definition, by inheriting from the trait~-- there would be no way of \enquote{saying that
  later}. But more importantly, this code does not even correctly capture the functionality we want
to have, since it operates only on the |Semigroup| base type~-- and by subtyping, the respective
parameters can also be in every other semigroup. The operation could not even be implemented
properly in an inheriting class, since the parameter type is too general to access specific members
of |this|. This problem is known (remember the peculiarities of defining an |equals| method!), and
has lead to a solution using generics (or templates), known under the name \enquote{curiously
  recurring template pattern}~\cite{coplien1995:template}, in which a type inherits from an
interface with itself as a parameter (like |AddableInt extends Semigroup<AddableInt>|). This still
is somehow tedious and does not resolve all issues; in particular, it still cannot express a type
class like the following:
\begin{lstlisting}
  trait Monad[M[_]] {
    def pure[A](a: A): M[A]
    def join[A](mma: M[M[A]]): M[A]
  }
\end{lstlisting}
This is because neither of |Monad|'s methods are properly expressible as methods of some |Monad|
object~-- |pure| is a factory, which should probably be a static method; and |join| can only operate
on objects of \emph{nested} monad type, requiring a form of constraining |this|, which is not
possible to express in Scala (or the \abbrev{JVM} in general).

From this we see that, while traits only allow to specify methods of the form %
\lstinline[mathescape,style=inline]|T => X$_1$ => $\cdots$ => X$_n$|, where |T| is the base type,
type classes can specify an interface with functions where the constrained type occurs multiple
times and at any position.

Examples of type-class based libraries \enquote{in the wild} are
Spire\footnote{\protect\githubcommit{https://github.com/non/spire}{/tree/c4202668bf71e4b0b2bf694b15d9011c97c3874d}
  (visited on 2015-06-03).}, which is a numerics library using type classes to form a mathematical
hierarchy of algebraic constructs (such as semigroup \(\rightarrow\) monoid \(\rightarrow\) group
\(\rightarrow\) semiring), and to abstract out kinds of \enquote{number types}, like integral
numbers, rationals, or complex numbers. In a similar direction goes
Algebird\footnote{\protect\githubcommit{https://github.com/twitter/algebird}{/tree/8e4270b440c9dc0361f07f216f9cf5f064b32f86}
  (visited on 2015-06-03).}, which utilizes mostly monoids and semigroups to implement various
algorithms with generalized matrices on map-reduce style platforms.


%--------------------------------------------------------------------------------
\section{Combinator Interfaces}
\label{sec:combinators}

The extensive use of combinators (infix functions, mostly symbolic ones, to shorten expressions) for
\dsls{} has probably been popularized by Haskell. The reason for exactly this language can be found
in the theoretical background of its inventors and users, which for the most part are strongly
influenced by mathematics and tend to apply similar symbolic conciseness to programming. But the
\enquote{philosophy} behind combinator interfaces is not (only) to replace function names and method
calls by (often cryptic) symbols; rather, they try to factor out certain kinds of patterns and
provide a succinct way of writing them. Such patterns are those for which traditional inline
notations exist. Most prominently, this is the mathematical notation for calculations containing
things like vectors or matrices; however, there are other patterns, such as pipe-and-filter (known
from \abbrev{UNIX} shells), or grammar notations, like regular expressions or the Backus-Naur-Form.

\begin{lstlisting}[style=floating, label=lst:lenses, language={[Modern]Haskell},
  caption={Usage example of lenses, a popular pattern for purely functional structural
    traversal/modification in Haskell, which utilizes an extreme variety of combinators. The
    \lstinline|(.)| operator is nothing more than regular function composition; using it, various lenses
    are combined to \enquote{focus} on the part of interest, on which then the function \lstinline|succ|
    is applied.\protect\footnotemark}]
  -- Increments all of the major versions of an array of JSON objects.
  someString ^.. _JSON        -- a parser/printer prism
  . _Array       -- another prism
  . traverse     -- a traversal (using Data.Traversable on Aeson's Vector)
  . _Object      -- yet another prism
  . ix "version" -- a traversal across a "map-like thing"
  . _1           -- a lens into a tuple (major, minor, patch)
  %~ succ        -- apply a function to our deeply focused lens
\end{lstlisting}

Starting from these patterns, and equipped with a frictionless treatment of operators as functions,
there has since long been established a common \enquote{basic set} of operators in the Haskell
standard library~-- not only for \dsls{}. These have layed the basic and inspiration for many other
libraries using combinators. %
\footnotetext{Example taken from
  \protect\url{https://www.fpcomplete.com/user/tel/a-little-lens-starter-tutorial} (visited on
  2015-06-04).}%
A very good example of extremely rich Haskell combinators is the Lens
package,\footnote{\protect\url{http://hackage.haskell.org/package/lens} (visited on 2015-06-04). A
  very good point of overview is
  \protect\href{http://hackage.haskell.org/package/lens-4.11/docs/Control-Lens-Lens.html}{\textcolor{black}{\lstinline[columns=fixed]|Control.Lens.Lens|}},
  which already contains exotically-looking combinators like
  \lstinline[keywordstyle=\color{black}]|(<<\%@~)|.} which provides combinators for generalized
getters, setters and modifiers for nested data structures. In \autoref{lst:lenses}, an example of
its usage is shown: the code there describes a way to modify the major version number of a given
\abbrev{JSON} object.

In Scala, the language often does not allow (or at least makes very impractical) such an extreme use
of combinators as in Haskell (this is due to the restrictions of how precedence of operators is
determined (\cf \autopageref{operators}), and to the lack of automatic currying). Nevertheless,
there are places and applications where they turn out to be useful and are commonly employed. One
such application is parser combinators. These have been popularized first by Haskell's efficiently
implemented Parsec
library~\cite{leijen2001:parsec}\footnote{\protect\url{http://hackage.haskell.org/package/parsec}
  (visited on 2015-06-04).}, but have since been ported to various other languages~-- including
Scala, where they are available as a separate module
|scala-parser-combinators|\footnote{\protect\url{http://www.scala-lang.org/files/archive/api/2.11.x/scala-parser-combinators/\#package}
  (visited on 2015-06-05).}.

In \autoref{lst:parser}, an example of its usage is shown for a simple grammar for
\abbrev{LISP}-style S-expressions (which look like |(foo (cons 0 '(1 2 3)) (bar ()))|). Like in a
\abbrev{BNF} grammar, nonterminals are given their own names; they are implemented as individual
|Parser| objects mutually calling each other. These nonterminal-parsers are then combined with each
other and primitive parsers for terminals, using a range of operators, many of which have mnemonic
names; for example, the expression
\begin{lstlisting}
  whiteSpace.? ~> (readMacro | cons | atom)
\end{lstlisting}
means \enquote{parse optional whitespace, discard it, then apply whichever of
  \lstinline[style=inline]|readMacro|, \lstinline[style=inline]|cons| and
  \lstinline[style=inline]|atom| succeeds, and return its result}~-- this uses a mixture of
\abbrev{BNF} alternation with \lstinline[style=inline]{|}, the regex \enquote{optional} modifier
|?|, and Haskell's left-discarding applicative combinator, which originally is |*>|, but has been
changed to |~>| in Scala, where it is also congruent with the sequencing operator |~|. 

\begin{lstlisting}[style=floating, label=lst:parser,
  caption={Parser for a simple S-expression syntax for \abbrev{LISP}, including quotes. The
    \lstinline|\^\^| combinator is essentially (\lstinline|f|)\lstinline|map|~-- it lifts a function
    to transform the result of a successful parser.
    \hfill\github{dsl-examples/blob/master/src/main/scala/dsl_examples/SExpParser.scala}}]
  object SExpParser extends RegexParsers {
    override val skipWhitespace = false

    def sexpr: Parser[SExpr] = whiteSpace.? ~> (readMacro | cons | atom)

    def readMacro: Parser[SExpr] = 
      (readMacroIdentifier ~ sexpr) <~ whiteSpace.? ^^ {
        case s~e => ReadMacro(s, e)
      }
    def cons: Parser[SExpr] = 
      parenthesized(rep(sexpr) ^^ ConsList) <~ whiteSpace.?
    def atom: Parser[SExpr] = (identifier ^^ Atom) <~ whiteSpace.?

    def parenthesized[T](p: Parser[T]) = 
      "(" ~ whiteSpace.? ~> p <~ whiteSpace.? ~ ")"
    def identifier: Parser[String] = "[^()' ]+".r
    def readMacroIdentifier: Parser[String] = Quote.macroCharacter
  }
\end{lstlisting}

In contrast to traditional \abbrev{BNF}, it is also possible to define \enquote{parametrized
  symbols} in the form of functions (such as |parenthesized|). This is an important point. Contrary
to specialized, external languages (in this case, parser generators such as \abbrev{ANTLR}),
embedded combinator libraries have the advantage of and should be designed to being able to interact
with the surrounding language \emph{inline}. This is not only done by using language constructs to
form new parsers (parser-transformers like |parenthesized|), but also by being able to embed Scala
in parsers with function combinators like |^^|. In that way, parser combinators are an excellent
example of \abbrev{SICP}'s principles of metalinguistic abstraction: they neatly combine means of
abstraction, combination, and factoring out patterns, by seamlessly interacting with the
functionalities of the host language.

When designing a combinator interface in Scala, not only the conciseness of expressions should be
respected, but also the conventions and habits of the users of the language. For example, while in
Haskell the |atom| parser might be expressed as
\begin{lstlisting}[language={[Modern]Haskell}]
  Atom <$> identifier <* optional whitespace
\end{lstlisting}
this is rather impractical in Scala, since the higher-order application would not work in expected
ways. Instead, using |^^| with an anonymous function is the preferred way to map over a parser. On
the other hand, the usage of methods for changing parsers, like in |whitespace.?|, is not easily
possible in Haskell, but quite effortless in Scala, and forms thus a good way of modifying
combinators. The same would also work in |cons| (line~10), where |rep(sexpr)| could also be written
as |sexpr.+|; but the tradeoff between symbolic names and static functions in a case like this is
more one of personal style. In general, a good solution would probably take into account both
backgrounds and provide method-style as well as combinator-style parts of the interface, allowing
the user to choose a suitable mixture.

% $ prevent emacs from not hightlighting the rest...

%--------------------------------------------------------------------------------
\section{Objects and Modules, Mixins and Cake}
\label{sec:modules}

This subsection's purpose is to introduce a few practical notions of modularization and organization
of code parts. They are the most important concepts for this used in Scala, and shown here in order
to demonstrate how a good \enquote{user experience} for a library can be constructed; besides, also
some specific tricks using traits are shown, which are examles of how to provide modularized and
customizable behaviour for classes.

\newthought{As it was stated in the introduction}, Scala has a highly sophisticated module
concept~-- in this respect, it allows much more flexibility than many other languages. Still, the
underlying building block of this system is just one thing: the |object| construct. An |object| is
basically just a syntactic sugar for declaring a singleton instance of some class at top level, with
the convenience of having the same syntax as a class. However, |object| declarations have really
more power than plain instances; most importantly
\begin{itemize}
\item every |object| forms a \emph{module}~-- a separate namespace, that can be |import|ed
  otherwhere, but can also contain its own state; furthermore,
\item |object|s can be used as \emph{companion objects} for classes or traits, in which case their
  members generalize the notion of static members known from other languages; and finally,
\item \emph{package objects} can be used to provide free functions and objects at the top level of a
  package, which are automatically available at import.
\end{itemize}
In the following, the term \enquote{object} will mostly denote such singletons declared by |object|,
not just arbitrary instances.

In the first, simple variant, objects can be used to modularize small pieces of code, without having
to use the package system. This is useful if one wants to provide helper methods or implicits to be
used for a specific \dsl{}, and provide them in an extra namespace. It can also be used to bundle
features in a kind of \enquote{overlapping} module~-- Scala has no explicit notion of exporting from
a package (everything that is not marked as package private, is considered public), so sometimes it
is practical to collect a range of members in an object and relay their definitions to other
namespaces, in that way providing a combined reexport.

A further common pattern for this use case is to separate code using traits. Since traits, unlike
interfaces, are not necessarily something purely abstract, but can be fully concrete, they are often
used to just split parts of an implementation (logically belonging together) into multiple internal
parts, and mix them together into one final exported object~-- which in the most extreme case does
not even need to contain any other code. A perfect example of this is the organization of the Scalaz
library.\footnote{\protect\githubcommit{https://github.com/scalaz/scalaz}{/tree/24cf7f3dcf9baa9ea438269eea6b24fd9476a544}
  (visited on 2015-05-16).} Scalaz is a very rich package, but tries to allow very granular import
behaviour; every part of its functionality can be imported separately (\eg, %
|import scalaz.Id._|). However, if one wants to bring into scope the full range of features, one
only needs to import two things:
\begin{lstlisting}
  import scalaz._
  import Scalaz._
\end{lstlisting}
The first one is the package itself, which contains mostly type synonyms and some very basic
instances; the second line, importing |Scalaz._|, imports the members of the object listed in
\autoref{lst:scalaz}: this object has no own members, but accumulates the members of seven other
traits providing all kinds of general functionality contained in the library, from syntactic helpers
to type class instances for standard library types.

\begin{lstlisting}[style=floating, label=lst:scalaz, 
  caption={The definition of the \texttt{Scalaz} object (defined in file
    \protect\href{https://github.com/scalaz/scalaz/blob/f7256ad19bf2bf92756d2c3168ed643aad07d356/core/src/main/scala/scalaz/Scalaz.scala}{\nolinkurl{scalaz/core/src/main/scala/scalaz/Scalaz.scala}}),
    collecting functionality from a range of implementation traits.}]
  object Scalaz
    extends StateFunctions        
    with syntax.ToTypeClassOps    
    with syntax.ToDataOps         
    with std.AllInstances         
    with std.AllFunctions         
    with syntax.std.ToAllStdOps   
    with IdInstances              
\end{lstlisting}

\newthought{If the set of helpers is specific} to a certain class or trait, it can be put in the
\emph{companion object} of that class. Such an object must have the same name and be defined in the
same file as the companion class. Defining members in the companion has the advantage that they then
have access to the private members of the class; in that way, companions incorporate everything that
would be |static| in Java. Additionally, the companion object is taken into account for implicit
scope resolution, so it makes sense to define known type class instances there.

\begin{lstlisting}[style=floating, label=lst:church, 
  caption={An implementation for polymorphic Church encoded lists. The \texttt{apply} and 
    \texttt{unapply} methods are the only way to construct and deconstruct them; from a 
    more theoretical point of view, they form the isomorphism between \texttt{ChurchList[T]}
    and the builtin \texttt{List[T]}. \texttt{Empty} is defined as a dedicated object to allow
    sharing and to make pattern matching look more natural.
    \hfill\github{dsl-examples/blob/master/src/main/scala/dsl_examples/ChurchList.scala}}]
  sealed trait ChurchList[+T] {
    def fold[K]: K => (T => K => K) => K
  }

  object ChurchList {
    def apply[T](l: T*): ChurchList[T] = l.toList match {
      case Nil => Empty
      case x::xs => new ChurchList[T] {
        val rest = ChurchList[T](xs: _*)
        def fold[K]: K => (T => K => K) => K =
          nil => plus => plus(x)(rest.fold(nil)(plus))
      }
    }
    
    def unapplySeq[T](l: ChurchList[T]): Option[Seq[T]] = 
      Some(l.fold[List[T]](Nil)(x => xs => x::xs))
  }

  object Empty extends ChurchList[Nothing] {
    def fold[K] = nil => _ => nil
  }
\end{lstlisting}

A common pattern for implementing factories in Scala is to use the companion object for construction
and destruction; that is, to define |apply| and |unapply| there, and hide the actual (maybe more
complicated) implementation from the user. This has the additional benefit that the implementation
can later be swapped without interface changes. An example for such an implementation is shown in
\autoref{lst:church}; this uses a rather unusable underlying implementation of lists, which
nevertheless can be used in a normal way via the companion object:
\begin{lstlisting}[style=break-lines]
  scala> ChurchList(1, 2) match { case ChurchList(x, xs) => s"Full: Cons($x, $xs)"; case Empty => "Empty!" }
  res0: String = Full: Cons(1, 2)

  scala> ChurchList() match { case ChurchList(x, xs) => s"Full: Cons($x, $xs)"; case Empty => "Empty!" }
  res1: String = Empty!
\end{lstlisting}
The definition of its behaviour is part of the sealed trait |ChurchList|, but the contents can only
be accessed through the static methods from the companion (|sealed| prevents implementing the trait
from elsewhere than the defining file, thus preventing \enquote{direct} creation without the factory
methods).

The third form of modularization throught objects are package objects. They are usually defined in a
separate file in the package's directory, called \texttt{package.scala}, and differ in their
definition only by the additional keyword |package|; as an example, consider again an according
(greatly simplified) definition in Scalaz%
\footnote{%
  \protect\href{https://github.com/scalaz/scalaz/blob/0bebf537f1b37588143d303732d6a8dfbe1ef061/core/src/main/scala/scalaz/package.scala}{\nolinkurl{https://github.com/scalaz/scalaz/blob/series/7.2.x/core/src/main/scala/scalaz/package.scala}},
  simplified (visited on 2015-11-26).}:
\begin{lstlisting}[mathescape]
  package object scalaz {
    import Id._
    implicit val idInstance: Traverse[Id] with ... with Cozip[Id] = Id.id
    type Tagged[T] = { type Tag = T }
    type @@[T, Tag] = T with Tagged[Tag] 
    $\vdots$
  }
\end{lstlisting}
The contents of package objects are always imported when the respective package name is imported,
and provide top-level \enquote{free definitions}. Usual contents of them are type synonyms, which
are desired to be avaliable globally (since |type| declarations are only allowed within classes or
objects), and widely used implicits for the package. Often, they also contain various small helpers
such as wrappers or functions intended to be used frequently, which are not considered complex or
important enough to deserve their own files or modules.

\newthought{Besides these patterns using objects}, there are also a variety of useful things that
can be done with traits, other than using them simply as interfaces. One of the common usages has
already been mentioned, namely, the splitting of code among several implementation traits. A variant
of this is providing helper methods and implicits to be used with a class or type class in an own
trait, and mix that into the class itself or the package object. That way, all utility functions
using some type are always imported together with it (these are often found in combination with type
classes' |Ops|-wrappers mentioned on \autopageref{ops}).

Other patterns using traits concern methods of modularizing optional behaviour of classes. This is
often wished when there is one \enquote{main} class, providing a basic behaviour or functionality by
inheritance, and multiple additional features for it, which are however optional. The conventional
way to solve such a scenario would be an \abbrev{OOP} pattern like Decorator; however, all these are
subsumed under traits.

There are two ways to conveniently provide \enquote{decoration traits}, both of which rely on a
stronger coupling between the trait and the basic class that mere implementation traits. One variant
is to give a trait a \emph{self-type annotation}:
\begin{lstlisting}
  class A {
    def modifyString(s: String) = s * 2
  }

  trait T { this: A => 
    def modifyAllStrings(ss: Seq[String]) = ss map (modifyString(_))
  }
\end{lstlisting}
Such an annotation consists of the statement |this: A =>| at the beginning of the body, which means
that \enquote{the type of \texttt{this} must be \texttt{A} (or a subtype thereof)}. In this way, |T|
can refer to |A|'s public methods as its own, since it \enquote{knows} that the resulting actual
type will always be an |A|:
\begin{lstlisting}
  scala> (new A with T).modifyAllStrings(List("sdf", "ab", ""))
  res0: Seq[String] = List(sdfsdf, abab, "")
\end{lstlisting}
This kind of mixin is useful when providing additional features to a class, such as testing methods,
or special optional syntax. Combining this with leaving members abstract in the \enquote{main}
class, this pattern is frequently used for dependency injection and known as \emph{cake pattern},
since it constructs a final implementation out of a variety of available layers (such as logging,
testing, persistence, \ldots).

Traits with self-type annotations however cannot influence the basic behaviour that was
\enquote{already there}. For that purpose, a more sophisticated kind of mixins can be employed: this
second variant is a bit more interesting from an object-oriented perspective. It exploits the fact
that Scala trait inheritance uses \emph{dynamic super-type resolution}. That is, when mixing traits
from an inheritance hierarchy, behaviour can not only be stacked, but overridden in a well-defined
manner. Consider the following trait, continuing the example from above:
\begin{lstlisting}
  trait Logging extends A {
    override def modifyString(s: String) = { 
      println(s"[Log] Old string was: $s")
      super.modifyString(s)
    }
  }
\end{lstlisting}%$
If we now say
\begin{lstlisting}
  scala> (new A with Logging).modifyString("sdf")
  [Log] Old string was: sdf
  res0: String = sdfsdf
\end{lstlisting}
we see that the old behaviour has in fact be extended with the trait. This can be used to combine
multiple available behaviours in a \enquote{vertical} way, instead of only \enquote{horizontally},
as with the first variant~-- the different implementations are added above each other. The dynamic
mixin inheritance provided by Scala is a practical and powerful alternative to other styles of
multiple inheritance. It differs from most other approaches by the fact that calls to |super| are
not resolved at the place where they appear~-- rather, in the final object, all mixed-in traits are
\emph{linearized} according to a specified order, which is then used to resolve the call
\cite[][Chapter~12.6]{odersky2008:programming}.

\newthought{The relation of all these patterns to \dsls} is the fact that they can be used to
provide a much more natural interface for \dsls{} which are thought to be implemented as classes. A
common way to design a library for this is to have an abstract class containing the basic
functionality, and intend it to be extended on a case-by-case basis. So, for one specific instance
of the usage of the functionality, the user would create a new subclass or even subobject, and
implement the concrete behaviour in the body of the class (either in the primary constructor, or by
implementing abstract members). Then, additional functionality can be provided via mixins; this can
range from mixing in the actual \dsl{} syntax to just adding some debug functionality.

A nice example of this kind of interface is shown by the Akka
library,\footnote{\protect\url{http://doc.akka.io/docs/akka/2.3.11/scala.html} (visited on
  2015-06-23).} which superseded Scala's earlier actor implementation in the standard library. To
define an Akka actor, one usually creates a subclass of |Actor|, and implements the |receive| method
there:
\begin{lstlisting}
  class PingPong extends Actor {
    def receive = {
      case Ping => sender() ! Pong
    }
  }

\end{lstlisting}

The basic thing an actor does is to receive input messages, and process them by doing something to
its internal state or sending messages to other actors. However, there's much more features
available in Akka. While many of these can be configured via properties or members, there's a range
of mixed-in behaviour for special |Actor|s:
\begin{itemize}
\item |Actor with ActorLogging| turns on the default logging mechanism for actor
  systems, which is useful for debugging.
\item |Actor with Stash| extends the default message box behaviour such that messages can be
  \enquote{put aside} currently (\enquote{stashed}), and treated again later.
\item |Actor with RequiresMessageQueue[BoundedMessageQueueSemantics]| (or with some other semantics)
  changes the way the message box of the actor is handled; this is a generalization of the behaviour
  found in |Stash| (which is a subtrait of |RequiresMessageQueue[DequeBasedMessageQueueSemantics]|).
\item |Actor with FSM[State, Data]| mixes in another \dsl{}, providing an additional wrapper over
  |receive| and allowing actors to be defined as Finite State Machines, reacting to input messages
  and updating their internal state according to the current state.
\end{itemize}
The last point is an especially interesting one, since this way of providing \dsl{} syntax is very
similar to the design used in the practical part of this work: a completely new layer of syntax is
mixed in, with new \enquote{statements} to be used in the primary constructor, which then internally
call the underlying implementation (in the case of actors, |receive|).

% How this looks like in contrast to
% a regular actor is shown in \autoref{lst:fsm}.

% \begin{lstlisting}[style=floating, label=lst:fsm,
%   caption={FSM\protect\footnotemark}]
%     class PingPont extends Actor with FSM[State, Data] {
%       startWith(Idle, Record(pings = 0, pongs = 0))
     
%       when(Pinging) {
%         case Event(Ping, record) =>
%           goto(Ponging) using record.copy(pings = record.pings + 1)
%       }
     
%       when(Ponging, stateTimeout = 1 second) {
%         case Event(Pong, record) =>
%           goto(Pinging) using record.copy(pongs = record.pongs + 1)
%       }

%       initialize()
%     }
% \end{lstlisting}

% How this looks like in contrast to
% a regular actor is shown in \autoref{lst:fsm}.

% \begin{lstlisting}[style=floating, label=lst:fsm,
%   caption={FSM\protect\footnotemark}]
%     class Buncher extends Actor with FSM[State, Data] {
%       startWith(Idle, Uninitialized)
     
%       when(Idle) {
%         case Event(SetTarget(ref), Uninitialized) =>
%           stay using Todo(ref, Vector.empty)
%       }
     
%       when(Active, stateTimeout = 1 second) {
%         case Event(Flush | StateTimeout, t: Todo) =>
%           goto(Idle) using t.copy(queue = Vector.empty)
%       }
     
%       whenUnhandled {
%         case Event(Queue(obj), t @ Todo(_, v)) =>
%           goto(Active) using t.copy(queue = v :+ obj)
%       }
     
%       initialize()
%     }
% \end{lstlisting}
% \footnotetext{Incomplete example taken from
%   \protect\url{http://doc.akka.io/docs/akka/2.3.11/scala/fsm.html} (visited on 2015-06-24); this
%   code is simplified and slightly modified, and the \texttt{State} and \texttt{Data} types are left
%   out.}

From these examples we can see that providing an interface using mixins in this style does not only
allow a user to finely choose what behaviour or variation thereof is wanted, but also give them
choice over what style of \dsl{} is wanted or needed. A more extreme example of this style is
ScalaTest~-- this library considers itself a \enquote{test toolkit}, and provides a wide selection
of classes and traits to be combined. Its philosophy is to \enquote{source out} every aspect of
testing into its own module (for example, underlying test platform, kind of testing, specification
syntax, documentation creation, and so on), and allow the tester to combine all these freely into
what fits them:\footnote{Example taken from
  \protect\url{http://www.scalatest.org/user_guide/defining_base_classes} (visited on 2015-06-24).}
\begin{lstlisting}
  abstract class UnitSpec extends FlatSpec with Matchers with
    OptionValues with Inside with Inspectors
\end{lstlisting}
This class does not need a body, all its functionality and style choices are mixed in; it will just
be used as a base class for all unit test modules in this project. (A larger example of ScalaTest
has already been shown in \autoref{lst:scalatest}.)

% Unified import behaviour (esp. with traits), \enquote{*Ops}-patterns and code organization with
% traits, companion objects, default constructors. Cake pattern, polymorphic embedding.
% 
% Implementation providers, behaviour stacking \& merging. Self-types and path-dependent types.


% %--------------------------------------------------------------------------------
% \section{Faking the Imperative}
% \label{sec:imperative}

% In this final pattern section, a few of the features described above will be combined into a \dsl{}
% for finite state machines, to illustrate their working together, and to show how Scala syntax allows
% to provide a library interface looking like actual syntactic constructions just by exploiting the
% various possibilities provided. This example will combine curried methods, call-by-name, partial
% functions and blocks, to fake the look of an actual imperative language. The inspiration for this
% has been taken from the |FSM| trait from the Akka actor
% library\footnote{\protect\url{http://doc.akka.io/docs/akka/2.3.11/scala/fsm.html} (visited on
% 2015-06-23).}.


%%% Local Variables: 
%%% TeX-master: "document"
%%% End:

\chapter{Action Systems and Testing: Definition \& Background}
\label{sec:action_systems}

\lstset{deletestring=[b]'}

In the remaining part of this bachelor thesis, the practical work done is described. It consists of
a library to program a restricted variant of so-called Action Systems, a formalism similar to state
machines or Labelled Transition Systems. The resulting library provides basic functionality to
execute such Action Systems using various methods, and, most importantly, a \dsl{} to write them
concisely in Scala. As many parts as possible are parametrized and thus left open for more specific
use cases or future enhancements. The library has the current working title \actium, which is the
latinized name of the small town of \kern1pt\greek{Ἄκτιον}\kern-4pt in ancient Greek, place of the
Battle of Actium~-- but mostly, a pun on the word \enquote{action}.

The description of the implementation consists of three parts. First, in this section, an
introductory overview of the principles and the usage of Actions Systems and their application in
model-based testing is given, to establish a background for the applicability of the library and
terms commonly used. Following that, details of the implementation are provided, with a focus on the
\dsl{} interface and how its form could be achieved programatically, exploiting the language
features of Scala. In the couse of this, also an overview of the functionalities of \actium{} is
given. Lastly, the finished state of the project is summarized, and the lessons learned are
pronounced, which includes notes about the development process, about language features and their
usage, and about other directions that could have been followed or were not tried out. This last
section also reflects on the state of the library and its limitations, and possible improvements.

\newthought{Action Systems were introduced} to formalize the collaboration of processes in
distributed systems~\cite{back1983:decentralization}, and proposed as an approach dually to other,
more process-focused approaches such as Communicating Sequential Processes
(\abbrev{CSP})~\cite{hoare1978:communicating}. Somewhat informally, an Action System in that
original formulation consists of \emph{processes}, each having associated some variables with their
initializations, and (named) \emph{actions}, each consisting of a collection of participating
processes, a \emph{guard} (predicate), and a statement. An action (or rather, its statement) can be
executed if all of its processes are not currently participating in another action, and the guard is
satisfied~-- in that case, the action is said to be \emph{enabled}~\cite{back1988:distributed}. The
order in which the actions are executed, or how they are chosen if there are multiple enabled
actions, is left unspecified; the system can be executed using different strategies, synchronuously
or concurrently.

In the programming work done for this bachelor thesis, a restricted variant of the above formulation
is used, which is applied in the context of \emph{model-based mutation
  testing}~\cite{aichernig2014:killing}. Mutation testing~\cite{hamlet1977:testing} is a technique
to automatically generate a suite of useful test cases for a given system, by systematically
injecting small syntactic faults (called \emph{mutants}). A test suite is then generated from the
mutants by incrementally adding test cases which can distinguish a mutant from the original
implementation. However, to ensure that mutants actually do anything \enquote{useful} (and not
change the behaviour in a trivial way, or only change dead code), they must first be analyzed in
some way, which traditionally involved manual inspection. Mutants with such \enquote{useless}
behaviour are called \emph{equivalent} (since they are indistinguishable by test cases).

Model-based mutation testing combines this approach with model-based testing: instead of directly
operating on the system under test (\abbrev{SUT}), a model of it, called a \emph{test model}, is
used. This model, commonly described in a symbolic, abstract fashion, is assumed to be correct and
usually simpler than the \abbrev{SUT}, since only the properties of interest for testing are
modelled. It has the advantages that one has more control of the test generation process; for one,
it allows to treat the \abbrev{SUT} through the model in a purely abstract, black-box manner, which
enables non-software systems to be tested as well. Equally important, checking mutants becomes
easier, because the mutation process can be resticted to a more controlled set of syntactic mutation
operators, and equivalent mutants can be excluded by means of constraint solving. Furthermore, by
using such a model, nondeterminism, which can occur both in the \abbrev{SUT} or because of the
abstractions arising from the modelling process, becomes easier to deal with. A rather complete
overview of this methodology is given in~\cite{jobstl2014:model-based}.

\newthought{The point of Action Systems} in this environent now is to serve as test models for
mutation testing, especially of non-deterministic systems. Their quite simple, formal nature allows
to concisely write a model of the \abbrev{SUT}, and then transform this system into a symbolic
representation. This representation can be relatively easily mutated, and the resulting
representations can be passed to a constraint or \abbrev{SMT} solver which analyzes the system and
the mutants. That such an approach is not only possible, but can also be done efficiently, is shown
in~\cite{aichernig2015:model}.\label{constraint_solving}

\begin{lstlisting}[style=floating, label=lst:move_action, language={},
  caption={Description of an action \lstinline|move| in a concrete syntax. The action has two
    parameters, a guard, and a statement consisting of two sub-statements. \lstinline|mode|,
    \lstinline|engine|, \lstinline|pos_x| and \lstinline|pos_y| are (global) state variables.}]
  move(x:MyNat, y:MyNat) if mode == Air && engine == 1 then
  {
    pos_x := pos_x + x;
    pos_y := pos_y + y;
  };
\end{lstlisting}

As said above, for this purpose, it is practical to use a more narrowed down form of Action Systems
than those defined initially. It turns out that it is possible to write a system in such a way that
it looks similar to a state machine, but keeps the meaning of it as an Action System: we define a
set of state variables with their initial values, and a set of labelled actions. Each of the actions
can, in addition to the state variables, have parameters. It can also have a guard, involving its
parameters and state variables, and a statement, assigning new values to state variables depending
on their old values and the parameters. How such an action definition can look like is shown in
\autoref{lst:move_action}, using a syntax very similar to the one defined in~\cite[p.~16]{tappler2015:symbolic}.

In this formulation, the definition of processes is left implicitly; such systems are equivalent to
Action Systems in which only one process exists, which also contains all the state variables. The
named similarity to a state machine is not an unuseful one, but should be taken with care~-- the
labels of actions should \emph{not} be mistaken for states. Rather, the similarity is of a different
nature: the system in such a form is more like a \abbrev{FSM} with data path, but only one state. In
such a setting, the actions are just labelled transitions from the one state to itself; they only
differ in their conditions and update operations. The relevant information for execution of the
system is then just the sequence of transition names, not of states. The data-path state is encoded
in the state variables; such a system could in principle be transformed into a real state machine,
but would soon suffer from state-space explosion (or turns out to be impossible, since the states
covered by an Action System are not even necessarily finite). 

A more accurate, though also more formal description of Action Systems (in the full original variant
as well as the simplified one used here) can be given via Labelled Transition Systems
~\cites[p.~11]{tappler2015:symbolic}[p.~21]{jobstl2014:model-based}; this formalism is also used to express formally the conformance
relations between different models (test models, or mutated models of one original model), and is
used practically as an intermediate representation in various conformance checking algorithms.

The Action Systems which can be described by \actium{} are exactly of this nature. The basic style
of the syntax is inspired by the one mentioned above; however, some keywords are taken from the
Gherking specification
language\footnote{\protect\url{https://github.com/cucumber/cucumber/wiki/Gherkin} (visited on
  2015-06-06)}, and some things had of course to be changed due to Scala's native syntax. (The idea
of using Gherkin-like syntax for a Scala \dsl{} is not new: it exists already at least in the
|FeatureSpec| test style of
ScalaTest.\footnote{\protect\url{http://doc.scalatest.org/2.2.4/\#org.scalatest.FeatureSpec}
  (visited on 2015-06-24)})%


%%% Local Variables: 
%%% TeX-master: "document"
%%% End:

\chapter[Actium~-- a DSL for Action Systems]{{\texttt{Actium}}~-- a DSL for Action Systems}
\label{sec:implementation}

%%%%%%%%%%%%%%%%%%%%%%%%%%%%%%%%%%%%%%%%%%%%%%%%%%%%%%%%%%%%%%%%%%%%%%%%%%%%%%%%%%%%%%%%%%%%%%%%%%%%%
% \section[Implementation of Actium: The Details]{Implementation of
%   \kern-0.5pt{\small\texttt{ACTIUM}}: The Details}
% \label{sec:implementation}

This section consists of four parts, which try to introduce \actium{} to the reader from the lowest
to the highest level. The first subsection describes the non-\dsl{} interface and the basic pattern
of how an Action System is defined as a conventional class. The second one explains the inner
workings of the library, and how systems are executed internally. Finally, the remaining two
sections show how the language features of Scala were used to built on top of that the desired
domain-specific interface, and describe some interesting implementation details, as well as some
additional functionalities.

%--------------------------------------------------------------------------------
\section{Building a System}
\label{sec:building}

\begin{figure}[b]
  \centering
  \scriptsize\sffamily
  \begin{tikzpicture}
    \usetikzlibrary{positioning}
    \tikzumlset{font=\scriptsize\sffamily}
      \umlbasicstate[x=0,y=0,name=Destroyed]{Destroyed}
      \umlbasicstate[x=4,y=0,name=AirAndOn]{AirAndOn}
      \umlbasicstate[x=4,y=-2.5,name=GroundAndOn]{GroundAndOn}
      \umlbasicstate[x=0,y=-2.5,name=GroundAndOff]{GroundAndOff}
      \umlstateinitial[x=0,y=-4,name=Start]
      \draw [->] (Start.north) -- (GroundAndOff.south);
      \draw [->, transform canvas={yshift=+2mm}] (GroundAndOff.east) -- (GroundAndOn.west)
          node[above, midway]{PowerOn};
      \draw [->, transform canvas={yshift=-2mm}] (GroundAndOn.west) -- (GroundAndOff.east)
          node[below, midway]{PowerOff};
      \draw [->, transform canvas={xshift=+2mm}] (GroundAndOn.north) -- (AirAndOn.south)
          node[right, midway]{Start};
      \draw [->, transform canvas={xshift=-2mm}] (AirAndOn.south) -- (GroundAndOn.north)
          node[left, midway]{Land};
      \draw [->] (AirAndOn.west) -- +(west:1.9cm) node[above, midway]{Destroy};
      \draw [rounded corners] ([yshift=-2mm] AirAndOn.east) 
          -| +(0.6, 0.4) -- ([yshift=2mm] AirAndOn.east) [->];
      \node [right=0.6cm of AirAndOn.east] {Move(x, y)};
  \end{tikzpicture}
  \caption{State diagram of the \lstinline|SimpleRocket| Action System. What happens internally, is
    not specified, and no data path operations are mentioned; the important information is that only
    sequences of actions (transitions) from this graph are allowed. When the \lstinline|Destroyed|
    state is reached, no more actions are possible and the system must
    stop.\label{fig:simple_rocket}}
\end{figure}

To explain the features of the implementation on an actual example, a very simple Action System for
an imaginary rocket will be used. The definitions and possibilities of \actium{} are shown
step-be-step using this system. In \autoref{fig:simple_rocket}, a state diagram of it can be
seen. Basically, the rocket can be turned on, start, fly, land, and be turned off; when flying, it
can also be destroyed. Note that implementing this as an Action System does not necessarily have to
follow the shown partition into states; how internal state is handled is completely up to what form
of modelling and refinement is used, and state variables can be chosen as needed. The important
information of that diagram is to show the allowed sequences of Actions by the transitions of the
graph.

\newthought{On the most basic level} of defining an Action System, the |ActionSystem| trait and
Scala's standard syntax are already enough to achive a certain level of expressiveness superior to
many other languages. A concrete implementation will typically be an |object|, if all aspects of the
behaviour are already fixed, or an abstract class leaving some parts, so that they can later be
mixed in on a per-instance basis (especially \enquote{choice traits}, as |RandomChoice| in the
example below~-- these will be explained in detail in the next subsection). Basically, an
|ActionSystem| instance can be created with \eg
\begin{lstlisting}
  val r1 = new SimpleRocket with RandomChoice
\end{lstlisting}
On this |r1|, we could now, in principle, call the methods |addAction| and |initialize| to add all
necessary behaviour; however, the idea is to specify everything possible in the primary constructor
of the class or object, since in most cases, the behaviour of an instance is known beforehand
(still, the possibility to construct an |ActionSystem| instance programatically, \eg, from an
external specification, is left open). Scala makes this construction in the constructor especially
easy by treating everything inside a class body as part of the primary constructor; so, if our
example class looks like
\begin{lstlisting}[mathescape]
  abstract class SimpleRocket extends ActionSystem with GherkinSyntax {
    $\vdots$
    initialize(something)
    addAction(Action('foo, True, stmts))
  }
\end{lstlisting}
at instance creation the initial environment will be set to |something|, and one action |'foo| be
added~-- there is no need to declare an extra constructor method. In fact, in many cases, an
implementation will do without any method declarations whatsoever, by just putting everything in the
primary constructor (that is, the class body).

Now, lets consider an actually useful instance for the above example. The following statements are
meant to be put directly into the above |SimpleRocket| class. First, we should define how we are
going to represent the state of the system; in this case, we encode everything as integers:
\begin{lstlisting}
  type State = Int
\end{lstlisting}
Then, in order to actually write something useful, it is convenient to first give some
initializations to the system, declaring shorthands for states and symbolic variables:
\begin{lstlisting}
  val F = 0
  val T = 1
  val Air = 0
  val Ground = 1
  val Destroyed = 2

  val engine = Variable('engine)
  val mode = Variable('mode)
  val pos_x = Variable('pos_x)
  val pos_y = Variable('pos_y)
\end{lstlisting}
The |Variable| constructor is provided by the library. Some of these definitions are not a strictly
necessary part of the system declaration, but will improve readability later. To get a runnable
system, we also need to initialize it:
\begin{lstlisting}
  initialize(
    Assignment(engine, Constant(F)),
    Assignment(mode, Constant(Ground)),
    Assignment(pos_x, Constant(0)),
    Assignment(pos_y, Constant(0))
  )
\end{lstlisting}
The only task of |initialize| is adding a list of |Assignment|s to the environment in order to
provide the state variables and their initial values. Again, |Assignment| and |Constant| are part of
the symbolic representation provided by the library. Here, only constants are assigned, but in
principle, any expression could stand on the right hand side of an assignment. The assignments are
evaluated individually, one after the other, at the time of adding (that is, at class construction),
and each evaluation is done using the current environment~-- this allows assigning state variables
based on previous values, like
\begin{lstlisting}
  Assignment(pos_x, Constant(0)),
  Assignment(pos_y, pos_x)
\end{lstlisting}
where at the definition of |pos_y|, |pos_x| is already existing in the environment.

\newthought{If all helpers are defined}, and everything is initialized, the only thing remaining to
do is adding the actual actions. For this purpose, there exists the method %
|addAction(action: Action): Unit|, which updates the internal table of actions. The exact definition
of |Action| will be shown below (\autopageref{lst:action}) and contains the name, the condition, the
statement, and optionally the parameters of an action. To actually construct an action, it is
necessary to put in the \abbrev{AST}s of all these parameters:
\begin{lstlisting}
  addAction(Action('PowerOn,
    And(Predicate2("==", engine, Constant(F)), 
        Predicate2("!=", mode, Constant(Destroyed))),
    Seq(Assignment(engine, Constant(T))))
  )
\end{lstlisting}
which corresponds to
\begin{lstlisting}[language={}]
  powerOn() if engine == 0 && !(mode == Destroyed) then
  {
    engine := 1;
  };
\end{lstlisting}
in the external syntax~\cite{tappler2015:symbolic}. For parametrized actions like the one shown in
\autoref{lst:move_action}, we can fill in |Action|'s optional last parameter (which otherwise
defaults to |Seq()|):
\begin{lstlisting}
  addAction(Action('Move,
    And(Predicate2("==", engine, Constant(T)), 
        Predicate2("==", mode, Constant(Air))),
    Seq(Assignment(pos_x, Application("+", pos_x, Variable('dx))),
        Assignment(pos_y, Application("+", pos_y, Variable('dy)))),
    Seq(Variable('dx), Variable('dy)))
  )
\end{lstlisting}
The whole system, defined on this level of the syntax, can be found in \aref{sec:simplerocketnodsl}
(for comparison, \aref{sec:simplerocket_original} contains the definition in the original external
syntax).

There is not much more to be said now about constructing the system; in principle, the shown methods
are all which is needed. Also, the term \dsls{} are not much more complicated than in the examples
above: value expressions are constants, variables, or function applications; conditions are
predicates on expressions or propositional terms thereof, using |And|, |Or|, and |Not|; and
statements are mostly |Assignment|s of a variable to an expression.


%--------------------------------------------------------------------------------
\section{Underlying Architecture and Funtionality}
\label{sec:execution}

The library is built up in as modular a way as possible~-- what can be parametrized, will be
parametrized, and what can be factored out in a trait, will be in a trait. On the other hand, as
Scala makes it very easy to introduce small wrappers and \abbrev{ADT}s in the form of case classes,
these are also frequently used, even if a plain type maybe would have been enough. A rough graphical
overview of the most important classes and traits is given in \autoref{fig:architecture}, which is
not really a class diagram, but rather the relations that could be found in an example case for some
|ConcreteActionSystem|. Note that the members listed there are not complete, and their types in some
cases are simplified; the diagram only tries to capture the essence of the typical hierarchy.

\newthought{The main part of \kern-0.5pt{\footnotesize\texttt{ACTIUM}}'s functionality} is
implemented in the trait |ActionSystem|. As indicated by the name, this trait represents the
functionality of an Action System, which consists of adding actions, initializing and resetting the
state, and executing the system. The respective methods for these abilities are provided by the
trait, together with fields containing the necessary state. There are two kinds of things left
abstract by |ActionSystem|: the |State| type, and the way actions and parameters are practically
chosen at execution.

While most methods of |ActionSystem| are actually just more complex getters or setters, the main
logic of execution lies in |run|. This is technically also not really a method (although it compiles
to one); rather, the result of |run| is a recursively defined |Stream[Choice]|. Streams are Scala's
standard implementation of lazy collections~-- they can be potentially infinitely long, since next
elements are calculated only on demand. |Choice| is not more than a case class containing the label
of an action, and a map of the chosen parameters:
\begin{lstlisting}
  final case class Choice[State](label: Label,
    params: Map[Variable[State], State] = Map[Variable[State], State]())
\end{lstlisting}


\begin{sidewaysfigure}[p!]
  \changecaptionwidth
  \captionwidth{0.72\stockheight}
  \centering
  \tikzumlset{font=\scriptsize\sffamily}
  \begin{tikzpicture}[]
    \usetikzlibrary{arrows, positioning}
    \tikzumlset{font=\scriptsize\sffamily}
  
    \tikzset{%
      inherit/.style={>={open triangle 60}},
      depend/.style={>={angle 60}, dashed}}

    \begin{umlpackage}{actium}
      \begin{umlpackage}[x=7, y=0]{expressions}
        \umlemptyclass[x=0]{trait Condition}
        \umlemptyclass[x=3]{trait Expression}
        \umlemptyclass[x=6]{trait Statement}
      \end{umlpackage}
      \umlclass[x=0, y=-2.5]{trait ActionSystem}{
        +\umlvirt{type State} \\
        -environment: MutableMap[Variable[State], State] \\
        -actionList: MutableMap[Label, MutableSet[Action]] \\
      }{
        +run: Stream[Choice] \\
        +addAction: Action => Unit \\
        +initialize: Assignment[State] => Unit \\
        +reset: () => Unit \\
        \#\umlvirt{chooseAction: Seq[Action] => Option[Action]} \\
        \#\umlvirt{chooseParameters: (Label, Seq[Variable[State]]) => Map[Variable[State], State]} \\
      }
      \umlclass[x=0, y=-7]{object ConcreteActionSystem}{
        +type State = ConcreteState
      }{}
      \umlclass[x=10, y=-7]{trait GherkinSyntax}{}{
        +when, +given, +and, +but, +then\_do 
      }
      \draw [->, inherit] (object ConcreteActionSystem.north) -- (trait ActionSystem.south);
      \draw [->, inherit] (object ConcreteActionSystem.east) -- (trait GherkinSyntax);
      \draw [->, depend] (trait GherkinSyntax.north) -- (expressions.south);
      \umlclass[x=11, y=-3]{trait RandomChoice}{
      }{
        +chooseAction: Seq[Action] => Option[Action] \\
        +chooseParameters: (Label, Seq[Variable[State]]) => Map[Variable[State], State] \\
      }
      \draw [inherit] (object ConcreteActionSystem.north) -- +(0, 1) [->] -| (trait RandomChoice.south);
      \draw [->, depend] (trait RandomChoice.west) -- +(west:1.15cm);
      \draw [->, depend] (trait RandomChoice.north) -- +(north:1.15cm);
      \draw [->, depend] (trait ActionSystem.10) -| (expressions.220);
      \draw [depend] (trait GherkinSyntax.north) -- +(0, 0.5) [->] -| (trait ActionSystem.300);
    \end{umlpackage}
  \end{tikzpicture}
  \caption{\footnotesize Overview of the most important parts of the architecture of \actium{}. This
    is an example configuration for an imaginary concrete Action System. The trait members are not
    all shown in their precise form; this is just to provide an overview from the implementor's
    point of view. Any concrete Action System needs to inherit from
    \lstinline[columns=fixed]|ActionSystem| and some implementation of choice, which provides
    \lstinline[columns=fixed]|chooseAction| and
    \lstinline[columns=fixed]|chooseParameters|. Furthermore, the implementation will mix in the
    \dsl{} syntax from \lstinline[columns=fixed]|GherkinSyntax|.  All these traits depend on
    \lstinline[columns=fixed]|ActionSystem| by restricting their self
    types.\label{fig:architecture}}
\end{sidewaysfigure}

In summary, when calling |run|, one has at hand a lazy |Stream| of the stepwise results of the
execution of the system (as it is done according to the concrete choice implementation). From that
stream, one can \enquote{take off} arbitrarily many steps by calling the stream's |force| method
(usually the number of taken elements will be restricted in some form, \eg using |take|):
\begin{lstlisting}[style=break-lines]
  scala> val r = new SimpleRocket with RandomChoice
  r: actium.examples.simpleRocket.SimpleRocket with actium.RandomChoice = $anon$1@69005b59

  scala> r.run.take(5).force
  res0: scala.collection.immutable.Stream[r.Choice] = Stream(Choice('PowerOn), Choice('Start), Choice('Destroy))
\end{lstlisting}
The actual updates to the state only happen when the elements of the stream are evaluated, and
persist afterwards~-- so, before calling |run| again from the initial state, the system has to be
|reset|. With functions like |take| it is possible to only run as many actions as desired, and then
continue later with the execution. In the example, there were not even run as many actions as
possible, since the execution ended in the final state |'Destroyed| after only three steps.

To define a concrete instance, the abstract members must be provided by the concrete object or class
for the system in question. Concrete methods for choosing parameters and actions are in the current
implementation usually factored out into \emph{choice traits}, which can be used to mix in various
behaviours of choosing; at the moment, there are three useful variants of them: |RandomChoice| uses
a uniform random distribution to choose actions and parameters, |IOChoice| provides a console
interface for entering all information, and |StaticChoice| is thought mainly for testing, as it
allows to fix a sequence of actions and parameters beforehand. Of course, these traits could also be
left out totally, and a new, custom way of choice be implemented.

The |State| type is typically defined inside the body of the system object, where the main behaviour
is defined using the necessary calls of |initialize| and |addAction|. It is also possible to just
define |State| and the actions, and leave the choice abstract, allowing it to be mixed in at a later
time. This can be used to treat a system differently, \eg, for testing its correct behaviour and
actually using it for running a simulation, or for simply exploring it in the console.

\pagebreak[4]
\newthought{The subpackage \kern-0.5pt{\footnotesize\texttt{ACTIUM.EXPRESSIONS}}} contains the
classes for the \abbrev{ADT} representation of expressions, statements and conditions (predicates on
expressions). The three of these are abstract sealed traits, implemented by case classes for the
possible values of each (such as |And| for |Condition|, or |Assignment| for |Statement|). The
classes are also all tagged with a |State| type, and the traits contain evaluation of the
representations as functions taking an environment and returning a |State| (or a |Boolean|, in the
case of |Condition|).

Expressions are the most important building blocks of the behaviour of an |ActionSystem|, since they
determine when and how the state changes. The whole representation of an action is in fact defined
just like this:
\begin{lstlisting}[label=lst:action]
    final case class Action[State](label: Label,
      condition: Condition[State],
      statements: Seq[Statement[State]],
      params: Seq[Variable[State]] = Seq[Variable[State]]())
\end{lstlisting}
That describes an action by its label, \abbrev{ADT}s for the conditions upon which it gets enabled
(these can depend on the state as well as the parameters of the action) and the statements which
will be executed in the action, and optionally a symbolic representation of the parameters the
action takes. The main part of the definition of an Action System consists of constructing such
|Actions| out of small \abbrev{AST}s for statements and conditions, and registering them using
|addAction|.

There is not really much \abbrev{DSL}-specific to say at this point but the following: since the
definition of |Action| shows that an Action System's behaviour is represented symbolically by an
\abbrev{AST}, \actium{} can be called a deep embedding (\cf \autopageref{dsl-definitions}). This
form of representation was chosen because it does not only allow execution in different ways, but
also can be transformed into other formats and used for different purposes than just running a
system. For example, there exists a translation function to the \abbrev{AST} of the |as2bmc|
library~\cite{maderbacher2016:weak}, which in turn is able to transform Action Systems into logical
expressions for the |Z3| \abbrev{SMT} solver (see~\autosubref{sec:extending_functionality}). These
can be used to perform mutation-testing by constraint solving, as mentioned in the introduction of
this part (see~\autopageref{constraint_solving}).


%--------------------------------------------------------------------------------
\section{Improving the Syntax}
\label{sec:improving_syntax}

This subsection is concerned mainly with what is happening in |GherkinSyntax|. This trait is an
optional mixin for |ActionSystem|, but forms the main point of exploration relevant to this
work. For implementing an |ActionSystem| in the \abbrev{DSL} style, one can use language features
and |GherkinSyntax|' helpers on three levels: basic definitions, action specifications, and
expressions.

\newthought{The main \dsl{} part} comes in at taking the methods shown above, and improving the way
they are written. Until now, all methods shown are already implemented in |ActionSystem|; now the
real \dsl{} part, as implemented in the mixin |GherkinSyntax|, is explained. Its purpose consists
mostly of providing methods to replace the above literal form of initialization and defining actions
by something more natural, ideally in a style very similar to the shown external
\dsl{}. Additionally, there are a variety of helpers provided to construct expressions and
conditions in a more natural way, looking like native expressions, relieving the need of
constructing their \abbrev{AST}s by hand. Both kinds of helpers make heavy use of implicits and
extension wrappers.

\begin{lstlisting}[style=floating, label=lst:expressions,
  caption={Simplified definitions of all \abbrev{AST} classes in the \lstinline|expressions|
    sub-package.}]]
  sealed trait Expression[+A]
  case class Application[+A](op: String, args: Expression[A]*) 
    extends Expression[A]
  sealed trait Value[+A] extends Expression[A]
  case class Constant[+A](value: A) extends Value[A]
  case class Variable[+A](name: Symbol) extends Value[A]

  sealed trait Statement[+A]
  case class Assignment[+A](variable: Variable[A], 
    assignment: Expression[A]) extends Statement[A]
  case class ExternalAction[A](run: () => ExternalResult)
    extends Statement[A]

  sealed trait Condition[+A]
  case class And[+A](a: Condition[A], b: Condition[A]) extends Condition[A]
  case class Or[+A](a: Condition[A], b: Condition[A]) extends Condition[A]
  case class Not[+A](a: Condition[A]) extends Condition[A]
  case class Predicate1[A](p: String, a: Expression[A]) 
    extends Condition[A]
  case class Predicate2[A](p: String, a: Expression[A], b: Expression[A])
    extends Condition[A]
  case object True extends Condition[Nothing]
  case object False extends Condition[Nothing]
\end{lstlisting}

To understand how the syntax for \actium{} was chosen, consider again the following action
definition from the external sytax:
\begin{lstlisting}[language={}]
  powerOn() if engine == 0 && !(mode == Destroyed) then {
    engine := 1; 
  };
\end{lstlisting}
To translate this into a Scala expression, we can inspect its parts and proper ways of representing
them. First, the guard of the action,
\begin{lstlisting}[language={}]
  engine == 0 && !(mode == Destroyed)
\end{lstlisting}
is just a boolean expression which should, when translated, end up in the following \abbrev{AST}
representation, as we have seen above:
\begin{lstlisting}
  And(Predicate2("==", engine, Constant(F)), 
      Predicate2("!=", mode, Constant(Destroyed)))
\end{lstlisting}
How such expression syntaxes can easily be embedded into Scala by the help of operators has been
shown in \autoref{sec:method_calling}. Because we are dealing essentially with predicate logic, the
definition of these operators spreads over two places: |Expression| and |Condition|. In |Condition|,
only the boolean combinators such as |&&| or \lstinline[style=inline]$||$ are defined, whereas
predicates live in |Expression| (since they operate on values and just return booleans). It should
be noted that, because of issues concerning covariance of the |Expression| trait when constructing a
|Predicate2| with the same type parameter, the respective operators are defined not directly in the
trait, though, but in an implicit wrapper; there, for example, we have the following definitions:
\begin{lstlisting}
  def ===(other: Expression[A]): Condition[A] = 
    Predicate2("==", expr, other)
  def =!=(other: Expression[A]): Condition[A] = !(expr === other)
\end{lstlisting}
The use of |===| instead of the more natural |==| is required, because the latter is already defined
in |Any| and deepley nested into the language, and thus is not overrideable in a practical way. The
choice for |=!=| instead of |!==| was of a different kind: here, one of the exceptions of the
precedence rules for operators (\autopageref{operators}) comes into play, namely, that an operator
ending in |=| is treated as an assignment operator~-- \emph{unless} its name also starts with |=|,
in which case it is treated like a comparison operator.

With the help of these operator methods, we are already able to write things like 
\begin{lstlisting}
  Variable('engine) === Constant(0)
\end{lstlisting}
What remains is to also get rid of the explicit constructors for constant literals. We might try to
achieve this by providing the according implicit conversions:
\begin{lstlisting}
  implicit def literalToConstant(s: State): Constant[State] = Constant(s)
  implicit def symbolToVariable[A](name: Symbol): Variable[A] = 
    Variable(name)
\end{lstlisting}
Unfortunately, this does not really work out using the current setting. The reason is that to be
able to write
\begin{lstlisting}
  'engine === 0
\end{lstlisting}
two implicit conversion would have to be called: one from |Symbol| to |Variable|, and one for the
wrapper for |Expression| containing |===|.

This problem can be resolved in three ways: on the one hand, one could change the place in
which |===| was defined to avoid the double wrapping~-- this can be done by implementing |===|
directly in |Expression|, or by additionally providing a wrapper for |Symbol| which contains |===|
and performs the wrapping in |Variable| automatically. Both of these have disadvantages, though:
making |===| part of |Expression| enforces defining the trait's type parameter for |State| to be
invariant, which might lead to future complications when introducing better support for more complex
|State| types; and implementing a double wrapper for |Symbol| will lead to more complex definitions,
while it does not even solve the problem fully, as using too many implicit conversions on
interferring levels leads to unexpected complications of them.

On the other hand, there is a very simple, although not that succint solution, which has been
chosen in the current implementation and the examples: we just do not use an implicit conversion for
the left hand side of a call to |===|. Instead, we can define the state variable |'engine| as a
|val| in the scope of the class; this has already been shown in the previous subsection:
\begin{lstlisting}
  val engine = Variable('engine)  
\end{lstlisting}
In order to make this definition a bit more linguistically applealing, there is provided a simple
method
\begin{lstlisting}
  def statevar(name: Symbol) = Variable[State](name)
\end{lstlisting}
allowing to write instead
\begin{lstlisting}
  val engine = statevar('engine)
\end{lstlisting}
(Although this is arguably not too big an improvement; we might instead wish to get rid of the
redundancy of mentioning the name of the state variable twice, but this is currently only possible
by using experimental, non-standard macro facilities such as macro annotations.)

Now that these operators and conversions are in place, we finally are able to write
\begin{lstlisting}
  engine === F && mode =!= Destroyed
\end{lstlisting}
whereby |engine| and |mode| are |Variables| defined as |val|s, and |F| and |Destroyed| are just
names for the ints |0| and |2|, in this case.

Assignment statements and value expressions work in a similar way to guards\slash |Condition|s. The
operator |:=| is just a method of |Variable|, taking an |Expression| and returning an
|Assignment|. In this case, the operator has an intuitive name, and its precedence is as
expected. The right hand side of this operator can be an arbitrary expression.  Expressions
themselves are either |Variable|s (containing a symbol for the name), |Constant| (containing a
literal value of type |State|), or |Application|s of one operator (represented by its name) to a
list of other expressions.


\newthought{Given these basic building blocks} (expressions, conditions, and statements) and their
symbolic (\dsl{}) representations, we are finally ready to look into the construction of action
specifications such as this:
\begin{lstlisting}
  when('Destroy) given mode === Air then_do (
    mode := Destroyed,
    engine := F
  )
\end{lstlisting}
Converted to the underlying \abbrev{AST} translation, this should look like the following code:
\begin{lstlisting}
  addAction(Action('Destroy,
    Predicate2("==", mode, Constant(Air)),
    Seq(Assignment(engine, Constant(F)), 
        Assignment(mode, Constant(Destroyed))))
  )
\end{lstlisting}

To be able to achieve this in Scala, the trait |GherkinSyntax| introduces the \enquote{keywords}
|when|, |given|, and |then_do| (as mentioned above, the first two names were taken from
Gherkin). These keywords are all methods called in dotless style, and work by repeatedly updating
information in immutable builder objects. These are keeping track of the data provided, which
finally gets used in |then_do| to call |addAction| of the |ActionSystem|. For example, the above
definition (|when('Destroy) ...|) can be reduced in one step to the following:
\begin{lstlisting}
  new WithLabel {
    def label = 'Destroy
    def params = Seq()
  }.given(Predicate2("==", mode, Constant(Air)))
   .then_do(Seq(Assignment(engine, Constant(F)), 
                Assignment(mode, Constant(Destroyed))))
\end{lstlisting}
The |WithLabel| builder is the actual result of the |when| method, and contains the name and
optionally the parameters of the action. |given| is a method of |WithLabel|, which in its body
constructs a new builder with more information, namely, the condition, and notes this information in
the result type; therefore, the above reduces further to this:
\begin{lstlisting}
  new WithLabel with WithCondition {
      def label = 'Destroy
      def params = Seq()
      def condition = Predicate2("==", mode, Constant(Air))
  }.then_do(Seq(Assignment(engine, Constant(F)), 
                Assignment(mode, Constant(Destroyed))))
\end{lstlisting}
We now have an object of type |WithLabel with WithCondition|. For this type, an implicit wrapper is
defined, containing the final |then_do| method, taking the list of statements as variadic parameter
list. It uses the accumulated information, puts it into an |Action| object, and adds it to the
system; inlining the accumulated data into |then_do| leads to this final result:
\begin{lstlisting}
   { val a = Action('Destroy, Predicate2("==", mode, Constant(Air)), 
                 Seq(Assignment(engine, Constant(F)), 
                     Assignment(mode, Constant(Destroyed))), 
                 Seq())
     addAction(a) }
\end{lstlisting}
which is exactly equivalent to the translation of the example we began with.

\begin{figure}
  \centering
  \begin{tikzpicture}[->, thick, grow=right, 
    level 2/.style={sibling distance=3.5em},
    level 1/.style={sibling distance=6em, level distance=5em},
    every loop/.style={min distance=5mm, max distance=7mm, looseness=10}]
    \node {}
      child {node (left when) {\lstinline|when|}
        child {node (left given) {\lstinline|given|}}
        child {node (left then_do) {\color{textred}\lstinline|then_do|}}}
      child {node (right given) {\lstinline|given|}
        child {node (right when) {\lstinline|when|}}
        child {node (right and_but) {\lstinline|and|, \lstinline|but|}}};
    \node (left and_but) [right=of left when, xshift=5em] {\lstinline|and|, \lstinline|but|};
    \node (right then_do) [right=of right given, xshift=5em] {\color{textred}\lstinline|then_do|};
    \draw (left given) -- (left then_do);
    \draw (right when) -- (right and_but);
    \draw (right and_but) -- (right then_do);
    \draw (right when) -- (right then_do);
    \draw (left and_but) -- (left then_do);
    \draw (left given) -- (left and_but);
    \path (right and_but) edge [loop above] (right and_but);
    \path (left and_but) edge [loop right] (left and_but);
  \end{tikzpicture}
  \caption{Hierarchy of valid command sequences. The highlighted \lstinline|then_do| always
    concludes a sequence; by applying it to a list of statements, the actions gets defined and 
    added to the system. 
    \label{fig:commands}}
\end{figure}

This describes in the most basic form the idea behind the definition statements. Actually, the
situation is more complicated; this is why there are different types of builders and extensions
involved. The reason for the complication is to allow the user a certain freedom in the order and
choice of the keywords. For example, the \dsl{} allows also the following to be written:
\begin{lstlisting}
  given (mode === Air) when 'Land then_do (
    mode := Ground
  )
\end{lstlisting}
In this case, the order of |given| and |when| is turned around. Or, we could also leave out the
condition, if the guard is trivial, like in this hypothetical action:
\begin{lstlisting}
  when('OpenWindow) then_do (
    window_state := Open
  )
\end{lstlisting}
The problem is that in a naive implementation of the keyword methods, |then_do| could be called on a
builder object in an invalid state; for example, if it were defined on the object returned by
|given|, we could define the following action:
\begin{lstlisting}
  given(window_state === Open) then_do (
    temperature := Cold
  )
\end{lstlisting}
which is not a proper action definition, since it lacks the label. 

To ensure that only valid sequences of keywords can be used to build up an action, we first fix the
allowed keywords and their order. This is done in \autoref{fig:commands}. Given that hierarchy, and
the knowledge that there are two relevant parts of information expressed by |given| and |when| (the
label and the guard of an action), a type-safe chain of calls can be formed. The necessary setup to
enforce this is listed in \autoref{lst:commands} (in a truncated way, since only the interface is
relevant here). 

\begin{lstlisting}[style=floating, label=lst:commands,
  caption={The trait setup to ensure that only valid sequences of statements can be used in an
    action definition. These methods and traits are all defined in \texttt{GherkinSyntax} and mixed
    in with it. The implementations are shown as unimplemented here (\texttt{???}).}]
  def when(lbl: LabelWithParams): WithLabel with Given = ???
  def given(cond: Condition[State]): WithCondition with When = ???

  trait WithLabel {
    def label: Label
    def params: Seq[Variable[State]]
  }

  trait WithCondition {
    def condition: Condition[State]
  }

  trait When { self: WithCondition =>
    def when(lbl: LabelWithParams): WithLabel with WithCondition = ???
  }

  trait Given { self: WithLabel =>
    def given(cond: Condition[State]): WithLabel with WithCondition = ???
  }

  implicit class WithLabelAndConditionWrapper(
      builder: WithLabel with WithCondition) {
    def then_do(stmts: Statement[State]*): Unit = ???
  }

  implicit class WithLabelWrapper(builder: WithLabel) {
    def then_do(stmts: Statement[State]*): Unit = ???
  }
\end{lstlisting}

To achieve the restriction of call sequences, two methods |given| and |when| are defined at top
level, each returning the respective builder object, extended with the possibility to call the
\enquote{opposite} method via the traits |Given| and |When|. By having the methods |given| and
|when| twice, and in separate place, it is ensured that each of them can only be called once, but in
any order. Both methods store the particular information collected at the point of their calling in
the partial builder traits |WithLabel| and |WithCondition|. At last, for |WithLabel| and %
|WithLabel with WithCondition|, implicit wrappers are provided implementing the final calls to
|then_do|. 

In addition to the methods shown in the listing, the wrappers for |WithCondition| and %
|WithLabel with WithCondition| also contain the methods |and|, |or|, and |but|, which can be used as
synonyms for |given| to add further conditions in a natural-language fashion~-- their only
functionality is to produce a conjunction or disjunction of the |Condition| on the left and on the
right; for example, the definition\enlargethispage{1em}
\begin{lstlisting}
  when('PowerOn) given engine === F but mode =!= Destroyed then_do (
    engine := T
  )
\end{lstlisting}
would be equivalent to simply
\begin{lstlisting}
  when('PowerOn) given engine === F && mode =!= Destroyed then_do (
    engine := T
  )
\end{lstlisting}


The only thing that remains to be explained now is how action parameters are allowed to be specified
in the convenient call-like fashsion looking like
\begin{lstlisting}
  when('Move('dx, 'dy))
\end{lstlisting}
The solution is relatively simple: the argument type of |when| is not |Symbol|, but actually
|LabelWithParams|, a small wrapper containing the label and optionally the parameters:
\begin{lstlisting}
  case class LabelWithParams(label: Label, params: Seq[Variable[State]])
\end{lstlisting}
To be able to just use symbols and normal application, the following implicit conversions are in scope:
\begin{lstlisting}
  implicit def symbolToLabel(s: Symbol): LabelWithParams = 
    LabelWithParams(s, Seq())

  implicit class LabelWithParamsBuilder(s: Symbol) {
    def apply(params: Variable[State]*): LabelWithParams = 
      LabelWithParams(s, params)
  }
\end{lstlisting}

In principle, we are now able to fully emulate the original syntax for Action Systems. Some more
intricate details of implicit conversion and syntactic tricks have been left out, but the basic
principles should have been made clear. The full |SimpleRocket| example at this point is listed in
\aref{sec:simplerocket}.


%--------------------------------------------------------------------------------
\section{Extensions to the Basic Functionality}
\label{sec:extending_functionality}

Finally, besides the basic \dsl{} framework described in the last section, there have been
implemented some small extensions. Two of these concern the semantics, as compared to the existing
implementation, while the third one deals with enabling integration of this \dsl{} into another
existing framework, which forms the basis for a mutation testing pipeline.

\newthought{The first one of these} is quite simple: \actium{} allows to define multiple actions
with the same label; that is, using |when('Something)| multiple times will not cause any
problems. This is handled internally by simply associating with each label not an |Action|, but a
|Set[Action]|, and flattening the sets accordingly at evaluation. The individual actions are,
however, always properly distinguished when needed; for example, when running the system with a
choice trait, or when converting a system to an alternative symbolic representation.

\newthought{The second semantic extension} is the introduction of an additional statement besides
assignment, which can not be found in other Action System formulations: |externally|. This statement
is defined in the extra trait |ExternalEffects| (which is a mixin to |GherkinSyntax|), and allows to
embed arbitrary Scala code in an action; this code will be executed each time the action is
executed. In the simplest form, this can be just a debug message:
\begin{lstlisting}
  when('Destroy) given mode === Air then_do (
    mode := Destroyed,
    engine := F,
    externally(println("BOOM!"))
  )
\end{lstlisting}
The content of |externally| is a closure (represented internally as |() => Unit| and passed
by-name), and can thus access external state in the form of |var|s defined in the class, or even
through library calls. However, what's even more powerful is that there is full access to the
systems state variables and the action's parameters at the point of executing the external
closure~-- which is somewhat more complex to realize. Concretely, we can have an action like this:
\begin{lstlisting}
  when('Position('x, 'y)) given mode === Air && engine === T then_do (
    pos_x := 'x,
    pos_y := 'y,
    externally {
      println(s"x = ${'x.value}, pos_x = ${'pos_x.value}")
    }
  )
\end{lstlisting}%$
(whereby |pos_x| is part of the state variables if the |SimpleRocket| system). 

What is important here is the usage of the form |'x.value| to access variables from the scope of the
action. The method |value| is defined as an extension for |Symbol|~-- and it takes as an implicit
parameter an |ImplicitEnv[State]|. This class is just a wrapper about a |Map[State]|. The fact
exploited to make this work is that the actual implicit value, that is provided for this parameter,
is defined as a |var|. That way, at each evaluation the environment can be filled with the current
values. For better understanding, look at the simplified variant of the pattern in
\autoref{lst:external_example}. There the structure of the relevant commands have been
\enquote{rebuilt}, in a way to resemble the original implementation. It is important to note that
|execute| already closes over the reference to the empty environment.

\begin{lstlisting}[style=floating, label=lst:external_example,
  caption={A simplified example of how the variable capturing in external statements is
    implemented: by capturing an implicit reference, and setting it later to the then current
    environment. \texttt{externally} and \texttt{ValueWrapper} are in reality provided by 
    the mixin \texttt{ExternalActions}; \texttt{ExternalAction} is defined in the statement
    \abbrev{ADT}.}]
  implicit class ValueWrapper(s: Symbol)(implicit env: Map[Symbol, Int]) {
    def value = env(s)
  }

  case class ExternalAction(block: () => Unit)

  def externally(block: => Unit) = ExternalAction(() => block)
  
  implicit var environment = Map[Symbol, Int]()
  def then_do(as: ExternalAction*) = new {
    def execute(env: Map[Symbol, Int]) = {
      environment = env
      as foreach (_.block())
    }
  }
\end{lstlisting}

If we now use this example framework to define an action, like so:
\begin{lstlisting}
  val pseudoAction = then_do(
    externally {
      println('y.value)
    }
  )
\end{lstlisting}
the reference to |environment| also gets passed implicitly to |value|, and is therefore used in
|ValueWrapper|'s definition to look up the value of |'y|. However, since all this is passed by-name,
the lookup will not happen immediately and use the empty dictionary; only when we execute the
pseudo-action, like this:
\begin{lstlisting}
  pseudoAction.execute(Map('x -> 1, 'y -> 42))
\end{lstlisting}
the closure will actually be called~-- but before calling, the environment reference will be updated
according to the parameter of the |execute| method, which in the case of the actual |ActionSystem|
resembles the current state and the action parameters. As a result, the value |42| will be printed
out, being declared as |'y|'s value only after the external effect has been defined in the code.

Furthermore, in order to make this statements usable for actual testing scenarios, |externally|
supports aborting the execution of the system. For this purpose, there exists a member
\begin{lstlisting}
  def abort: Nothing = throw AbortExternallyException
\end{lstlisting}
and |externally| is, in fact, not only a trivial wrapper around a thunk, but catches this exception
and \enquote{transforms} the result:
\begin{lstlisting}
  def externally(block: => Unit): ExternalAction[State] = 
    ExternalAction(() => {
      try {
        block
        Succeeded
      } catch {
        case AbortExternallyException => Failed
      }
    })
\end{lstlisting}
Therefore, also the result of the closure used in |ExternalAction| is not |Unit|, but
|ExternalResult|, which can be either |Succeeded| or |Failed|. This result is checked in the |run|
method of |ActionSystem|, and, if negative, stops it from producing further values for the
|Stream[Choice]|.

This possibility should help when, for example, using external statements to do side-by-side testing
of an underlying \abbrev{SUT}. One could imagine to use an Action System to simulate expected
behaviour, and at every step comparing its modelled output to the real system; if a erroneous
difference occurs, the execution can be halted immediately. A modification of the SimpleRocket
example, involving examples of external statements, is given in~\aref{sec:extendedsimplerocket}.

Using exceptions in such a way, namely, to introduce non-local control flow, might be considered
almost an abuse by some people~-- this estimation is, however, largely depending on one's
background. For example, Scala, unlike many other languages, does not have a |break| statement for
loops. Instead, a very similar construct to the above is used in the standard library to implement
|scala.util.control.Breaks|~-- providing a \emph{library} function implementing |break|%
\footnote{\protect\url{http://www.scala-lang.org/api/current/index.html\#scala.util.control.Breaks}
  (visited on 2015-06-30)}. This power of exceptions to implement non-local control flow has been
explored in~\cite{lillibridge1999:unchecked}, where a variant of them is shown to be equally
powerful to the |call/cc| operator of many \abbrev{LISP}s, and is also marginally discussed
in~\cite{pierce2002:types}. It can be considered very useful for \dsls{} in situations like this~--
if used wisely.

\newthought{As a third enhancement} to the original functionality, there is provided a mixin which
allows to translate the symbolic representation of the action system to another representation,
which might be used by an external library (for example, to actually perform mutation testing on the
defined system). As previously mentioned, there is one translator to the \texttt{as2Bmc}
library~\cite{maderbacher2016:weak}. This functionality is provided in a trait |As2BmcTranslator|,
having a self type of |ActionSystem { type State = Int }|, which is a refinement type ensuring that
the trait can only be mixed in to |ActionSystem|s with a |State| type of |Int|~-- the reason for
this being that currently, |ActionSystem| does not have a sufficient handling of arbitrary types,
and integers are in most cases enough to encode all necessary information of a system.

\begin{lstlisting}[style=floating, label=lst:translator,
  caption={Overview of the body of \texttt{As2BmcTranslator}. The actual implementations are
    left out.}]
  trait As2BmcTranslator { self: ActionSystem { type State = Int } =>
    def toAs2Bmc: ast.ActionSystemTy = ???
    private def translateExpression(
      expr: exp.Expression[State]): ast.ExpressionTyped = ???
    private def translateCondition(
      cond: exp.Condition[State]): ast.ExpressionTyped = ???
  }
\end{lstlisting}

An overview of the translation mixin's body is listed in~\autoref{lst:translator}. There, it can be
seen that there is only one relevant method, |toAs2Bmc|, returning the \abbrev{AST} of an action
system in the \enquote{foreign} representation. The other two methods are used only internally to
translate the respective parts of the expression syntax, as indicated by their names; their
implementation is quite mechanical, since both representations are quite similar, and both consist
only of a |match| statement. For example, the first three lines of |translateExpression| are
\begin{lstlisting}
  case exp.Variable(v) => ast.VariableExpTy(v.name.toString, stateType)
  case exp.Constant(c) => ast.NumConstantExpTy(c, stateType)
  case exp.Application("+", a, b) =>
    ast.BinOpExpTy(ast.Addition, translateExpression(a), 
                   translateExpression(b), stateType)
\end{lstlisting}

As indicated, the translation of the basic \abbrev{AST} is rather trivial, since both
representations store the same fashion of Action Systems in a similar structure; the one thing to be
improved is the handling of types. They are represented internally via tagging in \texttt{as2bmc},
whereas \actium{} currently uses a path-dependent type in Scala, and can currently only be
practically used for very simple settings (mostly integers).


%%% Local Variables: 
%%% TeX-master: "document"
%%% End:

\chapter{Résumé: What Can Be Learned From This}
\label{sec:resumee}

This section shall reflect on the practical part, with respect to the questions stated in the
introduction: how do the described language features help with real programming problems, which of
them are most useful, and what is missing? What are advantages and disadvantages of the taken
approach? Where could it be improved, and what are possible future directions? How did the
development process took course, and what can be learned from the implementation and its design?

\newthought{In general, it can be said} that Scala fulfilled most of the expectations about its
syntax and semantics' abilities. Comparing the original syntax:
\begin{lstlisting}[language={}]
  powerOn() if engine == 0 && !(mode == Destroyed) then
  {
    engine := 1;
  };
\end{lstlisting}
with the finally achieved one:
\begin{lstlisting}
  when('PowerOn) given engine === 0 && mode =!= Destroyed then_do (
    engine := 1
  )
\end{lstlisting}
we see that a well matching analogy could be created, which would probably be understood right away
by someone used to the external syntax. That this matching is quite accurate can also be observed by
comparing the full original example in~\aref{sec:simplerocket_original} with its translation
in~\aref{sec:simplerocket}.

Furthermore, a few useful extra features have been implemented, such as a bit more flexibility in
the keyword usage (|given| and |when| can be swapped, |and|, |but| and or used for building the
guards, and multiple actions given the same label), or the addition of external statements to allow
useful mixing between the system behaviour and Scala code. The feasibility of integrating the
library with an existing toolchain has not been practically tested, but a translation of the
symbolic representation of \actium{} Action Sytstems to the format used by \texttt{as2bmc} has been
provided with little effort. Since a fully symbolic approch is used anyway, other translations
should also not really be a problem. 

\newthought{The process of developing the library} went through multiple steps. At first, a kind of
top-down approach was tried: this consisted of only defining an interface, which should resemble the
\enquote{look and feel} of the existing syntax, and contained just dummy implementations. Using this
syntax, a few difficulties and dead ends could already be ruled out, and most of the concrete
keywords and the \enquote{style of writing} were fixed (like the definition of all actions within
the primary constructor of a subclass of |ActionSystem|, or the idea of using Gherkin's keywords
|given| and |when|). Still, some syntactic subtleties of the \dsl{} and some practical needs could
not be identified at this stage.

Therefore, when it came to actually implementing the functionality of running Action Systems, the
previous code needed to be rewritten completely, leaving behind only its main ideas. The reason for
this was that it turned out to be much easier to try out different variants of expressing domain
specific constructs when a working system was already given~-- this is because then the constructs
can immediately be tested for fitting the underlying system, and for not interfering with each other
later. With that insight in mind, at first, an almost completely working non-\dsl{} implementation
of |ActionSystem| was written, which essentially stayed the same until the final version. On top of
this underlying implementation, the additional mixin |GherkinSyntax| was provided, building on the
few underlying constructs provided by |ActionSystem| in the way described
in~\autosubref{sec:improving_syntax}.

This turned-around approach proved much more practical. While for a smaller, more strictly defined
target \dsl{} syntax, it might actually be easier to start at the \enquote{top} of the library and
then implement out dummy interfaces, until an underlying implementation is completed, the more
bottom-up way of layering the convenience syntax above a predefined simple, conventional, and
working functionality actually turned out to be more fruitful. In retrospection, having such a
separation is also more desirable from architectural and maintainance points of view; for it allows
to separate concerns much better, leads to less coupling, and probably facilitates testing
(although, honestly speaking, since the whole development was of a rather experimental nature, these
concerns were not much considered and often neglected in praxis~-- but this does not relativize named
points). The a-posteriori introduction of other, not primarily considered \dsls{} (for example, an
external variant), or stacking multiple ones, is also enabled and simplified in this way.

\newthought{Concerning the advantages of Scala's syntax}, a certain dichotomy can be observed. On
the one hand, it almost everything that was wanted to be expressed could be implemented somehow. In
this respect, the ease of defining \abbrev{ADT}s with custom operators was of much help; but
especially implicits turned out to be the rescue to many of the more intricate problems (so, for
example, the solution to \enquote{lazy capturing} of state variables described
in~\autoref{lst:external_example}). Their universal applicability to interfere almost anywhere in an
expression, and in a way transform the types of values there, is a unique and highly beneficial
concept.

On the other hand, there were some problems that could not be solved (or rather, some desired syntax
that could not be expressed) by standard means. The primary example of this is the need to put the
statements of an action in round parentheses, separated by commata, instead of a block and thus
looking like a language statement. This is due to the fact that \actium{} needs to use a deep
embedding (to be able to operate on its underlying representation), which requires passing every
\enquote{syntactic form} of the \dsl{} as a Scala expression. Now, if a block were used to pass
statements into |then_do|, that would always evaluate to only the last expression in its body. By
defining |then_do| to take a variadic parameter list, this can be resolved, at the cost of a less
intuitive syntax. This problem could in principle be solved by using a macro for |then_do|, which
takes the block argument as an expression, splits it into the individual statment expressions, and
puts them in a list; however, as mentioned, practical macro implementations are outside the scope of
this work.

Another feature of Scala's syntax has a sometimes unintuitive behaviour as well: the dotless calls
used for |given| and |when|. Concretely, due to the way the parser is working and how the resulution
of dottless calls is defined, they cannot be interrupted by newlines; that is, writing
\begin{lstlisting}
  when('PowerOn) 
  given engine === 0 && mode =!= Destroyed then_do (
    engine := 1
  )
\end{lstlisting}
would not work, since the |given| in the second line is not recognized as a continued invocation of
the result of |when|, but as another statement. This is a basic limitation of Scala, and common
knowledge, but can be unexpected from the thinking perspective of a \dsl{} user. It is not a new
problem when using a deep embedding; for example, the internal \dsl{} variant of\texttt{sbt}'s
configuration syntax (used for automatizing Scala builts, somewhat like \texttt{make}) suffers from
the same limitation~\cite{sbt2015:reference}.

In a similar fashion, though not as much striking, is the in some ways complicated definition of
operator precedence used by Scala. In many cases, defined operators tend to \enquote{just work},
because the most common symbols are assigned their \enquote{usual} precedences, and sensible
exceptions are made for some special cases (such as recognizing |:=| as an assignment
operator). However, sometimes, a certain operator name is desired to be used in a certain position,
but will not do the right thing there without parentheses, because it associates in wrong ways. This
was the case when, at first, the inequality operator for values was defined as |!==|; the redefinion
as |=!=| does the right thing due to another exception rule, but noting this, and having to think
about it for every operator, can be frustrating for both developer and \dsl{} user. An explicit
possibility to manually define precedences, as in some other languages with operators, would be
desireable here (like Haskell's |infixl|/|infixr| declarations).

\newthought{When it comes to the expressiveness} of the Action Systems definable by \actium{}, there
are some limitations compared to the original implementation. This is due to the fact that this
work's main concern was to explore the limits of Scala for \dsl{} implementation; some additional
features present in other languages based on Action Systems were therefore left out.

For one, while the |ActionSystem| trait is parametrized by the type |State|, this is currently not
of too much use, and |State| is in all examples and tests defaulted or constrained to |Int|. While,
with some further work, the \dsl{} could be adapted to some custom state type (this requires syntax
and semantics for values, expressions, and predicates on it), this quite complicated (if not
impossible with standard features) to be done in general, in a satisfying way. The reason for this
is, again, that we are working with a deep embedding~-- which means that type information needs to
be present symbolically in the representation at runtime. On the other hand, the implementation is
currently layed out to support static type checking in Scala, which is in principle desirable (since
it allows to prevent certain errors at compile time).

The difficulty now lies in the fact that to get all desired features, one would need to represent
types both symbolically in the \abbrev{AST} and statically in Scala. This could theoretically be
done by using a more complicated encoding (using type tags, singletons, or some other technique
relying on more advanced typing techniques), but it is unclear whether such an encoding could be
done without breaking the current syntax, and without having to require the definer of a type to
write too much unnecessary boilerplate again. It would maybe be possible to achieve a satisfying
result using macros, again. (The automatic derivation of generic type representations in
Shapeless\footnote{\protect\url{https://github.com/milessabin/shapeless/wiki/Feature-overview:-shapeless-2.0.0\#generic-representation-of-sealed-families-of-case-classes}
  (visited on 2015-07-03)} looks primising for this purpose.)

Furthermore, there are some syntactic constructs which \actium{} currently lacks. For example, many
concrete implementations using Action Systems allow to define actions in a nested way (that is, to
define additional guards with different statements inside of one guard); adding support for these
would mainly consist of adding a syntactic transformation, and adapting the \dsl{} to behave
properly when the keywords are used not at top level.

Moreover, there has also been come up with the idea of adding a kind of pattern matching support for
action parameters. This would allow to make the writing of actions with similar guards easier and
more readable, especially combined with some sort of wildcard pattern. Implementing this would
require extending the parameter-passing mechanism a bit, and, most importantly, an algorithm for
unification and ordering patterns by specificity. Whether implementing a reasonable variant of the
latter can be done easily, without having to resort to, \eg, an external solver, is unknown to the
author.

Besides these enhancements of functionality, it would also be worthwhile to examine and enhance the
coherence of the implementation with some formal execution semantics. For instance, currently, the
order of execution of statements in an action is somewhat undefined (in practice, they are always
executed sequentially). But since Action Systems are a means of formalizing concurrent systems,
being able to exactly specify or know execution order would be a desirable property. It might also,
even in the current variant, sometimes be unclear how, for example, the side effects of two external
statements will be sequenced and influence each other.

There is also an issue arising from the frequent use of traits and mixins, which in principle turned
out to be a very practical pattern to separate concerns. While traits mostly allow to specify
several degrees of coupling very exactly, they are sometimes still too weak to describe what is
wanted to be achieved. This limitation was most obvious for the choice mixins, which define how
parameters and actions are chosen during execution of an Action System. There, it soon became
visible that while mixins are very practical for adding orthogonal static implementations, they are
not so easy to use for configuring behaviour. Concretely, at the moment, one can only define an
|ActionSystem| with certain choice behaviour by using the |with| syntax; but it often would be nice
to allow finer configuration, such as setting specific weights for actions in the |RandomChoice|
trait. Here, one could either investigate ways of elegantly allowing parametrization of mixin
behaviour, or think about another alternative not using mixins in this way at all (such as a
conventional design pattern). Reflecting this, we can conclude that traits are extremely useful for
orthogonal static, but less so for runtime parametrization.

Finally, the current implementation is a bit unsatisfactory from a \enquote{purist} functional
programming point of view. This is because all settings, definitions, but also the execution of
systems themselves are based on mutable updates. While this is not a major flaw, it would be
desirable to rewrite as much as possible of the execution to using immutable updates~-- this would
enhance some minor points, such as allowing easier pesistance of immediate states, or removing the
necessity of resetting systems between test runs. More immutability would also perceived as more
elegant by the author, and could lead to more correct code, as errors in mutable updates tend to be
harder to find. The best way to introduce immutability would probably be to run an |ActionSystem|
like a Mealy machine, or similar to a state monad (copying new state, instead of updating). On the
other hand, there is little that can be done to remove mutable updates from the action definition
syntax (the |given|\slash |when| statements in the primary constructor); but this seems more
acceptable, as it is done only once at object initalization (at least one could revisit the visibity
of |addAction|, and make it as private as possible).

%%% Local Variables: 
%%% TeX-master: "document"
%%% End:


%-------------------------------------------------------------------------------
\clearpage
\vspace{2em}
\appendix


\begingroup

\microtypesetup{protrusion=false, expansion=false}
%\raggedright
\sloppy \defbibnote{sips}{Note: features to be included in Scala are first proposed in Scala
  Improvement Proposals (\abbrev{SIP}s), collected at
  \url{http://docs.scala-lang.org/sips/index.html} while they are reviewed or \enquote{pending}. The
  \abbrev{SIP}s cited here are always noted as \enquote{pending} or \enquote{accepted} as of the
  state at the time of writing this (May/June 2015).}

\phantomsection
\addcontentsline{toc}{chapter}{Bibliography}
\printbibliography[heading=memoir, prenote={sips}]

\endgroup

%-------------------------------------------------------------------------------
\appendix
\chapter{Example Programs}
\label{sec:appendix}

\section{SimpleRocket in Original Syntax}
\label{sec:simplerocket_original}
\lstinputlisting[style=fullpage,language={}]{code/SimpleRocket.acsys}

\section{SimpleRocketNoDSL}
\label{sec:simplerocketnodsl}
\lstinputlisting[style=fullpage,firstline=6]{code/SimpleRocketNoDSL.scala}

\section{SimpleRocket}
\label{sec:simplerocket}
\lstinputlisting[style=fullpage,firstline=9]{code/SimpleRocket.scala}

\section{ExtendedSimpleRocket}
\label{sec:extendedsimplerocket}
\lstinputlisting[style=fullpage,firstline=8]{code/ExtendedSimpleRocket.scala}


%-------------------------------------------------------------------------------
\cleartoverso
\thispagestyle{empty}
\renewcommand{\abstractname}{Colophon}
\begin{abstract}
  \noindent
  This document was typeset using the pdf\LaTeX{} typesetting system, with the memoir document
  class. The body text is set in 11\,pt Linux Libertine, enhanced by the microtype package. Other
  fonts include Biolinum, Inconsolata, and \abbrev{GFS} Neohellenic. The drawings are typeset using
  TikZ/PGF, and source code examples are formatted by the listings package.

  The document source has been written in Emacs with AUC\TeX{} mode, using TeXworks as \abbrev{PDF}
  viewer.
\end{abstract}

\end{document}

%%% Local Variables: 
%%% TeX-master: "document-print"
%%% TeX-command-extra-options: "-shell-escape"
%%% End:
