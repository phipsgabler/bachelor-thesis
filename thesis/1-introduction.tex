\phantomsection
\addcontentsline{toc}{chapter}{Introduction}
\chapter*{Introduction}

Exploring Domain Specific Languages (\dsls{}) is a very broad topic. They have been around for a
long time, and inspected in many ways, for many different purposes, in many implementations. This
work therefore tries to focus on some core aspects in a limited setting. On the one hand, Scala's
capabilities for writing embedded \dsls{} shall be investigated and summarized. This includes the
inquiry of the relevant syntactic and semantic features of the language, of how and to what extent
they help in the implementation of \dsls{}, and the examination of existing patterns for that
purpose (of theoretical nature as well as occuring in practical software). This part of the
exploration is mostly theoretical or based on small examples~-- it contains explanations about the
language, about its background, and about how it can be used.

On the other hand, a somewhat larger, concrete example of an embedded \dsl{} shall be constructed in
practice. This example is a library for the definition of Action Systems, as they are used in
model-based mutation testing~\cite{aichernig2014:killing,aichernig2009:mutation}. With this
practical project, the actual development of a \dsl{} library in Scala will be analyzed, with the
aim of gaining insight into what a development process with such goals can look like (and what
difficulties it might involve), into where the language helps with the embedding of a \dsl{} and
where it hampers it, and finally into how useful the previously examined language featureas are in
practice~-- expecially, if some desireable properties are missing, or if some parts could in fact be
done better with different means.

\newthought{The work is structured as follows}: There are two logical parts. In the first three
sections, concepts and techniques for \dsls{} in Scala are explained.  At the beginning,
\autoref{sec:concepts} gives an overall view of domain-specific languages and Scala as a programming
language. It shortly introduces Scala's background and reviews \dsls{} in general, explaining their
foundation and position in the history of programming languages, discusses advantages and
disadvantages, and defines some commonly used terms. Then, \autoref{sec:syntax} gives an intoduction
into parts of Scala's syntax which are the most relevant and useful to writing embedded \dsls{},
includung examples of their usage. Thirdly, \autoref{sec:patterns} shows how the previously
introduced syntactic constructs can be combined into some useful patterns, which help program
organization and development when working with \dsls{}.

Building on that, the remaining three sections consist of the description of the practical part of
the thesis: an implementation of an embedded \dsl{} for describing Action Systems, with the working
title \actium. Initially, \autoref{sec:action_systems} gives the theoretical introduction into
Action Systems, and explains their usage and special variant in model-based (mutation) testing. It
follows \autoref{sec:implementation}, in which the implementation of \actium{} is
summarized. Special focus is layed on the design parts where the described above language features
were used; notable decisions or interesting constructs used are pointed out. This section also gives
an overview of the functionality and usage of the implementation. Finally, \autoref{sec:resumee}
reflects on the practical part, mentions possible improvements or further directions, and summarizes
the advantages and disadvantages for this type of implementation, as opposed to other possibilities.


%%% Local Variables: 
%%% TeX-master: "document"
%%% End: