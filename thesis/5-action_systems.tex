\chapter{Action Systems and Testing: Definition \& Background}
\label{sec:action_systems}

\lstset{deletestring=[b]'}

In the remaining part of this bachelor thesis, the practical work done is described. It consists of
a library to program a restricted variant of so-called Action Systems, a formalism similar to state
machines or Labelled Transition Systems. The resulting library provides basic functionality to
execute such Action Systems using various methods, and, most importantly, a \dsl{} to write them
concisely in Scala. As many parts as possible are parametrized and thus left open for more specific
use cases or future enhancements. The library has the current working title \actium, which is the
latinized name of the small town of \kern1pt\greek{Ἄκτιον}\kern-4pt in ancient Greek, place of the
Battle of Actium~-- but mostly, a pun on the word \enquote{action}.

The description of the implementation consists of three parts. First, in this section, an
introductory overview of the principles and the usage of Actions Systems and their application in
model-based testing is given, to establish a background for the applicability of the library and
terms commonly used. Following that, details of the implementation are provided, with a focus on the
\dsl{} interface and how its form could be achieved programatically, exploiting the language
features of Scala. In the couse of this, also an overview of the functionalities of \actium{} is
given. Lastly, the finished state of the project is summarized, and the lessons learned are
pronounced, which includes notes about the development process, about language features and their
usage, and about other directions that could have been followed or were not tried out. This last
section also reflects on the state of the library and its limitations, and possible improvements.

\newthought{Action Systems were introduced} to formalize the collaboration of processes in
distributed systems~\cite{back1983:decentralization}, and proposed as an approach dually to other,
more process-focused approaches such as Communicating Sequential Processes
(\abbrev{CSP})~\cite{hoare1978:communicating}. Somewhat informally, an Action System in that
original formulation consists of \emph{processes}, each having associated some variables with their
initializations, and (named) \emph{actions}, each consisting of a collection of participating
processes, a \emph{guard} (predicate), and a statement. An action (or rather, its statement) can be
executed if all of its processes are not currently participating in another action, and the guard is
satisfied~-- in that case, the action is said to be \emph{enabled}~\cite{back1988:distributed}. The
order in which the actions are executed, or how they are chosen if there are multiple enabled
actions, is left unspecified; the system can be executed using different strategies, synchronuously
or concurrently.

In the programming work done for this bachelor thesis, a restricted variant of the above formulation
is used, which is applied in the context of \emph{model-based mutation
  testing}~\cite{aichernig2014:killing}. Mutation testing~\cite{hamlet1977:testing} is a technique
to automatically generate a suite of useful test cases for a given system, by systematically
injecting small syntactic faults (called \emph{mutants}). A test suite is then generated from the
mutants by incrementally adding test cases which can distinguish a mutant from the original
implementation. However, to ensure that mutants actually do anything \enquote{useful} (and not
change the behaviour in a trivial way, or only change dead code), they must first be analyzed in
some way, which traditionally involved manual inspection. Mutants with such \enquote{useless}
behaviour are called \emph{equivalent} (since they are indistinguishable by test cases).

Model-based mutation testing combines this approach with model-based testing: instead of directly
operating on the system under test (\abbrev{SUT}), a model of it, called a \emph{test model}, is
used. This model, commonly described in a symbolic, abstract fashion, is assumed to be correct and
usually simpler than the \abbrev{SUT}, since only the properties of interest for testing are
modelled. It has the advantages that one has more control of the test generation process; for one,
it allows to treat the \abbrev{SUT} through the model in a purely abstract, black-box manner, which
enables non-software systems to be tested as well. Equally important, checking mutants becomes
easier, because the mutation process can be resticted to a more controlled set of syntactic mutation
operators, and equivalent mutants can be excluded by means of constraint solving. Furthermore, by
using such a model, nondeterminism, which can occur both in the \abbrev{SUT} or because of the
abstractions arising from the modelling process, becomes easier to deal with. A rather complete
overview of this methodology is given in~\cite{jobstl2014:model-based}.

\newthought{The point of Action Systems} in this environent now is to serve as test models for
mutation testing, especially of non-deterministic systems. Their quite simple, formal nature allows
to concisely write a model of the \abbrev{SUT}, and then transform this system into a symbolic
representation. This representation can be relatively easily mutated, and the resulting
representations can be passed to a constraint or \abbrev{SMT} solver which analyzes the system and
the mutants. That such an approach is not only possible, but can also be done efficiently, is shown
in~\cite{aichernig2015:model}.\label{constraint_solving}

\begin{lstlisting}[style=floating, label=lst:move_action, language={},
  caption={Description of an action \lstinline|move| in a concrete syntax. The action has two
    parameters, a guard, and a statement consisting of two sub-statements. \lstinline|mode|,
    \lstinline|engine|, \lstinline|pos_x| and \lstinline|pos_y| are (global) state variables.}]
  move(x:MyNat, y:MyNat) if mode == Air && engine == 1 then
  {
    pos_x := pos_x + x;
    pos_y := pos_y + y;
  };
\end{lstlisting}

As said above, for this purpose, it is practical to use a more narrowed down form of Action Systems
than those defined initially. It turns out that it is possible to write a system in such a way that
it looks similar to a state machine, but keeps the meaning of it as an Action System: we define a
set of state variables with their initial values, and a set of labelled actions. Each of the actions
can, in addition to the state variables, have parameters. It can also have a guard, involving its
parameters and state variables, and a statement, assigning new values to state variables depending
on their old values and the parameters. How such an action definition can look like is shown in
\autoref{lst:move_action}, using a syntax very similar to the one defined in~\cite[p.~16]{tappler2015:symbolic}.

In this formulation, the definition of processes is left implicitly; such systems are equivalent to
Action Systems in which only one process exists, which also contains all the state variables. The
named similarity to a state machine is not an unuseful one, but should be taken with care~-- the
labels of actions should \emph{not} be mistaken for states. Rather, the similarity is of a different
nature: the system in such a form is more like a \abbrev{FSM} with data path, but only one state. In
such a setting, the actions are just labelled transitions from the one state to itself; they only
differ in their conditions and update operations. The relevant information for execution of the
system is then just the sequence of transition names, not of states. The data-path state is encoded
in the state variables; such a system could in principle be transformed into a real state machine,
but would soon suffer from state-space explosion (or turns out to be impossible, since the states
covered by an Action System are not even necessarily finite). 

A more accurate, though also more formal description of Action Systems (in the full original variant
as well as the simplified one used here) can be given via Labelled Transition Systems
~\cites[p.~11]{tappler2015:symbolic}[p.~21]{jobstl2014:model-based}; this formalism is also used to express formally the conformance
relations between different models (test models, or mutated models of one original model), and is
used practically as an intermediate representation in various conformance checking algorithms.

The Action Systems which can be described by \actium{} are exactly of this nature. The basic style
of the syntax is inspired by the one mentioned above; however, some keywords are taken from the
Gherking specification
language\footnote{\protect\url{https://github.com/cucumber/cucumber/wiki/Gherkin} (visited on
  2015-06-06)}, and some things had of course to be changed due to Scala's native syntax. (The idea
of using Gherkin-like syntax for a Scala \dsl{} is not new: it exists already at least in the
|FeatureSpec| test style of
ScalaTest.\footnote{\protect\url{http://doc.scalatest.org/2.2.4/\#org.scalatest.FeatureSpec}
  (visited on 2015-06-24)})%


%%% Local Variables: 
%%% TeX-master: "document"
%%% End: